\documentclass[a4paper]{article}
\usepackage[final]{microtype}
\usepackage{fontspec}
\setmonofont{inconsolata}
\usepackage{amsmath,amsthm,amssymb,stmaryrd,mathtools,biblatex,forester}
\addbibresource{forest.bib}

\title{Homogeneous Model}

\date{September 19, 2024}

\author{Ruohan Hu}

\begin{document}
\maketitle
\par{
    A \ForesterRef{BMT-d002}{structure} \(M\) is \(\lambda\)-homogeneous, for a cardinal \(\lambda\), 
    when for every cardinal \(\eta < \lambda\) and sequences \((a_i \in  M|i \in \eta ), \{ b_i \in  M|i \in \eta \}\) such that \((M,(a_i)_{i< \eta }) \equiv (M,(b_i)_{i< \eta })\), 
    for each \(c \in  M\) there is \(d \in  M\) such that \((M,(a_i)_{i< \eta },c) \equiv (M,(b_i)_{i< \eta },d)\). 

    Here \((M,(a_i)_{i< \eta }),(M,(b_i)_{i< \eta })\) refers to \(M\) with its language expanded by a constant \(C_i\) for each \(i< \eta\), such that \(C_i\) is interpreted as \(a_i\), \(b_i\) respectively. 
    \((M,(a_i)_{i< \eta },c)\) and \((M,(b_i)_{i< \eta },d)\) are further expansions, when the language is further expanded by a constant \(D\) that is interpreted as \(c\) or \(d\) respectively.

    Equivalently, \(M\) is \(\lambda\)-homogeneous if every partial elementary embedding \(\Phi :M \to  M\) such that \(|dom \Phi |< \lambda\) extends to another partial elementary mapping of \(M\) to itself with a larger domain.

    The first definition follows Chang & Keisler, and the second follows Shelah. The two definitions are equivalent.
}
\printbibliography
\end{document}