\documentclass[a4paper]{article}
\usepackage[final]{microtype}
\usepackage{fontspec}
\setmonofont{inconsolata}
\usepackage{amsmath,amsthm,amssymb,stmaryrd,mathtools,biblatex,forester}
\addbibresource{forest.bib}

\title{Quantifier Elimination}

\date{February 8, 2024}

\author{Connor Lockhart\thanks{With contributions from Oscar Coppola.}}

\begin{document}
\maketitle
\par{Quantifier Elimination is a part of a broader technique in Model Theory where for a structure in a given language, an arbitrary formula can be written as a boolean combination of perhaps simpler formulas}
\begin{tree}{title={Elimination Set}, taxon={definition}, slug={BMT-d014}}
An \emph{Elimination Set} for a language \(\mathcal {L}\) and class \(K\) of \(\mathcal {L}\)-structures, then a set \(\Gamma\) of formulas \(\phi\) is an elimination set for \(K\) if for every formula \(\phi ( \bar {x})\) of \(\mathcal {L}\) there is a formula \(\phi ^*( \bar {x})\) which is a boolean combinations of formulas in \(\Gamma\) and \(\phi\) is equivalent to \(\phi ^*\) in every structure in \(K\)
\end{tree}
\par{In particular, we will be most interested in elimination sets that are comprised of the set of quantifier free formulas. It is worth noting that, in some cases it is not possible to have a full quantifier elimination down to the level of a quantifier free set but perhaps we can restrict ourself to some reasonable set of formulas.}\par{Frequently, we use the following lemma to simplify the task of showing that \(\Phi\) is an elimination set}
\begin{tree}{title={Elimination of Atomic Formulas}, taxon={lemma}, slug={BMT-d015}}
Let \(K\) be a class of \(L\) structures and \(\Phi\) be a set of \(L\)-formulas. Denote \(\Phi ^-\) as the set of negations of formulas in \(\Phi\).\par{Suppose that 
  \begin{enumerate}
\item{every atomic formula of \(L\) is in \(\Phi\) and} 
    \item{for every formula \(\theta ( \overline {x})\) of \(L\) which is of the form \(\exists  y  \bigwedge _{i<n}  \psi _i ( \overline {x},y)\) with each \(\psi _i  \in   \Phi \cup   \Phi ^-\), there is an \(L\)-formula \(\theta ^*( \overline {x})\) which is a boolean combination of formulas in \(\Phi\) and is equivalent to \(\theta\) in every structure in \(K\).}
\end{enumerate}
  Then \(\Phi\) is an elimination set of \(K\)}
\end{tree}
\par{The art of quantifier elimination lies in choosing an appropriate elimination set that allows the above lemma to be used with minimal obstructions. }\par{There are many examples of theories which have quantifier elimination for a relatively tame set of formulas: }
  
  
\begin{tree}{title={Algebraically Closed Fields}, taxon={example}, slug={BMT-e031}}

    The theory of \emph{algebraically closed fields} (ACF) is a \ForesterRef{BMT-d017}{theory} in the \ForesterRef{BMT-d001}{language} \(\mathcal  L= \{ 0,1,+, \cdot \}\)
    consisting of the following \ForesterRef{BMT-d013}{sentences}.
    \begin{enumerate}
\item{\(\forall  x \forall  y \forall  z \, (x+y)+z=x+(y+z)\).
        }
        \item{\(\forall  x \, x+0=x \land0 +x=x\).
        }
        \item{\(\forall  x \exists  y \, x+y=0 \land  y+x=0\).
        }
        \item{\(\forall  x \forall  y \, x+y=y+x\).
        }
        \item{\(\forall  x \forall  y \forall  z \, (x \cdot  y) \cdot  z=x \cdot  (y+ \cdot  z)\).
        }
        \item{\(\forall  x \, x \cdot1 =x \land1 \cdot  x=x\).
        }
        \item{\(\forall  x \exists  y \, x \cdot  y=1 \land  y \cdot  x=1\).
        }
        \item{\(\forall  x \forall  y \, x+ \cdot  y=y \cdot  x\).
        }
        \item{\(\forall  x \forall  y \forall  z \, x \cdot (y+z)=(x \cdot  y)+(x \cdot  z)\).
        }
        \item{
            For each \(n \in \mathbb  N\), a sentence \(\forall  y_1 \cdots \forall  y_n \exists  x \, (y_1 \cdot  x)+ \cdots +(y_n \cdot  x)=0\).
        }
\end{enumerate}\par{
    The theory of algebraically closed fields of characteristic \(p\) (\(\text {ACF}_p\)) is ACF together with the sentence
    \(1+ \cdots +1=0\), where \(1\) is being added \(p\) times. \(\text {ACF}_p\) is a \ForesterRef{BMT-d017}{complete theory} having
    \ForesterRef{BMT-s003}{quantifier elimination}.
}\par{
    The theory of algebraically closed fields of characteristic \(0\) (\(\text {ACF}_0\)) is ACF together with the collection of sentences
    of the form \(\neg (1+ \cdots +1=0)\), for any number of \(1\)s being added together. \(\text {ACF}_0\) is also a
    \ForesterRef{BMT-d017}{complete theory} having \ForesterRef{BMT-s003}{quantifier elimination}.
}
\end{tree}


  
  
\begin{tree}{title={DLO}, taxon={example}, slug={BMT-e030}}

    The \emph{dense linear order without endpoints} (DLO) is a \ForesterRef{BMT-d017}{theory} in the \ForesterRef{BMT-d001}{language} \(\mathcal  L= \{ < \}\)
    consisting of the following \ForesterRef{BMT-d013}{sentences}.
    \begin{enumerate}
\item{\(\forall  x \, \neg  x<x\)}
        \item{\(\forall  x \forall  y \, (x<y \rightarrow \neg (y<x))\)}
        \item{\(\forall  x \forall  y \forall  z \, ((x<y \land  y<z) \rightarrow  x<z)\)}
        \item{\(\forall  x \forall  y \, (( \neg  x=y) \rightarrow (x<y \lor  y<x))\)}
        \item{\(\forall  x \forall  y \, x<y \rightarrow ( \exists  z \, x<z \land  z<y)\)}
        \item{\(\forall  x \exists  z \, z<x\)}
        \item{\(\forall  x \exists  z \, x<z\)}
\end{enumerate}\par{
    The set of rationals \(\mathbb  Q\) with the usual ordering is a model of DLO, and DLO is \ForesterRef{BMT-d201}{\(\aleph _0\)-categorical},
    so every countable model is isomorphic to \(( \mathbb  Q,<)\). DLO is also a \ForesterRef{BMT-d017}{complete theory} with
    \ForesterRef{BMT-s003}{quantifier elimination}.
}
\end{tree}

\par{Furthermore, if one is willing to accept arbitrary expansions of the language, we can always force a theory to have quantifier elimination. This is done as follows:}
\begin{tree}{title={Morleyisation}, taxon={definition}, slug={BMT-d026}}
 The Morleyisation of an \(L\)-theory \(T\) is the theory \(T^m\) in the language \(L^m \supset  L\). The expanded language \(L^m\) is formed by taking every \(L\)-formula \(\phi (x_1,...,x_n)\) and adding an \(n\)-placed relation symbol \(R_ \phi\). To \(T\) we add the axioms \par{\(\forall  x_1,...,x_n (R_ \phi  (x_1,...,x_n) \leftrightarrow \phi (x_1,...,x_n)) \)}\par{Morleyisation preserves many properties such as \ForesterRef{BMT-d201}{\(\kappa\)-categoricity}, though the relations \(R_ \phi\) and the corresponding definable sets may be quite hard to understand. }
\end{tree}

\printbibliography
\end{document}