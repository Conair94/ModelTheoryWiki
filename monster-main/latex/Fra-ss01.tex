\documentclass[a4paper]{article}
\usepackage[final]{microtype}
\usepackage{fontspec}
\setmonofont{inconsolata}
\usepackage{amsmath,amsthm,amssymb,stmaryrd,mathtools,biblatex,forester}
\addbibresource{forest.bib}

\title{Ramsey Properties}

\date{April 23, 2024}

\author{Morgan Bryant}

\begin{document}
\maketitle
\par{We define different Ramsey properties on classes. }
\begin{tree}{title={Ramsey Property on Substructures (RPS)}, taxon={definition}, slug={Fra-d007}}
For structures \(A,B\), let \(\binom {B}{A} =  \{ A_0: A_0 \leq  B & ; A_0 \cong  B \}\)\par{Given a class and relation \((K, \leq )\), we say that \(K\) has the Ramsey property on substructures if for every \(A \leq  B  \in  K\), there is some \(C \in  K\) with \(B \leq  C\)
such that for every \(k\)-coloring \(c:  \binom {C}{A} \rightarrow  k\), there is some \(B'  \in   \binom {C}{B}\) with \[c|_{ \binom {B'}{A}}\] constant.}\par{Ramsey classes are highly related to descriptive set theory, in particular, extremely amenable sets.}\par{Take care not to mix up RPS and \ForesterRef{Fra-d008}{RPE}}
\end{tree}

\begin{tree}{title={Ramsey Property on Embeddings (RPE)}, taxon={definition}, slug={Fra-d008}}
For structures \(A,B\), let \(\binom {B}{A} =  \{ f:A \rightarrow  B : f  \;   \text {is an embedding } \}\)\par{Given a class and relation \((K, \leq )\), we say that \(K\) has the Ramsey property on embeddings if for every \(A \leq  B  \in  K\), there is some \(C \in  K\) with \(B \leq  C\)
such that for every \(k\)-coloring \(c:  \binom {C}{A} \rightarrow  k\), there is some \(B'  \in   \binom {C}{B}\) with \[c|_{ \binom {B'}{A}}\] constant.}\par{Ramsey classes are highly related to descriptive set theory, in particular, extremely amenable sets.}\par{Take care not to mix up \ForesterRef{Fra-d007}{RPS} and RPE}\par{The fundamental difference between RPS and RPE is that \(RPE  \Rightarrow  RPS\), but the reverse implication is not true. Notice that if 
a class has RPE, then all structures are \ForesterRef{Fra-d009}{rigid}. There exist classes of structures which have RPS but not RPE, as they contain non-rigid structures. 
e.g., The class of structures with an equivalence relation, where the class is ordered by \(\subset\).}
\end{tree}

\begin{tree}{title={Rigid Structures}, taxon={definition}, slug={Fra-d009}}
A structure \(A\) in a language \(\mathcal {L}\) is called \emph{rigid} if \(Aut(A)\) is trivial. I.e., the only automorphism of \(A\) is the identity.\par{Ex. Finite linear orders are rigid structures.}
\end{tree}

\begin{tree}{title={Indivisible Property (Class version) on Substructures (InvP)}, taxon={definition}, slug={Fra-d010}}
Given a class and relation \((K, \leq )\), we say that \(K\) has the indivisibility property on substructures if for every \(A  \in  K\), there is some \(C \in  K\)
such that for every \(k\)-coloring \(c: C \rightarrow  k\), there is some \(A' \leq  C\) such that \(A'  \cong  A\) with \[c|_{A'}\] constant.\par{Indivisibility is a weakening of the Ramsey property. In some explicit cases, RPS implies InvP.}\par{Citations Needed***}
\end{tree}

\begin{tree}{title={Indivisible Property (Class version) on Embeddings (InvPE)}, taxon={definition}, slug={Fra-d011}}
Given a class and relation \((K, \leq )\), we say that \(K\) has the indivisibility property on embeddings if for every \(A  \in  K\), there is some \(C \in  K\)
such that for every \(k\)-coloring \(c: C \rightarrow  k\), there is some strong embedding \(f:A \rightarrow  C\) such that \(c \circ  f\) is constant.\par{Citations Needed***}
\end{tree}

\begin{tree}{title={Order Class}, taxon={definition}, slug={Fra-d012}}
A class \(K\) in a language \(\mathcal {L}  \supseteq   \{ \leq \}\) is an \emph{Order Class} if for every \(A \in  K\), \(\leq\) is a linear ordering on \(A\)\par{Citations Needed***}
\end{tree}

\begin{tree}{title={Kechris-Pestov-Todorcevic Theorem}, taxon={theorem}, slug={Fra-t013}}
For an an \ForesterRef{Fra-d012}{order class} \(K\) which is Fraisse with limit \(M\), the following are equivalent:\par{Aut(M) is extremely amenable}\par{\(K\) has the Ramsey Property.}\par{Citations Needed***}
\end{tree}

\printbibliography
\end{document}