\documentclass[a4paper]{article}
\usepackage[final]{microtype}
\usepackage{fontspec}
\setmonofont{inconsolata}
\usepackage{amsmath,amsthm,amssymb,stmaryrd,mathtools,biblatex,forester}
\addbibresource{forest.bib}

\title{Saturation}

\date{February 8, 2024}

\author{Connor Lockhart \and Ruohan Hu\thanks{With contributions from Francis Westhead.}}

\begin{document}
\maketitle
\par{In this section we give the basic definitions and theorems for saturation of models.}
\begin{tree}{title={Saturated Model}, taxon={definition}, slug={BMT-d209}}

    A \ForesterRef{BMT-d002}{structure} \(M\) is \(\lambda\)-saturated, for a cardinal \(\lambda\), when for every \(A \subseteq  M\) such that \(|A|< \lambda\), every \ForesterRef{BMT-d019}{type} over \(A\) is realized in \(M\). \(M\) is simply said to be saturated if \(M\) is \(|M|\)-saturated.

\end{tree}

\begin{tree}{title={Homogeneous Model}, taxon={definition}, slug={BMT-d210}}

    A \ForesterRef{BMT-d002}{structure} \(M\) is \(\lambda\)-homogeneous, for a cardinal \(\lambda\), 
    when for every cardinal \(\eta < \lambda\) and sequences \((a_i \in  M|i \in \eta ), \{ b_i \in  M|i \in \eta \}\) such that \((M,(a_i)_{i< \eta }) \equiv (M,(b_i)_{i< \eta })\), 
    for each \(c \in  M\) there is \(d \in  M\) such that \((M,(a_i)_{i< \eta },c) \equiv (M,(b_i)_{i< \eta },d)\). 

    Here \((M,(a_i)_{i< \eta }),(M,(b_i)_{i< \eta })\) refers to \(M\) with its language expanded by a constant \(C_i\) for each \(i< \eta\), such that \(C_i\) is interpreted as \(a_i\), \(b_i\) respectively. 
    \((M,(a_i)_{i< \eta },c)\) and \((M,(b_i)_{i< \eta },d)\) are further expansions, when the language is further expanded by a constant \(D\) that is interpreted as \(c\) or \(d\) respectively.

    Equivalently, \(M\) is \(\lambda\)-homogeneous if every partial elementary embedding \(\Phi :M \to  M\) such that \(|dom \Phi |< \lambda\) extends to another partial elementary mapping of \(M\) to itself with a larger domain.

    The first definition follows Chang & Keisler, and the second follows Shelah. The two definitions are equivalent.

\end{tree}

\begin{tree}{title={Strongly Homogeneous Model}, taxon={definition}, slug={BMT-d212}}

    A model \(M\) is strongly \(\lambda\)-homogeneous, if for every partial elementary embedding \(\Phi :M \to  M\) extends to an automorphism of \(M\). 

    Clearly, strongly \(\lambda\)-homogeneous implies \ForesterRef{BMT-d210}{\(\lambda\)-homogeneous}. 

\end{tree}

\begin{tree}{title={Ultrahomogeneous Model}, taxon={definition}, slug={BMT-d211}}

    A countable \ForesterRef{BMT-d002}{structure} \(M\) is ultrahomogeneous, per Hodges, when every \ForesterRef{BMT-d005}{isomorphism} between two finite \ForesterRef{BMT-d006}{substructures} of \(M\) extends to a global \ForesterRef{BMT-d005}{automorphism}.

    Also, ultrahomogeneous is strictly stronger than strongly \(\omega\)-homogeneous. For a strongly \(\omega\)-homogeneous structure that is not ultrahomogeneous, consider the disjoint union of a countable \ForesterRef{Fra-e001}{random graph} and a countable triangle-free random graph. This structure is \(\omega\)-homogenous but not ultrahomogeneous.

\end{tree}

\begin{tree}{title={Strongly Saturated Model}, taxon={definition}, slug={BMT-d213}}
{



}\par{
    A structure \(M\) is strongly \(\lambda\)-saturated if it is \ForesterRef{BMT-d209}{\(\lambda\)-saturated} and \ForesterRef{BMT-d212}{strongly \(\lambda\)-homogeneous}.
}
\end{tree}

\begin{tree}{title={Universal Model}, taxon={definition}, slug={BMT-d214}}

    A model \(M\) is \(\lambda\)-Universal, per Chang & Keisler, when for every model \(N\) such that \(|N|< \lambda\) that is elementarily equivalent to \(M\) can \ForesterRef{BMT-d030}{elementarily embeds} into \(M\). 
\par{
    A model \(M\) is \(\lambda\)-Universal, per Shelah, when for every model \(N\) that is elementarily equivalent to \(M\), and a set \(A \subseteq  N\) such that \(|A| \leq \lambda\), there is a \ForesterRef{BMT-d030}{partial elementary embedding} \(f:M \to  N\) such that \(dom(f)=A\).
}\par{\(\lambda\)-universality per Shelah implies \(\lambda ^+\)-universality per Chang & Keisler.
    By the Lowenheim-Skolem Theorem, if the language \(|L| \leq   \lambda\), then \(\lambda\)-universality per Shelah and \(\lambda ^+\)-universality per Chang & Keisler are equivalent.
}\par{
    Per Shelah, a model \(M\) is \(< \lambda\)-universal if it is \(\mu\)-universal for all \(\mu < \lambda\), and a model \(M\) simply said to be universal if it is \(|M|\)-universal. 
}
\end{tree}

\begin{tree}{title={A model is homogeneous if and only if it is strongly homogeneous}, taxon={theorem}, slug={BMT-t081}}
A back-and-forth argument establishes that if \(\mathcal {M}\) has infinite cardinality, then \(\mathcal {M}\) is homogeneous
if and only if \(\mathcal {M}\) is strongly homogeneous.
\end{tree}

\begin{tree}{title={Saturated elementarily equivalent models of the same cardinality are isomorphic.}, taxon={theorem}, slug={BMT-t082}}

    If \(\mathcal {M}\) and \(\mathcal {N}\) are each saturated, \(| \mathcal {M}|=| \mathcal {N}|\) and \(Th( \mathcal {M})=Th( \mathcal {N})\)
, then \(\mathcal {M}\) and \(\mathcal {N}\) are isomorphic.
\par{
    Moreover, every partial elementary embedding \(f:M \to  N\) extends to an isomorphism. 
    Therefore, for every saturated model \(M\), a subset \(A \subseteq  M\), and two elements \(x,y \in  M\) of the same \ForesterRef{BMT-d019}{type} over \(A\), there is an automorphism of \(M\) fixing \(A\) pointwise and takes \(x\) to \(y\).
    This is sometimes known as the Model-Theoretic Galois Theory.
}
\end{tree}

\begin{tree}{title={\(\kappa\)-saturated models are \(\kappa ^+\)-universal}, taxon={theorem}, slug={BMT-t083}}
If \(\mathcal {M}\) is \(\kappa\)-saturated, then \(\mathcal {M}\) is \(\kappa ^+\)-universal.
\end{tree}

\begin{tree}{title={Saturation is Equivalent to Universality and Homogeneity}, taxon={theorem}, slug={BMT-t084}}

    The following are equivalent, given a model \(M\) and a cardinal \(\lambda\).
\par{
    1: \(M\) is \(\lambda\)-saturated.
}\par{
    2: \(M\) is \(\lambda\)-universal and \(\lambda\)-homogeneous.
}\par{
    3: \(M\) is \(< \aleph _0\)-universal and \(\lambda\)-homogeneous.
}
\end{tree}

\printbibliography
\end{document}