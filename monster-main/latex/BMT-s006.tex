\documentclass[a4paper]{article}
\usepackage[final]{microtype}
\usepackage{fontspec}
\setmonofont{inconsolata}
\usepackage{amsmath,amsthm,amssymb,stmaryrd,mathtools,biblatex,forester}
\addbibresource{forest.bib}

\title{Saturation}

\date{February 8, 2024}

\author{Connor Lockhart \and Ruohan Hu}

\begin{document}
\maketitle
\par{In this section we give the basic definitions and theorems for saturation of models.}
\begin{tree}{title={Saturated Model}, taxon={definition}, slug={BMT-d209}}

    A \ForesterRef{BMT-d002}{structure} \(M\) is \(\lambda\)-saturated, for a cardinal \(\lambda\), when for every \(A \subseteq  M\) such that \(|A|< \lambda\), every \ForesterRef{BMT-d019}{type} over \(A\) is realized in \(M\). \(M\) is simply said to be saturated if \(M\) is \(|M|\)-saturated.

\end{tree}

\begin{tree}{title={Saturated Model}, taxon={definition}, slug={BMT-d210}}

    A \ForesterRef{BMT-d002}{structure} \(M\) is \(\lambda\)-homogeneous, for a cardinal \(\lambda\), 
    when for every cardinal \(\eta < \lambda\) and sequences \((a_i \in  M|i \in \eta ), \{ b_i \in  M|i \in \eta \}\) such that \((M,(a_i)_{i< \eta }) \equiv (M,(b_i)_{i< \eta })\), 
    for each \(c \in  M\) there is \(d \in  M\) such that \((M,(a_i)_{i< \eta },c) \equiv (M,(b_i)_{i< \eta },d)\). 

    Here \((M,(a_i)_{i< \eta }),(M,(b_i)_{i< \eta })\) refers to \(M\) with its language expanded by a constant \(C_i\) for each \(i< \eta\), such that \(C_i\) is interpreted as \(a_i\), \(b_i\) respectively. 
    \((M,(a_i)_{i< \eta },c)\) and \((M,(b_i)_{i< \eta },d)\) are further expansions, when the language is further expanded by a constant \(D\) that is interpreted as \(c\) or \(d\) respectively.

    The definition here follows Chang & Keisler, and is equivalent to the definition in Shelah.

\end{tree}

\begin{tree}{title={Saturated Model}, taxon={definition}, slug={BMT-d211}}

    A countable \ForesterRef{BMT-d002}{structure} \(M\) is ultrahomogeneous, per Hodges, when every \ForesterRef{BMT-d005}{isomorphism} between two finite \ForesterRef{BMT-d006}{substructure} of \(M\) extends to a global \ForesterRef{BMT-d005}{automorphism}.

    Ultrahomogeneous is strictly stronger than \(\omega\)-\ForesterRef{BMT-d210}{homogeneous}. For an \(\omega\)-\ForesterRef{BMT-d210}{homogeneous} structure that is not ultrahomogeneous, consider the disjoint union of a countable \ForesterRef{Fra-e001}{random graph} and a countable triangle-free random graph.

\end{tree}

\begin{tree}{title={}, taxon={}, slug={BMT-d212}}

\end{tree}

\begin{tree}{title={}, taxon={}, slug={BMT-d213}}

\end{tree}

\printbibliography
\end{document}