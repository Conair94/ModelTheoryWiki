\documentclass[a4paper]{article}
\usepackage[final]{microtype}
\usepackage{fontspec}
\setmonofont{inconsolata}
\usepackage{amsmath,amsthm,amssymb,stmaryrd,mathtools,biblatex,forester}
\addbibresource{forest.bib}

\title{Dividing lines}

\date{February 2, 2024}

\author{Adam Melrod}

\begin{document}
\maketitle
\par{Dividing lines play a fundamental role in model theory. They form the basis of Shelah's approach to classification theory and have been a central influence on model theory since their conception. Here is an interactive (though not complete) map of the \href{https://forkinganddividing.com}{model theoretic universe}, maintained by Gabriel Conant. Important examples of dividing lines include stability, NIP, and o-minimality.}
\begin{tree}{title={NIP Theories}, taxon={section}, slug={mon-s001}}
\textbf{NIP theories} are a class of theories generalizing stable
theories, but allowing for an ordering. Aside from stable theories,
important examples are real closed fields, ACVF, \(p\)-adically closed
fields, and o-minimal theories\par{Fix a complete theory \(T\) with monster model \(\mathbb {M}\).
Also fix a formula \(\varphi (x;y)\) with a fixed partitioning into
the two tuples \(x\) and \(y\).}
\begin{tree}{title={Definition of the independence property}, taxon={definition}, slug={mon-d004}}
The formula \(\varphi (x;y)\) has the \emph{independence property} if there
are sequences of tuples \((a_i : i  \in   \omega )\) and
\((b_S : S  \subseteq   \omega )\) such that for every subset \(S  \subseteq   \omega\)
\[i  \in  S  \Longleftrightarrow   \mathbb {M}  \models   \varphi (a_i; b_S).\]
\end{tree}

\begin{tree}{title={Definition of NIP}, taxon={definition}, slug={mon-d005}}
The formula \(\varphi (x;y)\) is said to be \emph{NIP} if it does not have
the \href{mon-0004}{independence property}.
\end{tree}

\begin{tree}{title={Characterizations of NIP for a formula}, taxon={theorem}, slug={mon-t001}}
The following conditions are equivalent:
\begin{enumerate}
\item{ The formula \(\varphi (x;y)\) has the independence property.}
  \item{ The formula \(\varphi ^{ \vee }(y;x)\) has the independence property,
	where \(\varphi ^{ \vee }(y;x)\) is the formula \(\varphi (x;y)\) with the
	opposite partition.}
	\item{ For any two finite sets \(U\) and \(V\), and any subset \(R  \subseteq  U  \times  V\), there are \((a_i : i  \in  U)\) and \((b_j : j  \in  U)\) such that \(\mathbb   \models   \varphi (a_i;b_j)  \Longleftrightarrow  (i,j)  \in  R\).}
	\item{ There is an indiscernible sequence \((a_i : i  \in   \omega )\)
	and some \(b\) such that \(\mathbb  M  \models   \varphi (a_i;b)  \Longleftrightarrow  i\)
	is even.}
\end{enumerate}\par{We are also interested in the following characterization, which is more amenable to computations.}
\begin{tree}{title={Alternation Number}, taxon={definition}, slug={mon-d007}}
The \emph{alternation number} of a formula \(\varphi (x;y)\), denoted \(\operatorname {alt}( \varphi (x;y))\) is the maximal number \(n  \in   \omega\) (if it exists) such that there is an indiscernible sequence \((a_i : i  \in   \omega )\), some \(b\), and indices \(i_0 <  \dots  < i_n\) with \(\mathbb  M  \models   \varphi (a_i,b)  \Longleftrightarrow  i  \text { is even}\). If no such maximum exists, we let \(\operatorname {alt}( \varphi (x;y)) =  \infty\).
\end{tree}

\begin{tree}{title={Alternation Lemma}, taxon={lemma}, slug={mon-d008}}
A formula \(\varphi (x;y)\) is NIP if and only if \(\operatorname {alt}( \varphi (x;y)) <  \infty\).
\end{tree}

\end{tree}

\end{tree}

\printbibliography
\end{document}