\documentclass[a4paper]{article}
\usepackage[final]{microtype}
\usepackage{fontspec}
\setmonofont{inconsolata}
\usepackage{amsmath,amsthm,amssymb,stmaryrd,mathtools,biblatex,forester}
\addbibresource{forest.bib}

\title{Definition of an \(\mathcal {L}\)-theory}

\date{May 2, 2024}

\author{Francis Westhead \and Oscar Coppola}

\begin{document}
\maketitle
\par{
    Given a \ForesterRef{BMT-d001}{language} \(\mathcal {L}\), a \emph{\(\mathcal {L}\)-theory} is a set of \ForesterRef{BMT-d013}{\(\mathcal {L}\)-sentences}.
    Sometimes it is notationally convenient to assume that theories are deductively closed.
}\par{
    An \(\mathcal  L\)-theory is \emph{complete} when for every \ForesterRef{BMT-d009}{\(\mathcal  L\)-formula} \(\phi\) we have that
    \(\phi\) may be deduced from \(T\) or \(\neg \phi\) may be deduced from \(T\).
}
\printbibliography
\end{document}