\documentclass[a4paper]{article}
\usepackage[final]{microtype}
\usepackage{fontspec}
\setmonofont{inconsolata}
\usepackage{amsmath,amsthm,amssymb,stmaryrd,mathtools,biblatex,forester}
\addbibresource{forest.bib}

\title{The Stone Space #[S_n(T)]}

\date{March 10, 2024}

\author{Francis Westhead}

\begin{document}
\maketitle
\par{Given a \ForesterRef{BMT-d001}{language}, \(\mathcal {L}\), and a \ForesterRef{BMT-d017}{\(\mathcal {L}\)-theory \(T\)},
\emph{#[S_n(T)]} denotes a topological space whose underlying set is the set of all complete \(n-types\) of \(T\).
The topology on \(S_n(T)\) has a basis of open sets given by all sets of the form \([ \varphi (x)] :=  \{ p(x) \in  S_n(T):
 \varphi (x)  \in  p(x) \}\) (for \(\varphi (x)\) an \(\mathcal {L}\)-formula).
This space is a Stone Space in the usual sense and the Boolean Algebra associated to it via Stone Duality is
the \(n\)'th Tarski Linedenbaum Algebra of \(T\).}
\printbibliography
\end{document}