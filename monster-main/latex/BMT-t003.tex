\documentclass[a4paper]{article}
\usepackage[final]{microtype}
\usepackage{fontspec}
\setmonofont{inconsolata}
\usepackage{amsmath,amsthm,amssymb,stmaryrd,mathtools,biblatex,forester}
\addbibresource{forest.bib}

\title{(Ryll-Nardjewski) Characterisations of \(\omega\)-categoricity}

\date{April 10, 2024}

\author{Francis Westhead}

\begin{document}
\maketitle
\par{For \(\aleph _0\)-categoricity, there are a number of useful characterisations. The following are due to Ryll-Nardjewski.}\par{Given a \ForesterRef{BMT-d001}{language}, \(\mathcal {L}\), and an \ForesterRef{BMT-d017}{\(\mathcal {L}\)-theory} \(T\), the following are equivalent:}\par{1 \(T\) is \(\aleph _0\)-categorical.}\par{For every \(n \in   \omega\), there are finitely many \ForesterRef{BMT-d009}{\(\mathcal {L}\)-formulas} in \(n\)-variables up to \ForesterRef{BMT-d018}{\(T\)-equivalence}.}\par{For every \(n \in   \omega\), the \ForesterRef{BMT-d202}{\(n\)'th Tarski-Lindenbaum algebra of \(T\)} is finite. 
Every \ForesterRef{BMT-d019}{type} over \(T\) is \ForesterRef{BMT-d020}{isolated}. }
\printbibliography
\end{document}