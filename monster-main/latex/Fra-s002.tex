\documentclass[a4paper]{article}
\usepackage[final]{microtype}
\usepackage{fontspec}
\setmonofont{inconsolata}
\usepackage{amsmath,amsthm,amssymb,stmaryrd,mathtools,biblatex,forester}
\addbibresource{forest.bib}

\title{Other Properties of Classes}

\date{February 17, 2024}

\author{Morgan Bryant}

\begin{document}
\maketitle
\par{We define other common properties and variants of the set up shown in the \ForesterRef{Fra-s001}{Fraisse section}}
\begin{tree}{title={Definition of Disjoint Amalgamation Property (dAP)}, taxon={definition}, slug={Fra-d006}}
Given a class and relation \((K, \leq )\), we say that \(K\) has the \emph{Disjoint Amalgamation Property} if for every \(A,B,C  \in  K\) such that \(B \cap  C =A\), if there 
exists an embedding \(f:A \rightarrow  B\) such that \(f(A)  \leq  B\) and and embedding \(g:A \rightarrow  C\) such that \(A \leq  B\), then there exists a structure \(D\) 
and embeddings \(h_1:C \rightarrow  D\) and \(h_2:B \rightarrow  D\) such that \(h_1(C) \leq  D\) and \(h_2(B)  \leq  D\), \(h_2(f(A)) = h_1(g(A))\), and \(h_2(C)  \cap  h_1(B) = h_1(A)\)\par{An example of a class which has the dAP is the \ForesterRef{Fra-e001}{Fraisse class of finite graphs}}\par{An example of a smooth class which does not have the dAP is \ForesterRef{Smc-e001}{The class of initial segments}}
\end{tree}

\printbibliography
\end{document}