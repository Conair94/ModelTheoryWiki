\documentclass[a4paper]{article}
\usepackage[final]{microtype}
\usepackage{fontspec}
\setmonofont{inconsolata}
\usepackage{amsmath,amsthm,amssymb,stmaryrd,mathtools,biblatex,forester}
\addbibresource{forest.bib}

\title{Categoricity}

\date{March 10, 2024}

\author{Francis Westhead}

\begin{document}
\maketitle
\par{In this section, we give the basic definitions for \(\kappa\)-categoricity. We then establish some classical characterisations of \(\aleph _0\)-categoricity.}
\begin{tree}{title={\(\kappa\)-categoricity}, taxon={definition}, slug={BMT-d201}}

    Given a cardinal \(\kappa\) and a \ForesterRef{BMT-d001}{language} \(\mathcal {L}\), an \ForesterRef{BMT-d017}{\(\mathcal {L}\)-theory} is \emph{\(\kappa\)-categorical} if whenever \(\mathcal {M}\) and \(\mathcal {N}\) are \(\mathcal {L}\)-structures with \(| \mathcal {M}|=| \mathcal {M}|= \kappa\), then \(\mathcal {M}  \cong   \mathcal {N}\). 

\end{tree}

\begin{tree}{title={Tarski-Lindenbaum Algebra}, taxon={definition}, slug={BMT-d202}}
Given a \ForesterRef{BMT-d001}{language}, \(\mathcal {L}\), and an \ForesterRef{BMT-d017}{\(\mathcal {L}\)-theory}, \(T\), the \emph{\(n\)'th Tarski-Lindenbaum algebra of \(T\)} is the set of \ForesterRef{BMT-d009}{\(\mathcal {L}\)-formulas} quotiented by the relation of \ForesterRef{BMT-d018}{\(T\)-equivalence}.
\end{tree}

\begin{tree}{title={(Ryll-Nardjewski) Characterisations of \(\omega\)-categoricity}, taxon={theorem}, slug={BMT-t003}}
For \(\aleph _0\)-categoricity, there are a number of useful characterisations. The following are due to Ryll and Nardjewski.\par{Given a \ForesterRef{BMT-d001}{language}, \(\mathcal {L}\), and an \ForesterRef{BMT-d017}{\(\mathcal {L}\)-theory} \(T\), the following are equivalent:}\par{1 \(T\) is \(\aleph _0\)-categorical.}\par{For every \(n \in   \omega\), there are finitely many \ForesterRef{BMT-d009}{\(\mathcal {L}\)-formulas} in \(n\)-variables up to \ForesterRef{BMT-d018}{\(T\)-equivalence}.}\par{For every \(n \in   \omega\), the \ForesterRef{BMT-d202}{\(n\)'th Tarski-Lindenbaum algebra of \(T\)} is finite. 
Every \ForesterRef{BMT-d019}{type} over \(T\) is \ForesterRef{BMT-d020}{isolated}. }
\end{tree}

\printbibliography
\end{document}