\documentclass[a4paper]{article}
\usepackage[final]{microtype}
\usepackage{fontspec}
\setmonofont{inconsolata}
\usepackage{amsmath,amsthm,amssymb,stmaryrd,mathtools,biblatex,forester}
\addbibresource{forest.bib}

\title{Examples}

\date{April 25, 2024}

\author{Oscar Coppola}

\begin{document}
\maketitle
\par{
    Here, we give the basic and canonical examples of Model Theory, summarizing their properties.
}
\begin{tree}{title={DLO}, taxon={example}, slug={BMT-e030}}

    The \emph{dense linear order without endpoints} (DLO) is a \ForesterRef{BMT-d017}{theory} in the \ForesterRef{BMT-d001}{language} \(\mathcal  L= \{ < \}\)
    consisting of the following (sentences)[BMT-d013].
    \begin{enumerate}
\item{\(\forall  x \, \neg  x<x\)}
        \item{\(\forall  x \forall  y \, (x<y \rightarrow \neg (y<x))\)}
        \item{\(\forall  x \forall  y \forall  z \, ((x<y \land  y<z) \rightarrow  x<z)\)}
        \item{\(\forall  x \forall  y \, (( \neg  x=y) \rightarrow (x<y \lor  y<x))\)}
        \item{\(\forall  x \forall  y \, x<y \rightarrow ( \exists  z \, x<z \land  z<y)\)}
        \item{\(\forall  x \exists  z \, z<x\)}
        \item{\(\forall  x \exists  z \, x<z\)}
\end{enumerate}\par{
    The set of rationals \(\mathbb  Q\) with the usual ordering is a model of DLO, and DLO is \ForesterRef{BMT-d201}{\(\aleph _0\)-categorical},
    so every countable model is isomorphic to \(( \Q ,<)\). DLO is also a \ForesterRef{BMT-d017}{complete theory} with
    \ForesterRef{BMT-s003}{quantifier elimination}.
}
\end{tree}

\begin{tree}{title={Algebraically Closed Fields}, taxon={example}, slug={BMT-e031}}

    The theory of \emph{algebraically closed fields} (ACF) is a \ForesterRef{BMT-d017}{theory} in the \ForesterRef{BMT-d001}{language} \(\mathcal  L= \{ 0,1,+, \cdot \}\)
    consisting of the following (sentences)[BMT-d013].
    \begin{enumerate}
\item{\(\forall  x \forall  y \forall  z \, (x+y)+z=x+(y+z)\).
        }
        \item{\(\forall  x \, x+0=x \land0 +x=x\).
        }
        \item{\(\forall  x \exists  y \, x+y=0 \land  y+x=0\).
        }
        \item{\(\forall  x \forall  y \, x+y=y+x\).
        }
        \item{\(\forall  x \forall  y \forall  z \, (x \cdot  y) \cdot  z=x \cdot  (y+ \cdot  z)\).
        }
        \item{\(\forall  x \, x \cdot1 =x \land1 \cdot  x=x\).
        }
        \item{\(\forall  x \exists  y \, x \cdot  y=1 \land  y \cdot  x=1\).
        }
        \item{\(\forall  x \forall  y \, x+ \cdot  y=y \cdot  x\).
        }
        \item{\(\forall  x \forall  y \forall  z \, x \cdot (y+z)=(x \cdot  y)+(x \cdot  z)\).
        }
        \item{
            For each \(n \in \mathbb  N\), a sentence \(\forall  y_1 \cdots \forall  y_n \exists  x \, (y_1 \cdot  x)+ \cdots +(y_n \cdot  x)=0\).
        }
\end{enumerate}\par{
    The theory of algebraically closed fields of characteristic \(p\) (\(\text {ACF}_p\)) is ACF together with the sentence
    \(1+ \cdots +1=0\), where \(1\) is being added \(p\) times. \(\text {ACF}_p\) is a \ForesterRef{BMT-d017}{complete theory} having
    \ForesterRef{BMT-s003}{quantifier elimination}.
}\par{
    The theory of algebraically closed fields of characteristic \(0\) (\(\text {ACF}_0\)) is ACF together with the collection of sentences
    of the form \(\neg (1+ \cdots +1=0)\), for any number of \(1\)s being added together. \(\text {ACF}_0\) is also a
    \ForesterRef{BMT-d017}{complete theory} having \ForesterRef{BMT-s003}{quantifier elimination}.
}
\end{tree}

\printbibliography
\end{document}