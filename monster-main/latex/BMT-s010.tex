\documentclass[a4paper]{article}
\usepackage[final]{microtype}
\usepackage{fontspec}
\setmonofont{inconsolata}
\usepackage{amsmath,amsthm,amssymb,stmaryrd,mathtools,biblatex,forester}
\addbibresource{forest.bib}

\title{Examples}

\date{April 25, 2024}

\author{Oscar Coppola}

\begin{document}
\maketitle
\par{
    Here, we give the basic and canonical examples of Model Theory, summarizing their properties.
}
\begin{tree}{title={DLO}, taxon={example}, slug={BMT-e030}}

    The \emph{dense linear order without endpoints} (DLO) is a \ForesterRef{BMT-d017}{theory} in the \ForesterRef{BMT-d001}{language} \(\mathcal  L= \{ < \}\)
    consisting of the following \ForesterRef{BMT-d013}{sentences}.
    \begin{enumerate}
\item{\(\forall  x \, \neg  x<x\)}
        \item{\(\forall  x \forall  y \, (x<y \rightarrow \neg (y<x))\)}
        \item{\(\forall  x \forall  y \forall  z \, ((x<y \land  y<z) \rightarrow  x<z)\)}
        \item{\(\forall  x \forall  y \, (( \neg  x=y) \rightarrow (x<y \lor  y<x))\)}
        \item{\(\forall  x \forall  y \, x<y \rightarrow ( \exists  z \, x<z \land  z<y)\)}
        \item{\(\forall  x \exists  z \, z<x\)}
        \item{\(\forall  x \exists  z \, x<z\)}
\end{enumerate}\par{
    The set of rationals \(\mathbb  Q\) with the usual ordering is a model of DLO, and DLO is \ForesterRef{BMT-d201}{\(\aleph _0\)-categorical},
    so every countable model is isomorphic to \(( \mathbb  Q,<)\). DLO is also a \ForesterRef{BMT-d017}{complete theory} with
    \ForesterRef{BMT-s003}{quantifier elimination}.
}
\end{tree}

\begin{tree}{title={Algebraically Closed Fields}, taxon={example}, slug={BMT-e031}}

    The theory of \emph{algebraically closed fields} (ACF) is a \ForesterRef{BMT-d017}{theory} in the \ForesterRef{BMT-d001}{language} \(\mathcal  L= \{ 0,1,+, \cdot \}\)
    consisting of the following \ForesterRef{BMT-d013}{sentences}.
    \begin{enumerate}
\item{\(\forall  x \forall  y \forall  z \, (x+y)+z=x+(y+z)\).
        }
        \item{\(\forall  x \, x+0=x \land0 +x=x\).
        }
        \item{\(\forall  x \exists  y \, x+y=0 \land  y+x=0\).
        }
        \item{\(\forall  x \forall  y \, x+y=y+x\).
        }
        \item{\(\forall  x \forall  y \forall  z \, (x \cdot  y) \cdot  z=x \cdot  (y+ \cdot  z)\).
        }
        \item{\(\forall  x \, x \cdot1 =x \land1 \cdot  x=x\).
        }
        \item{\(\forall  x \exists  y \, x \cdot  y=1 \land  y \cdot  x=1\).
        }
        \item{\(\forall  x \forall  y \, x+ \cdot  y=y \cdot  x\).
        }
        \item{\(\forall  x \forall  y \forall  z \, x \cdot (y+z)=(x \cdot  y)+(x \cdot  z)\).
        }
        \item{
            For each \(n \in \mathbb  N\), a sentence \(\forall  y_1 \cdots \forall  y_n \exists  x \, (y_1 \cdot  x)+ \cdots +(y_n \cdot  x)=0\).
        }
\end{enumerate}\par{
    The theory of algebraically closed fields of characteristic \(p\) (\(\text {ACF}_p\)) is ACF together with the sentence
    \(1+ \cdots +1=0\), where \(1\) is being added \(p\) times. \(\text {ACF}_p\) is a \ForesterRef{BMT-d017}{complete theory} having
    \ForesterRef{BMT-s003}{quantifier elimination}.
}\par{
    The theory of algebraically closed fields of characteristic \(0\) (\(\text {ACF}_0\)) is ACF together with the collection of sentences
    of the form \(\neg (1+ \cdots +1=0)\), for any number of \(1\)s being added together. \(\text {ACF}_0\) is also a
    \ForesterRef{BMT-d017}{complete theory} having \ForesterRef{BMT-s003}{quantifier elimination}.
}
\end{tree}

\begin{tree}{title={Vector Spaces}, taxon={example}, slug={BMT-e032}}

    The theory of vector spaces over a field \(F\) is a \ForesterRef{BMT-d017}{theory} in the \ForesterRef{BMT-d001}{language} \(\mathcal  L= \{ 0,+,( \times _f)_{f \in  F} \}\)
    consisting of the following \ForesterRef{BMT-d013}{sentences}.
    \begin{enumerate}
\item{\(\forall  x \forall  y \forall  z \, (x+y)+z=x+(y+z)\).
        }
        \item{\(\forall  x \, x+0=x \land0 +x=x\).
        }
        \item{\(\forall  x \exists  y \, x+y=0 \land  y+x=0\).
        }
        \item{\(\forall  x \forall  y \, x+y=y+x\).
        }
        \item{
            For each \(f,g \in  F\), the sentence \(\forall  x \, \times _f( \times _g(x)) =  \times _{fg}(x)\), where \(fg\) is the product of \(f\) and \(g\) in \(F\).
        }
        \item{\(\forall  x \, \times _1(x) = x\) where \(1\) is the multiplicative identity in \(F\).
        }
        \item{
            For each \(f \in  F\), the sentence \(\forall  x \forall  y \, \times _f(x+y) =  \times _f(x)+ \times _f(y)\).
        }
        \item{
            For each \(f,g \in  F\), the sentence \(\forall  x \, \times _(f+g)(x) =  \times _f(x)+ \times _g(x)\), where \(f+g\) is the sum of \(f\) and \(g\) in \(F\).
        }
\end{enumerate}
\end{tree}

\begin{tree}{title={Equivalence Relation}, taxon={example}, slug={BMT-e033}}

    The theory of an equivalence relation is a \ForesterRef{BMT-d017}{theory} in the \ForesterRef{BMT-d001}{language} \(\mathcal  L= \{ E \}\),
    where \(E\) is a binary relation, consisting of the following \ForesterRef{BMT-d013}{sentences}.
    \begin{enumerate}
\item{\(\forall  x \, xEx\).
        }
        \item{\(\forall  x \forall  y \, xEy \rightarrow  yEx\).
        }
        \item{\(\forall  x \forall  y \forall  z \, (xEy \land  yEz) \rightarrow  xEz\).
        }
\end{enumerate}\par{
    Being uncomplicated, this theory is usually shown as an initial example (or rather, counterexample) of various properties.
    It is straightforward to add more restrictions, such as stating that there are only \(n\) equivalence classes:
    \begin{enumerate}
\item{\(\exists  x_1 \cdots \exists  x_n \forall  y \, \bigvee _{i=1}^nx_iEy\).
        }
\end{enumerate}
    Or, that each equivalence class has at most \(k\) elements:
    \begin{enumerate}
\item{\(\forall  x \exists  y_1 \cdots \exists  y_k \forall  z \, (x=z) \rightarrow \bigvee _{i=1}^ny_iEz\).
        }
\end{enumerate}
    Or, that each equivalence class has infinitely many elements:
    \begin{enumerate}
\item{
            For each \(k\), the sentence \(\forall  x_1 \cdots \forall  x_k \exists  y \, \bigwedge _{i=1}^n \neg (x_iEy)\).
        }
\end{enumerate}}
\end{tree}

\begin{tree}{title={Graphs (Rado Graph)}, taxon={example}, slug={BMT-e034}}

    The theory of a graph is a \ForesterRef{BMT-d017}{theory} in the \ForesterRef{BMT-d001}{language} \(\mathcal  L= \{ R \}\), where \(R\) is a binary relation.
    Depending on the author, the theory may only demand symmetry, \(\forall  x \forall  y \, xRy \rightarrow  yRx\),
    or additionally include irreflexivity: \(\forall  x \, \neg  xRx\).
\par{
    The theory of the \emph{Rado graph} (also called the random graph) is an extension of the theory of the graph by the extension axiom schema:
    \begin{enumerate}
\item{
            For each \(n\) and \(m\), the sentence
            \(\forall  x_1 \cdots \forall  x_n \forall  y_1 \cdots \forall  y_m \exists  z \, \bigwedge _{i=1}^nzRx_i \land \bigwedge _{j=1}^m \neg  zRy_j\).
        }
\end{enumerate}
    This ``single'' schema is enough to make the theory \ForesterRef{BMT-d017}{complete},
    and in fact the theory of the Rado graph is \ForesterRef{BMT-d201}{\(\aleph _0\)-categorical}.
}
\end{tree}

\printbibliography
\end{document}