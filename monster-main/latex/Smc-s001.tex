\documentclass[a4paper]{article}
\usepackage[final]{microtype}
\usepackage{fontspec}
\setmonofont{inconsolata}
\usepackage{amsmath,amsthm,amssymb,stmaryrd,mathtools,biblatex,forester}
\addbibresource{forest.bib}

\title{Smooth Classes}

\date{February 13, 2024}

\author{Morgan Bryant}

\begin{document}
\maketitle
\par{We define Smooth class constructions and their properties. Sources for this section are the papers On Generic Structures~\cite{Smc-r001} and Stable generic structures~\cite{Smc-r002}}
\begin{tree}{title={Definition of a Smooth Class}, taxon={definition}, slug={Smc-d001}}
We assume the language \(\mathcal {L}\) is countable and only relational.\par{A class \((K, \leq )\) of finite \(\mathcal {L}\)-structures (closed under isomorphism) is called \emph{a smooth class} if \(\leq\) is transitive, \(A \leq  B  \Rightarrow  A \subsetq  B\), and for each \(A \in  K\), there is a set of universal formulas \(\Phi _A\) such that:
\[A \leq  B  \Leftrightarrow  B \models   \Phi _A(A)\]
and we require that \(A \cong  A'  \Leftrightarrow   \Phi _A =  \Phi _{A'}\)}\par{This definition comes from On Generic structures~\cite{Smc-r001}}\par{An alternative definition, or characterization, from Baldwin and Shi~\cite{Smc-r002}, is a class that satisfies the following:}\par{If \(A \in  K\), \(A \leq  A\)}\par{If \(A \leq  B\), then \(A \subseteq  B\)}\par{\(\leq\) is transitive}\par{If \(A \leq  C\), and \(A \subseteq  B \subseteq  C\), then \(A \leq  B\) for \(A,B,C  \in  K\)}\par{\(\emptyset \in  K\) and \(\emptyset   \leq  A\) for all \(A \in  K\)}\par{The only difference between these two characterizations is that in the first definition, we essentially stipulate that \(\leq\) is definable/determined by a 
set of universal formulas. This turns out to be an advantage in working with these classes. Certainly, classes satisfying the first definition will satisfy Baldwin-Shi's. In this way,
it is perhaps wiser to regard the second definition as a characterization as opposed to an actual definition.}
\end{tree}

\begin{tree}{title={Ex: Initial Segments}, taxon={example}, slug={Smc-e001}}
 Let \((K, \leq _*)\) be the class of be finite, initial segements of the linear order \(( \omega , \leq )\), where \(A \leq _* B\) is "\(A\) is an initial segment of \(B\)".\par{It is clear that \(\leq\) fits into the definition of a smooth class, simply by the universal formula \(\Phi _A(y_1, \dots , y_n) =  \forall  x(( x=y_1 \lor \dots \lor  x=y_n)  \lor   \bigwedge _i^n x \geq  y_i)\)}\par{This class has \ForesterRef{Fra-d003}{AP} and \ForesterRef{Fra-d004}{JEP}, but not \ForesterRef{Fra-d005}{HP} or \ForesterRef{Fra-d006}{dAP}}
\end{tree}

\begin{tree}{title={Smooth Extension of Fraisse's Theorem}, taxon={theorem}, slug={Smc-t001}}
If \(\mathcal {L}\) is countable, and \((K, \leq )\) is a smooth class of finite structures has the \ForesterRef{Fra-d003}{amalgamation property} and the \ForesterRef{Fra-d004}{joint embedding property}, 
then there is a unique, countable structure \(\mathcal {M}\)  with the following properties:\par{1. Any isomorphism \(f:A \rightarrow  B\) with \(A,B \leq   \mathcal {M}\) and \(A,B \in  K\) extends to
an automorphism of \(\mathcal {M}\).}\par{2. \(\mathcal {M} =  \bigcup ^ \omega _n A_n\) where \(A_n  \leq  A_{n+1}\) and for all \(n\), \(A_n \in  K\)}\par{3. \(For every  A \in  K , there is an embedding  f:A \rightarrow  M  of  A  into  M  such that  f(A)  \leq  M\)}\par{The structure \(\mathcal {M}\) is the \emph{generic}, or sometimes, the \emph{limit} of \(K\).}\par{An equivalent characterization of a generic is properties 2 and 3 and the following property:}\par{4. If \(A \leq  M\) and \(A \leq  B\) for \(B \in  K\), then there is an isomorphism \(f: B \rightarrow  M\) extending the identity map \(id: A \rightarrow  M\) 
such that \(f(B)  \leq  M\)}
\end{tree}

\begin{tree}{title={Smooth Classes and Saturation}, taxon={theorem}, slug={Smc-t002}}
The following theorem first appears in \href{}{this paper} by Laskowski and Kueker\par{Let \((K,  \leq )\) be a smooth class with a \href{}{generic} \(\mathcal {A}\). If \(\mathcal {A}\) is weakly saturated, then \(\mathcal {A}\) is indeed saturated}
\end{tree}

\begin{tree}{title={Atomic Generic of Smooth Class (Kueker & Laskowski)}, taxon={theorem}, slug={Smc-t003}}
Let \((K, \leq )\) be a smooth class with a generic \(\mathcal {A}\). If for every \(A  \in  K\), \(\Phi _A\) (see \ForesterRef{Smc-d001}{here}) consists of a single universal formula,
then \(\mathcal {A}\) is an atomic model.
\end{tree}

\printbibliography
\end{document}