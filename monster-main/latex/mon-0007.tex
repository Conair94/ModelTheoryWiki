\documentclass[a4paper]{article}
\usepackage[final]{microtype}
\usepackage{fontspec}
\setmonofont{inconsolata}
\usepackage{amsmath,amsthm,amssymb,stmaryrd,mathtools,biblatex,forester}
\addbibresource{forest.bib}

\title{Alternation Number}

\date{February 2, 2024}

\author{Adam Melrod}

\begin{document}
\maketitle
\par{The \emph{alternation number} of a formula \(\varphi (x;y)\), denoted \(\operatorname {alt}( \varphi (x;y))\) is the maximal number \(n  \in   \omega\) (if it exists) such that there is an indiscernible sequence \((a_i : i  \in   \omega )\), some \(b\), and indices \(i_0 <  \dots  < i_n\) with \(\mathbb  M  \models   \varphi (a_i,b)  \Longleftrightarrow  i  \text { is even}\). If no such maximum exists, we let \(\operatorname {alt}( \varphi (x;y)) =  \infty\).}
\printbibliography
\end{document}