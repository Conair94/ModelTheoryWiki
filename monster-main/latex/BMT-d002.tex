\documentclass[a4paper]{article}
\usepackage[final]{microtype}
\usepackage{fontspec}
\setmonofont{inconsolata}
\usepackage{amsmath,amsthm,amssymb,stmaryrd,mathtools,biblatex,forester}
\addbibresource{forest.bib}

\title{Structure}

\date{February 8, 2024}

\author{Morgan Bryant}

\begin{document}
\maketitle
\par{Given a \ForesterRef{BMT-d001}{language} \(\mathcal {L}\), an \(\mathcal {L}\)-structure \(\mathcal {M}\) in this language has a universe \(M\) (often, the structure and its universe 
are written the same, by abuse of notation). The structure \(\mathcal {M}\) "interprets" the symbols of \(\mathcal {L}\) as follows:}\par{For a constant symbol \(c \in   \mathcal {L}\), the interpretation of \(c\) in \(\mathcal {M}\), denoted \(c^{ \mathcal {M}}\) represents a fixed 
element of \(M\)}\par{For an \(n\)-ary function symbol \(f \in   \mathcal {L}\), the interpretation of \(f\) in \(\mathcal {M}\), denoted \(f^{ \mathcal {M}}\),
is a function from \(M^n\) to \(M\)}\par{For an \(n\)-ary relation symbol \(R \in   \mathcal {L}\), the interpretation of \(R\) in \(\mathcal {M}\), denoted \(R^{ \mathcal {M}}\),
is a subset \(M^n\)}\par{Often, we are interested in the cardinality of a structure, denoted \(| \mathcal {M}|\), which is defined to be the cardinality of its universe.}
\printbibliography
\end{document}