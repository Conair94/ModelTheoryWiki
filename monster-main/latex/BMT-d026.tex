\documentclass[a4paper]{article}
\usepackage[final]{microtype}
\usepackage{fontspec}
\setmonofont{inconsolata}
\usepackage{amsmath,amsthm,amssymb,stmaryrd,mathtools,biblatex,forester}
\addbibresource{forest.bib}

\title{Morleyisation}

\date{September 26, 2024}

\author{Connor Lockhart}

\begin{document}
\maketitle
\par{ The Morleyisation of an \(L\)-theory \(T\) is the theory \(T^m\) in the language \(L^m \supset  L\). The expanded language \(L^m\) is formed by taking every \(L\)-formula \(\phi (x_1,...,x_n)\) and adding an \(n\)-placed relation symbol \(R_ \phi\). To \(T\) we add the axioms }\par{\(\forall  x_1,...,x_n (R_ \phi  (x_1,...,x_n) \leftrightarrow \phi (x_1,...,x_n)) \)}\par{Morleyisation preserves many properties such as \ForesterRef{BMT-d201}{\(\kappa\)-categoricity}, though the relations \(R_ \phi\) and the corresponding definable sets may be quite hard to understand. }
\printbibliography
\end{document}