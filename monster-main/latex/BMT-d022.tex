\documentclass[a4paper]{article}
\usepackage[final]{microtype}
\usepackage{fontspec}
\setmonofont{inconsolata}
\usepackage{amsmath,amsthm,amssymb,stmaryrd,mathtools,biblatex,forester}
\addbibresource{forest.bib}

\title{Satisfiability}

\date{March 10, 2024}

\author{Francis Westhead \and Oscar Coppola}

\begin{document}
\maketitle
\par{
    The satisfaction connective \(\vDash\) is defined as follows.
    To begin, it is defined with \(\mathcal  L\)-structures satsifying certain \(\mathcal  L\)-formulas.
}\par{
    Let \(\phi (x_1, \dots ,x_n)\) be an \(\mathcal  L\)-formula, and let \(a_1, \dots ,a_n \in  M\).
    \begin{enumerate}
\item{
            If \(\phi\) is an \ForesterRef{BMT-d008}{atomic formula} \(R(x_1, \dots ,x_n)\),
            then \(\mathcal  M \vDash \phi (a_1, \dots ,a_n)\) when \((a_1, \dots ,a_n) \in  R^{ \mathcal  M}\).
        }
        \item{
            If \(\phi\) is of the form \(\psi \land \theta\), where \(\psi\) and \(\theta\)
            are \(\mathcal  L\)-formulas, then \(\mathcal  M \vDash \phi (a_1, \dots ,a_n)\)
            when \(\mathcal  M \vDash \psi (a_1, \dots ,a_n)\) and \(\mathcal  M \vDash \theta (a_1, \dots ,a_n)\).
        }
        \item{
            If \(\phi\) is of the form \(\psi \lor \theta\), where \(\psi\) and \(\theta\)
            are \(\mathcal  L\)-formulas, then \(\mathcal  M \vDash \phi (a_1, \dots ,a_n)\)
            when \(\mathcal  M \vDash \psi (a_1, \dots ,a_n)\) or \(\mathcal  M \vDash \theta (a_1, \dots ,a_n)\).
        }
        \item{
            If \(\phi\) is of the form \(\neg \psi\), where \(\psi\) is a \(\mathcal  L\)-formula,
            then \(\mathcal  M \vDash \phi (a_1, \dots ,a_n)\) when \(\mathcal  M \not \vDash \psi (a_1, \dots ,a_n)\).
        }
        \item{
            If \(\phi\) is of the form \(\exists  y \, \psi (x_1, \dots ,x_n,y)\),
            where \(\psi\) is a \(\mathcal  L\)-formula, then \(\mathcal  M \vDash \phi (a_1, \dots ,a_n)\)
            when there is \(b \in  M\) such that \(\mathcal  M \vDash \psi (a_1, \dots ,a_n,b)\).
        }
        \item{
            If \(\phi\) is of the form \(\forall  y \, \psi (x_1, \dots ,x_n,y)\),
            where \(\psi\) is a \(\mathcal  L\)-formula, then \(\mathcal  M \vDash \phi (a_1, \dots ,a_n)\)
            when for every \(b \in  M\), \(\mathcal  M \vDash \psi (a_1, \dots ,a_n,b)\).
        }
\end{enumerate}}\par{
    Given an \(\mathcal  L\)-structure \(\mathcal  M\), and an \(\mathcal  L\)-theory \(T\),
    \(\mathcal  M \vDash  T\) when \(\mathcal  M \vDash \phi\) for all \(\phi \in  T\).
}\par{
    Given \(\mathcal  L\)-theories \(\Gamma\) and \(\Delta\), \(\Gamma \vDash \Delta\)
    when for any \(\mathcal  L\)-structure \(\mathcal  M\), if \(\mathcal  M \vDash \Gamma\)
    then \(\mathcal  M \vDash \Delta\).
    If \(\Delta = \{ \phi \}\), then one may write \(\Gamma \vDash \phi\); similarly if \(| \Gamma |=1\).
}\par{
    A \(\mathcal  L\)-theory \(T\) is \emph{satisfiable} when there exists an \(L\)-structure \(\mathcal  M\)
    such that \(\mathcal  M \vDash  T\).
}
\printbibliography
\end{document}