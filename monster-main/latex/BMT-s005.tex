\documentclass[a4paper]{article}
\usepackage[final]{microtype}
\usepackage{fontspec}
\setmonofont{inconsolata}
\usepackage{amsmath,amsthm,amssymb,stmaryrd,mathtools,biblatex,forester}
\addbibresource{forest.bib}

\title{Types}

\date{March 14, 2024}

\author{Connor Lockhart \and Francis Westhead \and Oscar Coppola}

\begin{document}
\maketitle
\par{In this section we give the basic definitions and theorems for types.}
\begin{tree}{title={Definition of a type}, taxon={definition}, slug={BMT-d019}}
Given a \ForesterRef{BMT-d001}{language}, \(\mathcal {L}\), and a \ForesterRef{BMT-d017}{\(\mathcal {L}\)-theory \(T\)}, a \emph{partial type} of \(T\) is a \ForesterRef{BMT-d022}{finitely satisfiable} set of formulas in a fixed tuple of variables \(x\).\par{A partial type \(p(x)\) (meaning that the denoted type consists of formulas each in the variables \(x\)) is \emph{complete} if for every \(\mathcal {L}\)-formula in the variables \(x\), \(\varphi (x) \in  p(x)\), either \(\varphi (x) \in  p(x)\) or \(\neg   \varphi (x) \in  p(x)\).}
\end{tree}

\begin{tree}{title={Definition of an isolated/principal type}, taxon={definition}, slug={BMT-d020}}

    Given a \ForesterRef{BMT-d001}{language}, \(\mathcal {L}\), and a \ForesterRef{BMT-d017}{\(\mathcal {L}\)-theory \(T\)},
    a partial type of T {p(x)} is \emph{principal} when there is an \(\mathcal {L}\)-formula \(\varphi (x)\) such that
    \(\models   \exists  x  \varphi (x)\) and for every \(\varpsi (x)  \in  p(x)\) we have that
    \(\models   \forall  x ( \varphi (x)  \rightarrow   \varpsi (x))\).
    In the case that \(p(x)\) is complete, we must have that \(\varphi (x) \in  p(x)\),
    and the first condition is redundant.
\par{
    A complete type of T, \(p(x)\) is \emph{isolated} if it is principal as a partial type.
    Equivalently, {p(x)} is \emph{isolated} if \(\{ p(x) \}\) is open \ForesterRef{BMT-d021}{S_n(T)}.
    This coincides with the usual topological terminology.
}
\end{tree}

\begin{tree}{title={The Stone Space \(S_n(T)\)}, taxon={definition}, slug={BMT-d021}}
Given a \ForesterRef{BMT-d001}{language}, \(\mathcal {L}\), and a \ForesterRef{BMT-d017}{\(\mathcal {L}\)-theory \(T\)},
\(S_n(T)\) denotes a topological space whose underlying set is the set of all complete \emph{n-types} of \(T\).
The topology on \(S_n(T)\) has a basis of open sets given by all sets of the form \([ \varphi (x)] :=  \{ p(x) \in  S_n(T):
 \varphi (x)  \in  p(x) \}\) (for \(\varphi (x)\) an \(\mathcal {L}\)-formula).
This space is a Stone Space in the usual sense and the Boolean Algebra associated to it via Stone Duality is
the \(n\)th Tarski-Lindenbaum Algebra of \(T\).
\end{tree}

\begin{tree}{title={Finite Satisfiability}, taxon={definition}, slug={BMT-d023}}

    An \ForesterRef{BMT-d017}{\(\mathcal  L\)-theory} \(T\) is \emph{finitely satisfiable} when every finite subset of \(T\) is \ForesterRef{BMT-d022}{satisfiable}.

\end{tree}

\printbibliography
\end{document}