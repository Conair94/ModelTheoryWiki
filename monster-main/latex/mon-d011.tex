\documentclass[a4paper]{article}
\usepackage[final]{microtype}
\usepackage{fontspec}
\setmonofont{inconsolata}
\usepackage{amsmath,amsthm,amssymb,stmaryrd,mathtools,biblatex,forester}
\addbibresource{forest.bib}

\title{Weak Elimination of Imaginaries}

\date{November 7, 2024}

\author{Ruohan Hu}

\begin{document}
\maketitle
\par{  
    A model \(M\) has weak elimination of imaginaries, when given a formula \(\theta ( \bar {x}, \bar {y})\) where \(l( \bar {x})=l( \bar {y})=n\) that defines an equivalence relation on \(M^n\), for every equivalence class \(\bar {a}/ \theta\), there is a formula \(\phi\), and a finite set of tuples \(X\) in \(M\) such that the equivalence class \(\bar {a}/ \theta\) is defined by \(\phi ( \bar {x}, \bar {b})\) if and only if \(\bar {b} \in  X\).
    
    A theory \(T\) is said to have weak elimination of imaginaries, if every model \(M\) of \(T\) has weak elimination of imaginaries.
}
\printbibliography
\end{document}