\documentclass[a4paper]{article}
\usepackage[final]{microtype}
\usepackage{fontspec}
\setmonofont{inconsolata}
\usepackage{amsmath,amsthm,amssymb,stmaryrd,mathtools,biblatex,forester}
\addbibresource{forest.bib}

\title{Equivalence Relation}

\date{October 10, 2024}

\author{Oscar Coppola}

\begin{document}
\maketitle
\par{
    The theory of an equivalence relation is a \ForesterRef{BMT-d017}{theory} in the \ForesterRef{BMT-d001}{language} \(\mathcal  L= \{ E \}\),
    where \(E\) is a binary relation, consisting of the following \ForesterRef{BMT-d013}{sentences}.
    \begin{enumerate}
\item{\(\forall  x \, xEx\).
        }
        \item{\(\forall  x \forall  y \, xEy \rightarrow  yEx\).
        }
        \item{\(\forall  x \forall  y \forall  z \, (xEy \land  yEz) \rightarrow  xEz\).
        }
\end{enumerate}}\par{
    Being uncomplicated, this theory is usually shown as an initial example (or rather, counterexample) of various properties.
    It is straightforward to add more restrictions, such as stating that there are only \(n\) equivalence classes:
    \begin{enumerate}
\item{\(\exists  x_1 \cdots \exists  x_n \forall  y \, \bigvee _{i=1}^nx_iEy\).
        }
\end{enumerate}
    Or, that each equivalence class has at most \(k\) elements:
    \begin{enumerate}
\item{\(\forall  x \exists  y_1 \cdots \exists  y_k \forall  z \, (x=z) \rightarrow \bigvee _{i=1}^ny_iEz\).
        }
\end{enumerate}
    Or, that each equivalence class has infinitely many elements:
    \begin{enumerate}
\item{
            For each \(k\), the sentence \(\forall  x_1 \cdots \forall  x_k \exists  y \, \bigwedge _{i=1}^n \neg (x_iEy)\).
        }
\end{enumerate}}
\printbibliography
\end{document}