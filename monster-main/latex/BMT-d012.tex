\documentclass[a4paper]{article}
\usepackage[final]{microtype}
\usepackage{fontspec}
\setmonofont{inconsolata}
\usepackage{amsmath,amsthm,amssymb,stmaryrd,mathtools,biblatex,forester}
\addbibresource{forest.bib}

\title{Definition of Free and Bound Variables}

\date{February 16, 2024}

\author{Oscar Coppola}

\begin{document}
\maketitle
\par{
    The set of \emph{free variables} of an \ForesterRef{BMT-d009}{\(\mathcal  L\)-formula} is defined inductively as follows.
    \begin{enumerate}
\item{
            The set of free variables of an atomic formula is the set of all variables appearing in the formula.
        }
        \item{\(\mathrm {var}( \neg  A) =  \mathrm {var}(A)\), where \(A\) is a formula.
        }
        \item{\(\mathrm {var}(A \lor  B) =  \mathrm {var}(A) \cup \mathrm {var}(B)\), where \(A\) and \(B\) are formulas.
        }
        \item{\(\mathrm {var}(A \land  B) =  \mathrm {var}(A) \cup \mathrm {var}(B)\), where \(A\) and \(B\) are formulas.
        }
        \item{\(\mathrm {var}( \exists  v \, A) =  \mathrm {var}(A) \setminus \set  v\), where \(A\) is a formula and \(v\) is a variable.
        }
        \item{\(\mathrm {var}( \forall  v \, A) =  \mathrm {var}(A) \setminus \set  v\), where \(A\) is a formula and \(v\) is a variable.
        }
\end{enumerate}}\par{
    If a variable appears in a formula but is not free, then it is called \emph{bound}.
}
\printbibliography
\end{document}