\documentclass[a4paper]{article}
\usepackage[final]{microtype}
\usepackage{fontspec}
\setmonofont{inconsolata}
\usepackage{amsmath,amsthm,amssymb,stmaryrd,mathtools,biblatex,forester}
\addbibresource{forest.bib}

\title{Basic Model Theory}

\date{February 8, 2024}

\author{Morgan Bryant\thanks{With contributions from Connor Lockhart, Francis Westhead, Garrett Peters.}}

\begin{document}
\maketitle
\par{We give the common definitions and theorems for basic/introductory model theory}
  
  
\begin{tree}{title={Structures, Isomorphisms, Substructures}, taxon={section}, slug={BMT-s001}}
In this section we give the basic notions and theorems for structures, isomorphisms, and substructures
\begin{tree}{title={Language}, taxon={definition}, slug={BMT-d001}}
A language, also called a vocabulary or signature by different authors, is a set \(\mathcal {L}\) consisting of symbols for constants, relations, and functions, 
often denoted by \(c\), \(R\), and \(f\) respectively. Languages may have any cardinality.
\end{tree}

\begin{tree}{title={Structure}, taxon={definition}, slug={BMT-d002}}
Given a \ForesterRef{BMT-d001}{language} \(\mathcal {L}\), an \(\mathcal {L}\)-structure \(\mathcal {M}\) in this language has a universe \(M\) (often, the structure and its universe 
are written the same, by abuse of notation). The structure \(\mathcal {M}\) "interprets" the symbols of \(\mathcal {L}\) as follows:\par{For a constant symbol \(c \in   \mathcal {L}\), the interpretation of \(c\) in \(\mathcal {M}\), denoted \(c^{ \mathcal {M}}\) represents a fixed 
element of \(M\)}\par{For an \(n\)-ary function symbol \(f \in   \mathcal {L}\), the interpretation of \(f\) in \(\mathcal {M}\), denoted \(f^{ \mathcal {M}}\),
is a function from \(M^n\) to \(M\)}\par{For an \(n\)-ary relation symbol \(R \in   \mathcal {L}\), the interpretation of \(R\) in \(\mathcal {M}\), denoted \(R^{ \mathcal {M}}\),
is a subset \(M^n\)}\par{Often, we are interested in the cardinality of a structure, denoted \(| \mathcal {M}|\), which is defined to be the cardinality of its universe.}
\end{tree}

\begin{tree}{title={Homomorphism}, taxon={definition}, slug={BMT-d003}}
Given two \ForesterRef{BMT-d002}{structures} \(\mathcal {A}\) and \(\mathcal {B}\) in a \ForesterRef{BMT-d001}{language} \(\mathcal {L}\), a \emph{homomorphism} \(\varphi :  \mathcal {A}  \rightarrow   \mathcal {B}\)
is a function between the universes \(A\) and \(B\) of \(\mathcal {A}\) and \(\mathcal {B}\) respectively such that:\par{ For every n-ary function \(f \in   \mathcal {L}\), for all \(a_1, \dots , a_n \in  A\), \(\varphi (f^{ \mathcal {A}}(a_1, \dots , a_n)) = f^{ \mathcal {B}}( \varphi (a_1), \dots ,  \varphi (a_n))\)}\par{For every n-ary relation \(R \in   \mathcal {L}\), for all \(a_1, \dots , a_n  \in  A\), \(\varphi (R^{ \mathcal {A}}(a_1, \dots , a_n))  \text { holds }  \Rightarrow  R^{ \mathcal {B}}( \varphi (a_1),  \dots ,  \varphi (a_n))\) holds}\par{For every constant symbol \(c  \in   \mathcal {L}\), \(\varphi (c^{ \mathcal {A}}) =c^{ \mathcal {B}}\)}
\end{tree}

\begin{tree}{title={Embedding of \(L\)-structures}, taxon={definition}, slug={BMT-d004}}
Given two structures in a language \(\mathcal {L}\) \(\mathcal {A}\) and \(\mathcal {B}\), an \emph{embedding} \(\varphi :  \mathcal {A}  \rightarrow   \mathcal {B}\)
is a \ForesterRef{BMT-d003}{homomorphism} such that:\par{For every n-ary relation \(R \in   \mathcal {L}\), for all \(a_1, \dots , a_n  \in  A\), \(\varphi (R^{ \mathcal {A}}(a_1, \dots , a_n))  \text { holds }  \Leftrightarrow  R^{ \mathcal {B}}( \varphi (a_1),  \dots ,  \varphi (a_n))\) holds}\par{This is stronger than a homomorphism because we now require a two way implication in the above property.}
\end{tree}

\begin{tree}{title={Isomorphism}, taxon={definition}, slug={BMT-d005}}
An \ForesterRef{BMT-d004}{embedding} \(\varphi :  \mathcal {A}  \rightarrow   \mathcal {B}\) between \(\mathcal {L}\)-structures is a \emph{isomorphism} if it is surjective.
\end{tree}

\begin{tree}{title={Substructure}, taxon={definition}, slug={BMT-d006}}
Given a \(\mathcal {L}\)-structure \(\mathcal {A}\), a subset \(B  \subseteq  A\) is called a \emph{substructure} of \(\mathcal {A}\) if:\par{For every constant \(c \in   \mathcal {L}\), \(c^{ \mathcal {A}}  \in  B\)}\par{For every n-ary function \(f \in   \mathcal {L}\), for any \(b_1, \dots , b_n  \in  B\), \(f^{ \mathcal {A}}(b_1, \dots ,b_n)  \in  B\)}\par{For every n-ary relation, \(R \in   \mathcal {L}\), for any \(b_1, \dots , b_n  \in  B\), we write \(R^B = R^{ \mathcal {A}} \cap   \mathcal {P}(B^n)\) and require that
\(R^{B}(b_1, \dots , b_n)\) holds (i.e., \((b_1, \dots , b_n)  \in  R^{B}\))  if and only if \(R^{ \mathcal {A}}(b_1, \dots , b_n)\) holds (i.e., \((b_1, \dots , b_n)  \in  R^{ \mathcal {A}}\)) 
Note that this condition is vacuously true, so only the first two conditions need to be checked when verifying a substructure.}
\end{tree}

\end{tree}


  
  
\begin{tree}{title={Formulas and Models}, taxon={section}, slug={BMT-s002}}
In this section we define formulas, elementary equivalence, elementary substructures, theories, and models
\begin{tree}{title={Term}, taxon={definition}, slug={BMT-d007}}
Given a language \(\mathcal {L}\), we define a \emph{term} to be any of the following:\par{If \(c \in   \mathcal {L}\) is a constant symbol, then \(c\) is a term}\par{If \(x\) is a variable symbol, then \(x\) is a term. We generally assume we have countably infinitely many variable symbols}\par{If \(t_1,..., t_n\) are terms, and \(f \in   \mathcal {L}\) is an \(n\)-ary function, then \(f(t_1,..., t_n)\) is a term.}
\end{tree}

\begin{tree}{title={Atomic Formula}, taxon={definition}, slug={BMT-d008}}
Given a language \(\mathcal {L}\), an \emph{atomic formula} is defined as follows:\par{If \(R \in   \mathcal {L}\) is an n-ary relation symbol, and \(t_1, \dots , t_n\) are \ForesterRef{BMT-d007}{terms}, then \(R(t_1, \dots , t_n)\) is an atomic formula.}\par{If \(t_1\) and \(t_2\) are \(\mathcal {L}\)-terms, then \(t_1 = t_2\) is an atomic formula. }
\end{tree}

\begin{tree}{title={\(\mathcal {L}\)-Formula}, taxon={definition}, slug={BMT-d009}}
Given a language \(\mathcal {L}\), an \(\mathcal {L}\)-formula is defined inductively as follows:\par{If \(\phi\) is an \(\mathcal {L}\)-\ForesterRef{BMT-d008}{atomic formula} then \(\phi\) is a formula}\par{If \(\phi\) and \(\psi\) are both formulas, then \(\neg \phi\), \(\phi   \land   \psi\), and \(\phi \lor \phi\) are formulas, where \(\lor , \neg , \land\) are the usual \ForesterRef{BMT-d010}{logical connectives}}\par{If \(\phi\) is a formula, then \(\exists  x  \phi\) and \(\forall  x  \phi\) are formulas, where \(\exists\) and \(\forall\) are \ForesterRef{BMT-d011}{quantifiers}}
\end{tree}

\end{tree}


  
  
\begin{tree}{title={Quantifier Elimination}, taxon={section}, slug={BMT-s003}}
Quantifier Elimination is a part of a broader technique in Model Theory where for a structure in a given language, an arbitrary formula can be written as a boolean combination of perhaps simpler formulas
\begin{tree}{title={Elimination Set}, taxon={definition}, slug={BMT-d014}}
An \emph{Elimination Set} for a language \(\mathcal {L}\) and class \(K\) of \(\mathcal {L}\)-structures, then a set \(\Gamma\) of formulas \(\phi\) is an elimination set for \(K\) if for every formula \(\phi ( \bar {x})\) of \(\mathcal {L}\) there is a formula \(\phi ^*( \bar {x})\) which is a boolean combinations of formulas in \(\Gamma\) and \(\phi\) is equivalent to \(\phi ^*\) in every structure in \(K\)
\end{tree}
\par{In particular, we will be most interested in elimination sets that are comprised of the set of quantifier free formulas. It is worth noting that, in some cases it is not possible to have a full quantifier elimination down to the level of a quantifier free set but perhaps we can restrict ourself to some reasonable set of formulas.}
\end{tree}


  
  
\begin{tree}{title={Back and Forth}, taxon={section}, slug={BMT-s004}}
In this section we give the basic definitions and theorems for quantifier Elimination
\end{tree}


  
  
\begin{tree}{title={Types}, taxon={section}, slug={BMT-s005}}
In this section we give the basic definitions and theorems for types.
\end{tree}


  
  
\begin{tree}{title={Saturation}, taxon={section}, slug={BMT-s006}}
In this section we give the basic definitions and theorems for saturation of models.
\end{tree}


  
  
\begin{tree}{title={Ultraproducts}, taxon={section}, slug={BMT-s007}}
In this section we give the basic definitions and theorems for the use of ultraproducts in model theory.
\begin{tree}{title={Definition of a Filter}, taxon={definition}, slug={BMT-d071}}
Let \(I\) be any set. Then a filter \(\mathcal {F}\) on the power set \(\mathcal {P}(I)\) is a collection of subsets of \(I\) with the following properties:\par{1: \(I \in \mathcal {F}\).}\par{2: If \(A \in   \mathcal {F}\) and \(A \subseteq  B\), then \(B \in   \mathcal {F}\).}\par{3: If \(A,B  \in   \mathcal {F}\), then we have \(A \cap  B \in   \mathcal {F}\).}\par{Furthermore, a filter is proper if \(\emptyset \notin \mathcal {F}\).}
\end{tree}

\begin{tree}{title={Definition of an Ultrafilter}, taxon={definition}, slug={BMT-d072}}
Let \(I\) be any set. Then an ultrafilter \(\mathcal {U}\) on \(\mathcal {P}(I)\) is a proper \ForesterRef{BMT-d071}{filter} on \(I\) with the additional property:\par{For every \(A \subseteq  I\), either \(A \in \mathcal {U}\) or \(I \backslash  A \in \mathcal {U}\).}\par{Every proper filter can be extended into an ultrafilter.}
\end{tree}

\begin{tree}{title={Principal Ultrafilters}, taxon={example}, slug={BMT-e071}}
Fix any \(i \in  I\). Then \(\mathcal {U}_i :=  \{ A \subseteq  I: i \in  A \}\) is an \ForesterRef{BMT-d072}{ultrafilter}. This is referred to as the principal ultrafilter concentrated at \(i\).
\end{tree}

\begin{tree}{title={The Fréchet Filter}, taxon={example}, slug={BMT-e072}}
For any set \(I\) of infinite cardinality, the family of cofinite sets forms a proper \ForesterRef{BMT-d071}{filter}. This filter is referred to as the Fréchet filter.\par{When extending the Fréchet filter to an \ForesterRef{BMT-d072}{ultrafilter}, the result is not \ForesterRef{BMT-e071}{principal}.}
\end{tree}

\begin{tree}{title={Ultraproduct}, taxon={definition}, slug={BMT-d073}}
Fix an index set \(I\) and a \ForesterRef{BMT-d001}{language} \(L\). Let \(\{ \mathcal {A}_i: i \in  I \}\) be a family of \(L\)-\ForesterRef{BMT-d002}{structures}, and let \(\mathcal {U}\) be any \ForesterRef{BMT-d072}{ultrafilter} on \(\mathcal {P}(I)\). Then the ultraproduct is the \(L\)-structure \(\mathcal {A}_* :=  \prod _{i \in  I} A_i / \mathcal {U}\). \par{In \(\mathcal {A}_*\), the elements of the universe of \(\mathcal {A}_*\) are functions \(\underline {a}\) with domain \(I\) and \(\underline {a}(i) \in  A_i\) under the equivalence class \(\sim _{ \mathcal {U}}\), where \[\underline {a} \sim _{ \mathcal {U}} \underline {b}  \text { if and only if } \{ i \in  I:  \underline {a}(i)= \underline {b}(i) \}   \in   \mathcal {U}.\]}
\end{tree}

\begin{tree}{title={ŁOś's Theorem}, taxon={Theorem}, slug={BMT-t071}}
Let \(\mathcal {A}_* =  \prod \limits _{i \in  I}  \mathcal {A}_i /  \mathcal {U}\) be an \ForesterRef{BMT-d073}{ultraproduct}. Then for any \(L\)-formula \(\phi (x_1, \dots ,x_n)\), we have \[\mathcal {A}_* \models   \phi ( \underline {a}_1, \dots , \underline {a}_n)  \text { if and only if }  \{ i \in  I: A_i \models \phi ( \underline {a}_1(i), \dots ,  \underline {a}_n(i)) \} \in \mathcal {U}.\]
\end{tree}

\end{tree}


  
  
\begin{tree}{title={Model Completenes}, taxon={Section}, slug={BMT-s008}}
An \emph{Elimination Set} for a language \(\mathcal {L}\) and class \(K\) of \(\mathcal {L}\)-structures, then a set \(\Gamma\) of formulas \(\phi\) is an elimination set for \(K\) if for every formula \(\phi ( \bar {x})\) of \(\mathcal {L}\) there is a formula \(\phi ^*( \bar {x})\) which is a boolean combinations of formulas in \(\Gamma\) and \(\phi\) is equivalent to \(\phi ^*\) in every structure in \(K\)
\end{tree}


  
  
\begin{tree}{title={Categoricity}, taxon={section}, slug={BMT-s009}}
In this section, we give the basic definitions for \(\kappa\)-categoricity. We then establish some classical characterisations of \(\aleph _0\)-categoricity.
\begin{tree}{title={Definition of \(\kappa\)-categoricity}, taxon={definition}, slug={BMT-d201}}

    Given a cardinal \(\kappa\) and a \ForesterRef{BMT-d001}{language} \(\mathcal {L}\), an \ForesterRef{BMT-d017}{\(\mathcal {L}\)-theory} is \emph{\(\kappa\)-categorical} if whenever \(\mathcal {M}\) and \(\mathcal {N}\) are \(\mathcal {L}\)-structures with \(| \mathcal {M}|=| \mathcal {M}|= \kappa\), then \(\mathcal {M}  \cong   \mathcal {N}\). 

\end{tree}

\begin{tree}{title={Definition of a Tarski-Lindenbaum Algebra}, taxon={definition}, slug={BMT-d202}}
Given a \ForesterRef{BMT-d001}{language}, \(\mathcal {L}\), and an \ForesterRef{BMT-d017}{\(\mathcal {L}\)-theory}, \(T\), the \emph{\(n\)'th Tarski-Lindenbaum algebra of \(T\)} is the set of \ForesterRef{BMT-d009}{\(\mathcal {L}\)-formulas} quotiented by the relation of \ForesterRef{BMT-d018}{\(T\)-equivalence}.
\end{tree}

\begin{tree}{title={(Ryll-Nardjewski) Characterisations of \(\omega\)-categoricity}, taxon={theorem}, slug={BMT-t003}}
For \(\aleph _0\)-categoricity, there are a number of useful characterisations. The following are due to Ryll and Nardjewski.\par{Given a \ForesterRef{BMT-d001}{language}, \(\mathcal {L}\), and an \ForesterRef{BMT-d017}{\(\mathcal {L}\)-theory} \(T\), the following are equivalent:}\par{1 \(T\) is \(\aleph _0\)-categorical.}\par{For every \(n \in   \omega\), there are finitely many \ForesterRef{BMT-d009}{\(\mathcal {L}\)-formulas} in \(n\)-variables up to \ForesterRef{BMT-d018}{\(T\)-equivalence}.}\par{For every \(n \in   \omega\), the \ForesterRef{BMT-d202}{\(n\)'th Tarski-Lindenbaum algebra of \(T\)} is finite. 
Every \ForesterRef{BMT-d019}{type} over \(T\) is \ForesterRef{BMT-d020}{isolated}. }
\end{tree}

\end{tree}


\printbibliography
\end{document}