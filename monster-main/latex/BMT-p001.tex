\documentclass[a4paper]{article}
\usepackage[final]{microtype}
\usepackage{fontspec}
\setmonofont{inconsolata}
\usepackage{amsmath,amsthm,amssymb,stmaryrd,mathtools,biblatex,forester}
\addbibresource{forest.bib}

\title{Basic Model Theory}

\date{February 8, 2024}

\author{Morgan Bryant}

\begin{document}
\maketitle
\par{We give the common definitions and theorems for basic/introductory model theory}
\begin{tree}{title={Structures, Isomorphisms, Substructures}, taxon={section}, slug={BMT-s001}}
In this structure we give the basic notions and theorems for structures, isomorphisms, and substructures
\begin{tree}{title={Definition of a language}, taxon={definition}, slug={BMT-d001}}
The formula \(\varphi (x;y)\) has the \emph{independence property} if there
are sequences of tuples \((a_i : i  \in   \omega )\) and
\((b_S : S  \subseteq   \omega )\) such that for every subset \(S  \subseteq   \omega\)
\[i  \in  S  \Longleftrightarrow   \mathbb {M}  \models   \varphi (a_i; b_S).\]\par{A language, also called a vocabulary, is a set \(\mathcal {L}\) consisting of symbols for constants, relations, and functions, 
often denoted by \(c\), \(R\), and \(f\) respectively. Languages may have any cardinality.}
\end{tree}

\begin{tree}{title={Definition of a Structure}, taxon={definition}, slug={BMT-d002}}
Given a \ForesterRef{BMT-d001}{language} \(\mathcal {L}\), an \(\mathcal {L}\)-structure \(\mathcal {M}\) in this language has a universe \(M\) (often, the structure and its universe 
are written the same, by abuse of notation). The structure \(\mathcal {M}\) "interprets" the symbols of \(\mathcal {L}\) as follows:\par{For a constant symbol \(c \in   \mathcal {L}\), the interpretation of \(c\) in \(\mathcal {M}\), denoted \(c^{ \mathcal {M}}\) represents a fixed 
element of \(M\)}\par{For an \(n\)-ary function symbol \(f \in   \mathcal {L}\), the interpretation of \(f\) in \(\mathcal {M}\), denoted \(f^{ \mathcal {M}}\),
is a function from \(M^n\) to \(M\)}\par{For an \(n\)-ary relation symbol \(R \in   \mathcal {L}\), the interpretation of \(R\) in \(\mathcal {M}\), denoted \(R^{ \mathcal {M}}\),
is a subset \(M^n\)}\par{Often, we are interested in the cardinality of a structure, denoted \(| \mathcal {M}|\), which is defined to be the cardinality of its universe.}
\end{tree}

\begin{tree}{title={Definition of a Homomorphism}, taxon={definition}, slug={BMT-d003}}
Given two \ForesterRef{BMT-d002}{structures} in a \ForesterRef{BMT-d001}{language} \(\mathcal {L}\) \(\mathcal {A}\) and \(\mathcal {B}\), a \emph{homomorphism} \(\varphi :  \mathcal {A}  \rightarrow   \mathcal {B}\)
is a function between the universes \(A\) and \(B\) of \(\mathcal {A}\) and \(\mathcal {B}\) respectively such that:\par{ For every n-ary function \(f \in   \mathcal {L}\), for all \(a_1, \dots , a_n \in  A\), \(\varphi (f^{ \mathcal {A}}(a_1, \dots , a_n)) = f^{ \mathcal {B}}( \varphi (a_1), \dots ,  \varphi (a_n))\)}\par{For every n-ary relation \(R \in   \mathcal {L}\), for all \(a_1, \dots , a_n  \in  A\), \(\varphi (R^{ \mathcal {A}}(a_1, \dots , a_n))  \text { holds }  \Rightarrow  R^{ \mathcal {B}}( \varphi (a_1),  \dots ,  \varphi (a_n))\) holds}\par{For every constant symbol \(c  \in   \mathcal {L}\), \(\varphi (c^{ \mathcal {A}}) =c^{ \mathcal {B}}\)}
\end{tree}

\begin{tree}{title={Definition of an embedding of \(L\)-structures}, taxon={definition}, slug={BMT-d004}}
Given two structures in a language \(\mathcal {L}\) \(\mathcal {A}\) and \(\mathcal {B}\), an \emph{embedding} \(\varphi :  \mathcal {A}  \rightarrow   \mathcal {B}\)
is a \ForesterRef{BMT-d003}{homomorphism} such that:\par{For every n-ary relation \(R \in   \mathcal {L}\), for all \(a_1, \dots , a_n  \in  A\), \(\varphi (R^{ \mathcal {A}}(a_1, \dots , a_n))  \text { holds }  \Leftrightarrow  R^{ \mathcal {B}}( \varphi (a_1),  \dots ,  \varphi (a_n))\) holds}\par{This is stronger than a homomorphism because we now require a two way implication in the above property.}
\end{tree}

\end{tree}

\printbibliography
\end{document}