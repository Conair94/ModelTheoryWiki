\documentclass[a4paper]{article}
\usepackage[final]{microtype}
\usepackage{fontspec}
\setmonofont{inconsolata}
\usepackage{amsmath,amsthm,amssymb,stmaryrd,mathtools,biblatex,forester}
\addbibresource{forest.bib}

\title{Interpretation of Terms}

\date{March 14, 2024}

\author{Oscar Coppola}

\begin{document}
\maketitle
\par{
    Given a \ForesterRef{BMT-d001}{language} \(\mathcal  L\) and an \(\mathcal  L\)-structure \(\mathcal  M\),
    the interpretation of an \ForesterRef{BMT-d007}{\(\mathcal  L\)-term} in \(\mathcal  M\) gives an element of \(M\).
    This definition is inductive.
}\par{
    Let \(t\) be an \(\mathcal  L\)-term.
    \begin{enumerate}
\item{
            If \(t\) is a constant symbol \(c\) from \(\mathcal  L\),
            then the interpretation of \(t\) in \(\mathcal  M\) is simply \(c^{ \mathcal  M}\).
        }
        \item{
            If \(t\) is a function symbol \(f\) from \(\mathcal  L\) applied to a series of
            \(\mathcal  L\)-terms \(t_1, \dots ,t_n\), then the interpretation of \(t\) in \(\mathcal  M\) is
            \(f^{ \mathcal  M}(t_1^{ \mathcal  M}, \dots ,t_n^{ \mathcal  M})\).
        }
\end{enumerate}}
\printbibliography
\end{document}