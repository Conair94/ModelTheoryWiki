\documentclass[a4paper]{article}
\usepackage[final]{microtype}
\usepackage{fontspec}
\setmonofont{inconsolata}
\usepackage{amsmath,amsthm,amssymb,stmaryrd,mathtools,biblatex,forester}
\addbibresource{forest.bib}

\title{Definition of a language}

\date{February 8, 2024}

\author{Morgan Bryant}

\begin{document}
\maketitle
\par{The formula \(\varphi (x;y)\) has the \emph{independence property} if there
are sequences of tuples \((a_i : i  \in   \omega )\) and
\((b_S : S  \subseteq   \omega )\) such that for every subset \(S  \subseteq   \omega\)
\[i  \in  S  \Longleftrightarrow   \mathbb {M}  \models   \varphi (a_i; b_S).\]}\par{A language, also called a vocabulary, is a set \(\mathcal {L}\) consisting of symbols for constants, relations, and functions, 
often denoted by \(c\), \(R\), and \(f\) respectively. Languages may have any cardinality.}
\printbibliography
\end{document}