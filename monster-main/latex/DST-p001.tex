\documentclass[a4paper]{article}
\usepackage[final]{microtype}
\usepackage{fontspec}
\setmonofont{inconsolata}
\usepackage{amsmath,amsthm,amssymb,stmaryrd,mathtools,biblatex,forester}
\addbibresource{forest.bib}

\title{Basics of Descriptive Set Theory and Relations to Model Theory}

\date{March 1, 2024}

\author{Morgan Bryant}

\begin{document}
\maketitle
\par{We present the basics of descriptive set theory (DST) and its relationship to model theory.}\par{The main resource will be Kechris's Classical Descriptive Set Theory~\cite{DST-r001}.}
\begin{tree}{title={Basic DST}, taxon={definition}, slug={DST-s001}}
We give basic definitions of Borel sets, topology, polish spaces, Baire spaces, etc.
\begin{tree}{title={Definition of Borel Sets}, taxon={definition}, slug={DST-d001}}
Given a topology, the Borel sets are the sets generated by closing the open sets under complementation and countable union.\par{A \(G_ \delta\) set is a Borel set formed by a countable intersection of open sets.}\par{A \(F_ \sigma\) set is a Borel set formed by a countable union of closed sets.}\par{(Recall a countable union of open sets is open, and a countable intersection is closed.)}
\end{tree}

\begin{tree}{title={Definition of Borel Hierarchy}, taxon={definition}, slug={DST-d002}}
Building off of the \ForesterRef{DST-d001}{Borel sets}, welet \(\Sigma _0\) denote open sets, \(\Pi _0\) the closed sets, and then 
\[\Sigma _n =  \{ \bigcup _i^ \omega  A_i : A_i  \in   \Pi _{n-1} \}\]
\[\Pi _n =  \{ \bigcap _i^ \omega  A_i : A_i  \in   \Sigma _{n-1} \}\]\par{This is the Borel Hierarchy, and it mirrors similar hierarchies in recursion theory. 
Notice in particular \(\Sigma _1\) is the set of \(F_ \sigma\) sets, \(\Pi _1\) is the set of \(G_ \delta\) sets. }
\end{tree}

\begin{tree}{title={Definition of Polish Space}, taxon={definition}, slug={DST-d003}}
Given a topological space \(X\), \(X\) is a Polish space if \(X\) is completely metrisable and seperable\par{(Seperable = Second Countable = There is a dense, countable set in \(X\))}\par{(Completely metrisable = there is a metric on \(X\) which is complete = there is a metric on \(X\) so that every cauchy sequence 
has a limit)}\par{Closed subsets of Polish spaces are themselves Polish (closed = the restricted metric is complete), Product and sums of topologies of Polish spaces are Polish}
\end{tree}

\begin{tree}{title={Definition of Meagre and Comeagre Sets}, taxon={definition}, slug={DST-d004}}
Given a topological space \(X\), \(A \subseteq  X\) is meagre if \[A =  \bigcup _i^ \omega  A_i\] where the closure of \(A_i\) has empty interior (i.e., nowhere dense)\par{Comeagre sets are the complements of meagre sets}
\end{tree}

\end{tree}

\printbibliography
\end{document}