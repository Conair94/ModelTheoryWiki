\documentclass[a4paper]{article}
\usepackage[final]{microtype}
\usepackage{fontspec}
\setmonofont{inconsolata}
\usepackage{amsmath,amsthm,amssymb,stmaryrd,mathtools,biblatex,forester}
\addbibresource{forest.bib}

\title{Stability Theory}

\date{February 2, 2024}

\author{Connor Lockhart\thanks{With contributions from Ruohan Hu.}}

\begin{document}
\maketitle
\par{\textbf{Stable theories}}
\begin{tree}{title={Indiscernible Sequence}, taxon={definition}, slug={mon-d014}}


Given a cardinal \(\kappa\), the sequence of \(n\)-tuples of elements in a model \(M\), \(\{ \overline {b_i} \} _{i< \kappa }\) is an indiscernibles sequence in \(M\) over \(A \subseteq  M\) means that: for any \(k< \omega\) and indices \(i_1,...,i_k\) and \(j_1,...,j_k\), and any formula \(\varphi ( \overline {x_1},..., \overline {x_k}, \overline {a})\) where \(\overline {a}\) is a tuple of parameters from \(A\), we have \(M \models   \varphi ( \overline {b_{i_1}},..., \overline {b_{i_k}}, \overline {a}) \leftrightarrow   \varphi ( \overline {b_{j_1}},..., \overline {b_{j_k}}, \overline {a})\).

In other words, any two finite subsequence of \(\{ \overline {b_i} \} _{i< \kappa }\) of the same length have the same \ForesterRef{BMT-d019}{type} over \(A\), i.e, \(tp(( \overline {b_{i_1}},..., \overline {b_{i_k}})/A)=tp(( \overline {b_{j_1}},..., \overline {b_{j_k}})/A)\). 

Notice that the above definition, the index of the sequence, i.e. the ordinal \(\omega\), can be substituted by any other ordinal, or just any linear order. This would define another kind of "indiscernible seuqnece," but we still refer to it with the same term.

A set of \(n\)-tuples is an Indiscernible set if it is an indiscernible sequence under any well-order over the set.

\end{tree}

\begin{tree}{title={Forking and Dividing}, taxon={subsection}, slug={mon-s003}}
\textbf{Forking and Dividing}
\begin{tree}{title={Dividing}, taxon={definition}, slug={mon-d012}}

The formula \(\varphi ( \overline {x}, \overline {a})\) (with parameter \(\overline {a}\)) \emph{divides} over \(A\), if there are \(n< \omega\) and sequence \(\{ \overline {a_i}:i< \omega \}\) such that:
\begin{enumerate}
\item{\(tp( \overline {a},A)=tp( \overline {a_i},A)\);}
    \item{\(\{ \varphi (x, \overline {a_i}):i< \omega \}\) is \(n\)-inconsistent.}
\end{enumerate}
Usually we have \(\bar {a} \not \subset  A\), because obviously \(\varphi ( \bar {x}, \bar {a})\) does not divide over \(A\) if \(\bar {a} \subseteq  A\).

We say a type \(\mathfrak {p}\) divides over \(A\), if it implies a formula dividing over \(A\).

\end{tree}

\begin{tree}{title={Forking}, taxon={definition}, slug={mon-d013}}
\ForesterRef{BMT-d019}{Type} \(p \in  S(B)\) forks over \(A\) if for some finite \(n\), and formulas \(\varphi _k( \overline {x}, \overline {a_{k}})\) where \(\overline {a_k} \subset  B\), we have \(p \vdash \bigvee _{k<n} \varphi _k( \overline {x}, \overline {a_{k}})\) where each \(\varphi _k( \overline {x}, \overline {a_{k}})\) divides over \(A\), and \(\bar {x}\) consist of some of the variables in \(p\)/

Usually, we have \(p \in  S(B)\) when \(B \not \subseteq  A\), because if \(B \subseteq  A\) then we will see that that \(p\) does not fork over \(A\). Moreover, we often takes \(A \subseteq  B\).

Notice that forking can also be defined for types over infinitely many variables.

\end{tree}

\end{tree}

\printbibliography
\end{document}