\documentclass[a4paper]{article}
\usepackage[final]{microtype}
\usepackage{fontspec}
\setmonofont{inconsolata}
\usepackage{amsmath,amsthm,amssymb,stmaryrd,mathtools,biblatex,forester}
\addbibresource{forest.bib}

\title{Fraisse's Theorem}

\date{February 13, 2024}

\author{Morgan Bryant}

\begin{document}
\maketitle
\par{If \(\mathcal {L}\) is countable, and \((K, \subset )\) a class of finitely generated structures has the \ForesterRef{Fra-d003}{amalgamation property}, the \ForesterRef{Fra-d004}{joint embedding property}, and the \ForesterRef{Fra-d005}{hereditary property}, 
Then there is a unique, countable structure \(\mathcal {M}\) whose \ForesterRef{Fra-d002}{age} is \(K\) with the property that any isomorphism \(f:A \rightarrow  B\) with \(A,B \subseteq   \mathcal {M}\) and \(A,B \in  K\) extends to
an automorphism of \(\mathcal {M}\).}\par{The structure \(\mathcal {M}\) is the \emph{Fraisse limit} of \(K\).}\par{We call a class \((K, \subseteq )\) satisfying AP, HP, and JEP a Fraisse class}\par{When a structure has the above isomorphism extension property, we say it is \emph{ultrahomogeneous} (or \emph{homogeneous}). 
The converse of Fraisse's Theorem is also true in the sense that if we have a structure which is ultrahomogeneous with respect to its age, then its age is a Fraisse class.}
\printbibliography
\end{document}