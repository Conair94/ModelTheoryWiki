\documentclass[a4paper]{article}
\usepackage[final]{microtype}
\usepackage{fontspec}
\setmonofont{inconsolata}
\usepackage{amsmath,amsthm,amssymb,stmaryrd,mathtools,biblatex,forester}
\addbibresource{forest.bib}

\title{Definition of Model Complete}

\date{February 27, 2024}

\author{Connor Lockhart\thanks{With contributions from Morgan Bryant.}}

\begin{document}
\maketitle
\par{A Theory \(T\) in a first order language \(L\) is \emph{Model-Complete} if  every 
\begin{tree}{title={Embedding of \(L\)-structures}, taxon={definition}, slug={BMT-d004}}
Given two structures in a language \(\mathcal {L}\) \(\mathcal {A}\) and \(\mathcal {B}\), an \emph{embedding} \(\varphi :  \mathcal {A}  \rightarrow   \mathcal {B}\)
is a \ForesterRef{BMT-d003}{homomorphism} such that:\par{For every n-ary relation \(R \in   \mathcal {L}\), for all \(a_1, \dots , a_n  \in  A\), \(\varphi (R^{ \mathcal {A}}(a_1, \dots , a_n))  \text { holds }  \Leftrightarrow  R^{ \mathcal {B}}( \varphi (a_1),  \dots ,  \varphi (a_n))\) holds}\par{This is stronger than a homomorphism because we now require a two way implication in the above property.}
\end{tree}
 between \(L\)-structures which are models of \(T\) is an elementary embedding. }
\printbibliography
\end{document}