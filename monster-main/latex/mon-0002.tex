\documentclass[a4paper]{article}
\usepackage[final]{microtype}
\usepackage{fontspec}
\setmonofont{inconsolata}
\usepackage{amsmath,amsthm,amssymb,stmaryrd,mathtools,biblatex,forester}
\addbibresource{forest.bib}

\title{Characterizations of NIP for a formula}

\date{February 2, 2024}

\author{Adam Melrod}

\begin{document}
\maketitle
\par{The following conditions are equivalent:
\begin{enumerate}
\item{ The formula \(\varphi (x;y)\) has the independence property.}
  \item{ The formula \(\varphi ^{ \vee }(y;x)\) has the independence property,
	where \(\varphi ^{ \vee }(y;x)\) is the formula \(\varphi (x;y)\) with the
	opposite partition.}
	\item{ For any two finite sets \(U\) and \(V\), and any subset \(R  \subseteq  U  \times  V\), there are \((a_i : i  \in  U)\) and \((b_j : j  \in  U)\) such that \(\mathbb   \models   \varphi (a_i;b_j)  \Longleftrightarrow  (i,j)  \in  R\).}
	\item{ There is an indiscernible sequence \((a_i : i  \in   \omega )\)
	and some \(b\) such that \(\mathbb  M  \models   \varphi (a_i;b)  \Longleftrightarrow  i\)
	is even.}
\end{enumerate}}\par{We are also interested in the following characterization, which is more amenable to computations.}
\begin{tree}{title={Alternation Number}, taxon={definition}, slug={mon-0007}}
The \emph{alternation number} of a formula \(\varphi (x;y)\), denoted \(\operatorname {alt}( \varphi (x;y))\) is the maximal number \(n  \in   \omega\) (if it exists) such that there is an indiscernible sequence \((a_i : i  \in   \omega )\), some \(b\), and indices \(i_0 <  \dots  < i_n\) with \(\mathbb  M  \models   \varphi (a_i,b)  \Longleftrightarrow  i  \text { is even}\). If no such maximum exists, we let \(\operatorname {alt}( \varphi (x;y)) =  \infty\).
\end{tree}

\begin{tree}{title={Alternation Lemma}, taxon={lemma}, slug={mon-0008}}
A formula \(\varphi (x;y)\) is NIP if and only if \(\operatorname {alt}( \varphi (x;y)) <  \infty\).
\end{tree}

\printbibliography
\end{document}