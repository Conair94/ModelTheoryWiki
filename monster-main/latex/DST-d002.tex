\documentclass[a4paper]{article}
\usepackage[final]{microtype}
\usepackage{fontspec}
\setmonofont{inconsolata}
\usepackage{amsmath,amsthm,amssymb,stmaryrd,mathtools,biblatex,forester}
\addbibresource{forest.bib}

\title{Definition of Borel Hierarchy}

\date{March 1, 2024}

\author{Morgan Bryant}

\begin{document}
\maketitle
\par{Building off of the \ForesterRef{DST-d001}{Borel sets}, welet \(\Sigma _0\) denote open sets, \(\Pi _0\) the closed sets, and then 
\[\Sigma _n =  \{ \bigcup _i^ \omega  A_i : A_i  \in   \Pi _{n-1} \}\]
\[\Pi _n =  \{ \bigcap _i^ \omega  A_i : A_i  \in   \Sigma _{n-1} \}\]}\par{This is the Borel Hierarchy, and it mirrors similar hierarchies in recursion theory. 
Notice in particular \(\Sigma _1\) is the set of \(F_ \sigma\) sets, \(\Pi _1\) is the set of \(G_ \delta\) sets. }
\printbibliography
\end{document}