\documentclass[a4paper]{article}
\usepackage[final]{microtype}
\usepackage{fontspec}
\setmonofont{inconsolata}
\usepackage{amsmath,amsthm,amssymb,stmaryrd,mathtools,biblatex,forester}
\addbibresource{forest.bib}

\title{The Model Companion}

\date{}

\author{Adam Melrod \and Connor Lockhart \and Morgan Bryant}

\begin{document}
\maketitle
\par{This is the potential start of a new online resource on model theory, replacing the (now defunct) model theory wiki.}\par{This website is made with the software Forester (http://www.jonmsterling.com/jms-005P.xml). Forester allows you to explore the site in many interesting ways, we have provided a structured starting off points as in below but you may search for any page using Command K}\par{The site is nonlinear in its arrangement of information. At the high level, the website is broken into parts which are comprised of sections. Currently parts correspond to large themes or subjects in model theory and the sections subdivide these themes further. These parts and sections may not be disjoint but the nature of forester is well suited to navigating these overlaps since definitions and theorems can be transcluded easily. }\par{In the future it would be nice to have parts which organize the material as one would encounter it as a course or set of lecture notes intended for self study but for now, we intend to set up the backbone as a wikipedia or stacks project reference document and then reassemble the information after a critical mass has been achieved. }\textbf{Here are the parts which currently exist or are being worked on}
  
  
\begin{tree}{title={Basic Model Theory}, taxon={part}, slug={BMT-p001}}
We give the common definitions and theorems for basic/introductory model theory
  
  
\begin{tree}{title={Structures, Isomorphisms, Substructures}, taxon={section}, slug={BMT-s001}}
In this section we give the basic notions and theorems for structures, isomorphisms, and substructures
\begin{tree}{title={Definition of a language}, taxon={definition}, slug={BMT-d001}}
A language, also called a vocabulary or signature by different authors, is a set \(\mathcal {L}\) consisting of symbols for constants, relations, and functions, 
often denoted by \(c\), \(R\), and \(f\) respectively. Languages may have any cardinality.
\end{tree}

\begin{tree}{title={Definition of a Structure}, taxon={definition}, slug={BMT-d002}}
Given a \ForesterRef{BMT-d001}{language} \(\mathcal {L}\), an \(\mathcal {L}\)-structure \(\mathcal {M}\) in this language has a universe \(M\) (often, the structure and its universe 
are written the same, by abuse of notation). The structure \(\mathcal {M}\) "interprets" the symbols of \(\mathcal {L}\) as follows:\par{For a constant symbol \(c \in   \mathcal {L}\), the interpretation of \(c\) in \(\mathcal {M}\), denoted \(c^{ \mathcal {M}}\) represents a fixed 
element of \(M\)}\par{For an \(n\)-ary function symbol \(f \in   \mathcal {L}\), the interpretation of \(f\) in \(\mathcal {M}\), denoted \(f^{ \mathcal {M}}\),
is a function from \(M^n\) to \(M\)}\par{For an \(n\)-ary relation symbol \(R \in   \mathcal {L}\), the interpretation of \(R\) in \(\mathcal {M}\), denoted \(R^{ \mathcal {M}}\),
is a subset \(M^n\)}\par{Often, we are interested in the cardinality of a structure, denoted \(| \mathcal {M}|\), which is defined to be the cardinality of its universe.}
\end{tree}

\begin{tree}{title={Definition of a Homomorphism}, taxon={definition}, slug={BMT-d003}}
Given two \ForesterRef{BMT-d002}{structures} \(\mathcal {A}\) and \(\mathcal {B}\) in a \ForesterRef{BMT-d001}{language} \(\mathcal {L}\), a \emph{homomorphism} \(\varphi :  \mathcal {A}  \rightarrow   \mathcal {B}\)
is a function between the universes \(A\) and \(B\) of \(\mathcal {A}\) and \(\mathcal {B}\) respectively such that:\par{ For every n-ary function \(f \in   \mathcal {L}\), for all \(a_1, \dots , a_n \in  A\), \(\varphi (f^{ \mathcal {A}}(a_1, \dots , a_n)) = f^{ \mathcal {B}}( \varphi (a_1), \dots ,  \varphi (a_n))\)}\par{For every n-ary relation \(R \in   \mathcal {L}\), for all \(a_1, \dots , a_n  \in  A\), \(\varphi (R^{ \mathcal {A}}(a_1, \dots , a_n))  \text { holds }  \Rightarrow  R^{ \mathcal {B}}( \varphi (a_1),  \dots ,  \varphi (a_n))\) holds}\par{For every constant symbol \(c  \in   \mathcal {L}\), \(\varphi (c^{ \mathcal {A}}) =c^{ \mathcal {B}}\)}
\end{tree}

\begin{tree}{title={Definition of an embedding of \(L\)-structures}, taxon={definition}, slug={BMT-d004}}
Given two structures in a language \(\mathcal {L}\) \(\mathcal {A}\) and \(\mathcal {B}\), an \emph{embedding} \(\varphi :  \mathcal {A}  \rightarrow   \mathcal {B}\)
is a \ForesterRef{BMT-d003}{homomorphism} such that:\par{For every n-ary relation \(R \in   \mathcal {L}\), for all \(a_1, \dots , a_n  \in  A\), \(\varphi (R^{ \mathcal {A}}(a_1, \dots , a_n))  \text { holds }  \Leftrightarrow  R^{ \mathcal {B}}( \varphi (a_1),  \dots ,  \varphi (a_n))\) holds}\par{This is stronger than a homomorphism because we now require a two way implication in the above property.}
\end{tree}

\begin{tree}{title={Definition of an isomorphism}, taxon={definition}, slug={BMT-d005}}
An \ForesterRef{BMT-d004}{embedding} \(\varphi :  \mathcal {A}  \rightarrow   \mathcal {B}\) between \(\mathcal {L}\)-structures is a \emph{isomorphism} if it is surjective.
\end{tree}

\begin{tree}{title={Definition of a Substructure}, taxon={definition}, slug={BMT-d006}}
Given a \(\mathcal {L}\)-structure \(\mathcal {A}\), a subset \(B  \subseteq  A\) is called a \emph{substructure} of \(\mathcal {A}\) if:\par{For every constant \(c \in   \mathcal {L}\), \(c^{ \mathcal {A}}  \in  B\)}\par{For every n-ary function \(f \in   \mathcal {L}\), for any \(b_1, \dots , b_n  \in  B\), \(f^{ \mathcal {A}}(b_1, \dots ,b_n)  \in  B\)}\par{For every n-ary relation, \(R \in   \mathcal {L}\), for any \(b_1, \dots , b_n  \in  B\), we write \(R^B = R^{ \mathcal {A}} \cap   \mathcal {P}(B^n)\) and require that
\(R^{B}(b_1, \dots , b_n)\) holds (i.e., \((b_1, \dots , b_n)  \in  R^{B}\))  if and only if \(R^{ \mathcal {A}}(b_1, \dots , b_n)\) holds (i.e., \((b_1, \dots , b_n)  \in  R^{ \mathcal {A}}\)) 
Note that this condition is vacuously true, so only the first two conditions need to be checked when verifying a substructure.}
\end{tree}

\end{tree}


  
  
\begin{tree}{title={Formulas and Models}, taxon={section}, slug={BMT-s002}}
In this section we define formulas, elementary equivalence, elementary substructures, theories, and models
\begin{tree}{title={Definition of a term}, taxon={definition}, slug={BMT-d007}}
Given a language \(\mathcal {L}\), we define a \emph{term} to be any of the following:\par{If \(c \in   \mathcal {L}\) is a constant symbol, then \(c\) is a term}\par{If \(x\) is a variable symbol, then \(x\) is a term. We generally assume we have countably infinitely many variable symbols}\par{If \(t_1,..., t_n\) are terms, and \(f \in   \mathcal {L}\) is an \(n\)-ary function, then \(f(t_1,..., t_n)\) is a term.}
\end{tree}

\begin{tree}{title={Definition of an atomic formula}, taxon={definition}, slug={BMT-d008}}
Given a language \(\mathcal {L}\), an \emph{atomic formula} is defined as follows:\par{If \(R \in   \mathcal {L}\) is an n-ary relation symbol, and \(t_1, \dots , t_n\) are \ForesterRef{BMT-d007}{terms}, then \(R(t_1, \dots , t_n)\) is an atomic formula.}\par{If \(t_1\) and \(t_2\) are \(\mathcal {L}\)-terms, then \(t_1 = t_2\) is an atomic formula. }
\end{tree}

\begin{tree}{title={Definition of an \(\mathcal {L}\)-formula}, taxon={definition}, slug={BMT-d009}}
Given a language \(\mathcal {L}\), an \(\mathcal {L}\)-formula is defined inductively as follows:\par{If \(\phi\) is an \(\mathcal {L}\)-\ForesterRef{BMT-d008}{atomic formula} then \(\phi\) is a formula}\par{If \(\phi\) and \(\psi\) are both formulas, then \(\neg \phi\), \(\phi   \land   \psi\), and \(\phi \lor \phi\) are formulas, where \(\lor , \neg , \land\) are the usual \ForesterRef{BMT-d010}{logical connectives}}\par{If \(\phi\) is a formula, then \(\exists  x  \phi\) and \(\forall  x  \phi\) are formulas, where \(\exists\) and \(\forall\) are \ForesterRef{BMT-d011}{quantifiers}}
\end{tree}

\end{tree}


  
  
\begin{tree}{title={Quantifier Elimination}, taxon={section}, slug={BMT-s003}}
In this section we give the basic definitions and theorems for quantifier Elimination
\end{tree}


  
  
\begin{tree}{title={Back and Forth}, taxon={section}, slug={BMT-s004}}
In this section we give the basic definitions and theorems for quantifier Elimination
\end{tree}


  
  
\begin{tree}{title={Types}, taxon={section}, slug={BMT-s005}}
In this section we give the basic definitions and theorems for types
\end{tree}


  
  
\begin{tree}{title={Saturation}, taxon={section}, slug={BMT-s006}}
In this section we give the basic definitions and theorems for saturation of models.
\end{tree}


  
  
\begin{tree}{title={Ultraproducts}, taxon={section}, slug={BMT-s007}}
In this section we give the basic definitions and theorems for the use of ultraproducts in model theory.
\end{tree}


\end{tree}


  
  
\begin{tree}{title={Dividing lines}, taxon={part}, slug={mon-p001}}
Dividing lines play a fundamental role in model theory. They form the basis of Shelah's approach to classification theory and have been a central influence on model theory since their conception. Here is an interactive (though not complete) map of the \href{https://forkinganddividing.com}{model theoretic universe}, maintained by Gabriel Conant. Important examples of dividing lines include stability, NIP, and o-minimality.
  
  
\begin{tree}{title={Stability Theory}, taxon={section}, slug={mon-s002}}
\textbf{Stable theories}
\begin{tree}{title={Forking and Dividing}, taxon={subsection}, slug={mon-s003}}
\textbf{Forking and Dividing}
\end{tree}

\end{tree}


    
    
\begin{tree}{title={NIP Theories}, taxon={section}, slug={mon-s001}}
\textbf{NIP theories} are a class of theories generalizing stable
theories, but allowing for an ordering. Aside from stable theories,
important examples are real closed fields, ACVF, \(p\)-adically closed
fields, and o-minimal theories\par{Fix a complete theory \(T\) with monster model \(\mathbb {M}\).
Also fix a formula \(\varphi (x;y)\) with a fixed partitioning into
the two tuples \(x\) and \(y\).}
\begin{tree}{title={Definition of the independence property}, taxon={definition}, slug={mon-d004}}
The formula \(\varphi (x;y)\) has the \emph{independence property} if there
are sequences of tuples \((a_i : i  \in   \omega )\) and
\((b_S : S  \subseteq   \omega )\) such that for every subset \(S  \subseteq   \omega\)
\[i  \in  S  \Longleftrightarrow   \mathbb {M}  \models   \varphi (a_i; b_S).\]
\end{tree}

\begin{tree}{title={Definition of NIP}, taxon={definition}, slug={mon-d005}}
The formula \(\varphi (x;y)\) is said to be \emph{NIP} if it does not have
the \href{mon-0004}{independence property}.
\end{tree}

\begin{tree}{title={Characterizations of NIP for a formula}, taxon={theorem}, slug={mon-t001}}
The following conditions are equivalent:
\begin{enumerate}
\item{ The formula \(\varphi (x;y)\) has the independence property.}
  \item{ The formula \(\varphi ^{ \vee }(y;x)\) has the independence property,
	where \(\varphi ^{ \vee }(y;x)\) is the formula \(\varphi (x;y)\) with the
	opposite partition.}
	\item{ For any two finite sets \(U\) and \(V\), and any subset \(R  \subseteq  U  \times  V\), there are \((a_i : i  \in  U)\) and \((b_j : j  \in  U)\) such that \(\mathbb   \models   \varphi (a_i;b_j)  \Longleftrightarrow  (i,j)  \in  R\).}
	\item{ There is an indiscernible sequence \((a_i : i  \in   \omega )\)
	and some \(b\) such that \(\mathbb  M  \models   \varphi (a_i;b)  \Longleftrightarrow  i\)
	is even.}
\end{enumerate}\par{We are also interested in the following characterization, which is more amenable to computations.}
\begin{tree}{title={Alternation Number}, taxon={definition}, slug={mon-d007}}
The \emph{alternation number} of a formula \(\varphi (x;y)\), denoted \(\operatorname {alt}( \varphi (x;y))\) is the maximal number \(n  \in   \omega\) (if it exists) such that there is an indiscernible sequence \((a_i : i  \in   \omega )\), some \(b\), and indices \(i_0 <  \dots  < i_n\) with \(\mathbb  M  \models   \varphi (a_i,b)  \Longleftrightarrow  i  \text { is even}\). If no such maximum exists, we let \(\operatorname {alt}( \varphi (x;y)) =  \infty\).
\end{tree}

\begin{tree}{title={Alternation Lemma}, taxon={lemma}, slug={mon-d008}}
A formula \(\varphi (x;y)\) is NIP if and only if \(\operatorname {alt}( \varphi (x;y)) <  \infty\).
\end{tree}

\end{tree}

\end{tree}


\end{tree}


  
  
\begin{tree}{title={Infinitary Model Theory}, taxon={part}, slug={inf-p001}}
We present the fundamentals of infinitary model theory. The main reference will be David Marker's book Lectures on Infinitary Logic
\end{tree}


  
  
\begin{tree}{title={Monadic Expansions}, taxon={part}, slug={moe-p001}}
Model Theory of Monadic Expansions
  
  
\begin{tree}{title={Definition of }, taxon={section}, slug={moe-s001}}

\end{tree}


\end{tree}


\printbibliography
\end{document}