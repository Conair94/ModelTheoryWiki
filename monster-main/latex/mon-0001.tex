\documentclass[a4paper]{article}
\usepackage[final]{microtype}
\usepackage{fontspec}
\setmonofont{inconsolata}
\usepackage{amsmath,amsthm,amssymb,stmaryrd,mathtools,biblatex,forester}
\addbibresource{forest.bib}

\title{The Model Companion}

\date{}

\author{Connor Lockhart \and Adam Melrod \and Morgan Bryant \and Francis Westhead \and Garrett Peters \and Oscar Coppola \and Ruohan Hu}

\begin{document}
\maketitle
\par{This is the potential start of a new online resource on model theory, replacing the (now defunct) model theory wiki.}\par{This website is made with the software Forester (http://www.jonmsterling.com/jms-005P.xml). Forester allows you to explore the site in many interesting ways, we have provided a structured starting off points as in below but you may search for any page using Command K}\par{The site is nonlinear in its arrangement of information. At the high level, the website is broken into parts which are comprised of sections. Currently parts correspond to large themes or subjects in model theory and the sections subdivide these themes further. These parts and sections may not be disjoint but the nature of forester is well suited to navigating these overlaps since definitions and theorems can be transcluded easily. }\par{In the future it would be nice to have parts which organize the material as one would encounter it as a course or set of lecture notes intended for self study but for now, we intend to set up the backbone as a wikipedia or stacks project reference document and then reassemble the information after a critical mass has been achieved. }\textbf{Here are the parts which currently exist or are being worked on}
  
  
\begin{tree}{title={Basic Model Theory}, taxon={part}, slug={BMT-p001}}
We give the common definitions and theorems for basic/introductory model theory
  
  
\begin{tree}{title={Structures, Isomorphisms, Substructures}, taxon={section}, slug={BMT-s001}}
In this section we give the basic notions and theorems for structures, isomorphisms, and substructures
\begin{tree}{title={Language}, taxon={definition}, slug={BMT-d001}}
A language, also called a vocabulary or signature by different authors, is a set \(\mathcal {L}\) consisting of symbols for constants, relations, and functions, 
often denoted by \(c\), \(R\), and \(f\) respectively. Languages may have any cardinality.
\end{tree}

\begin{tree}{title={Structure}, taxon={definition}, slug={BMT-d002}}
Given a \ForesterRef{BMT-d001}{language} \(\mathcal {L}\), an \(\mathcal {L}\)-structure \(\mathcal {M}\) in this language has a universe \(M\) (often, the structure and its universe 
are written the same, by abuse of notation). The structure \(\mathcal {M}\) "interprets" the symbols of \(\mathcal {L}\) as follows:\par{For a constant symbol \(c \in   \mathcal {L}\), the interpretation of \(c\) in \(\mathcal {M}\), denoted \(c^{ \mathcal {M}}\) represents a fixed 
element of \(M\)}\par{For an \(n\)-ary function symbol \(f \in   \mathcal {L}\), the interpretation of \(f\) in \(\mathcal {M}\), denoted \(f^{ \mathcal {M}}\),
is a function from \(M^n\) to \(M\)}\par{For an \(n\)-ary relation symbol \(R \in   \mathcal {L}\), the interpretation of \(R\) in \(\mathcal {M}\), denoted \(R^{ \mathcal {M}}\),
is a subset \(M^n\)}\par{Often, we are interested in the cardinality of a structure, denoted \(| \mathcal {M}|\), which is defined to be the cardinality of its universe.}
\end{tree}

\begin{tree}{title={Homomorphism}, taxon={definition}, slug={BMT-d003}}
Given two \ForesterRef{BMT-d002}{structures} \(\mathcal {A}\) and \(\mathcal {B}\) in a \ForesterRef{BMT-d001}{language} \(\mathcal {L}\), a \emph{homomorphism} \(\varphi :  \mathcal {A}  \rightarrow   \mathcal {B}\)
is a function between the universes \(A\) and \(B\) of \(\mathcal {A}\) and \(\mathcal {B}\) respectively such that:\par{ For every n-ary function \(f \in   \mathcal {L}\), for all \(a_1, \dots , a_n \in  A\), \(\varphi (f^{ \mathcal {A}}(a_1, \dots , a_n)) = f^{ \mathcal {B}}( \varphi (a_1), \dots ,  \varphi (a_n))\)}\par{For every n-ary relation \(R \in   \mathcal {L}\), for all \(a_1, \dots , a_n  \in  A\), \(\varphi (R^{ \mathcal {A}}(a_1, \dots , a_n))  \text { holds }  \Rightarrow  R^{ \mathcal {B}}( \varphi (a_1),  \dots ,  \varphi (a_n))\) holds}\par{For every constant symbol \(c  \in   \mathcal {L}\), \(\varphi (c^{ \mathcal {A}}) =c^{ \mathcal {B}}\)}
\end{tree}

\begin{tree}{title={Embedding of \(L\)-structures}, taxon={definition}, slug={BMT-d004}}
Given two structures in a language \(\mathcal {L}\) \(\mathcal {A}\) and \(\mathcal {B}\), an \emph{embedding} \(\varphi :  \mathcal {A}  \rightarrow   \mathcal {B}\)
is a \ForesterRef{BMT-d003}{homomorphism} such that:\par{For every n-ary relation \(R \in   \mathcal {L}\), for all \(a_1, \dots , a_n  \in  A\), \(\varphi (R^{ \mathcal {A}}(a_1, \dots , a_n))  \text { holds }  \Leftrightarrow  R^{ \mathcal {B}}( \varphi (a_1),  \dots ,  \varphi (a_n))\) holds}\par{This is stronger than a homomorphism because we now require a two way implication in the above property.}
\end{tree}

\begin{tree}{title={Isomorphism}, taxon={definition}, slug={BMT-d005}}
An \ForesterRef{BMT-d004}{embedding} \(\varphi :  \mathcal {A}  \rightarrow   \mathcal {B}\) between \(\mathcal {L}\)-structures is a \emph{isomorphism} if it is surjective.

An automorphism of a \ForesterRef{BMT-d002}{structure} \(\mathcal {A}\) is an isomorphism to itself.

\end{tree}

\begin{tree}{title={Substructure}, taxon={definition}, slug={BMT-d006}}
Given a \(\mathcal {L}\)-structure \(\mathcal {A}\), a subset \(B  \subseteq  A\) is called a \emph{substructure} of \(\mathcal {A}\) if:\par{For every constant \(c \in   \mathcal {L}\), \(c^{ \mathcal {A}}  \in  B\)}\par{For every n-ary function \(f \in   \mathcal {L}\), for any \(b_1, \dots , b_n  \in  B\), \(f^{ \mathcal {A}}(b_1, \dots ,b_n)  \in  B\)}\par{For every n-ary relation, \(R \in   \mathcal {L}\), for any \(b_1, \dots , b_n  \in  B\), we write \(R^B = R^{ \mathcal {A}} \cap   \mathcal {P}(B^n)\) and require that
\(R^{B}(b_1, \dots , b_n)\) holds (i.e., \((b_1, \dots , b_n)  \in  R^{B}\))  if and only if \(R^{ \mathcal {A}}(b_1, \dots , b_n)\) holds (i.e., \((b_1, \dots , b_n)  \in  R^{ \mathcal {A}}\)) 
Note that this condition is vacuously true, so only the first two conditions need to be checked when verifying a substructure.}
\end{tree}

\begin{tree}{title={Elementary Embedding of \(L\)-structures}, taxon={definition}, slug={BMT-d030}}

    Given two \ForesterRef{BMT-d002}{structures} \(\mathcal {A}\) and \(\mathcal {B}\) in a \ForesterRef{BMT-d001}{language} \(\mathcal {L}\),
    a \ForesterRef{BMT-d003}{homomorphism} \(\Phi \colon \mathcal {A}  \to   \mathcal {B}\) is an \emph{elementary embedding} when
    for every \(\mathcal  L\)-formula \(\varphi (x_1, \dots ,x_n)\) and \(a_1, \dots ,a_n \in  A\),
    we have \(\mathcal  A \vDash \varphi (a_1, \dots ,a_n)\) if and only if \(\mathcal  B \vDash \varphi ( \Phi (a_1), \dots , \Phi (a_n))\).

    A partial homomorphism \(\Phi : \mathcal {A} \to \mathcal {B}\) can also be said to be an elementary embedding when
    for every \(\mathcal  L\)-formula \(\varphi (x_1, \dots ,x_n)\) and \(a_1, \dots ,a_n \in  dom \Phi\),
    we have \(\mathcal  A \vDash \varphi (a_1, \dots ,a_n)\) if and only if \(\mathcal  B \vDash \varphi ( \Phi (a_1), \dots , \Phi (a_n))\).



\end{tree}

\begin{tree}{title={Elementary Substructures}, taxon={definition}, slug={BMT-d031}}

    Given two \ForesterRef{BMT-d002}{structures} \(\mathcal {A}\) and \(\mathcal {B}\) in a \ForesterRef{BMT-d001}{language} \(\mathcal {L}\)
    such that \(\mathcal  A\) is a \ForesterRef{BMT-d006}{substructure} of \(\mathcal  B\),
    \(\mathcal  A\) is an \emph{elementary substructure} of \(\mathcal  B\) when the inclusion map \(\mathcal  A \hookrightarrow \mathcal  B\)
    is an \ForesterRef{BMT-d030}{elementary embedding}.
    Conversely, \(\mathcal  B\) is called an elementary extension of \(\mathcal  A\).
\par{
    One writes \(\mathcal  A \preceq \mathcal  B\) to mean \(\mathcal  A\) is an elementary substructure of \(\mathcal  B\).
}
\end{tree}

\end{tree}


  
  
\begin{tree}{title={Formulas and Models}, taxon={section}, slug={BMT-s002}}
In this section we define formulas, elementary equivalence, elementary substructures, theories, and models
\begin{tree}{title={Term}, taxon={definition}, slug={BMT-d007}}
Given a language \(\mathcal {L}\), we define a \emph{term} to be any of the following:\par{If \(c \in   \mathcal {L}\) is a constant symbol, then \(c\) is a term}\par{If \(x\) is a variable symbol, then \(x\) is a term. We generally assume we have countably infinitely many variable symbols}\par{If \(t_1,..., t_n\) are terms, and \(f \in   \mathcal {L}\) is an \(n\)-ary function, then \(f(t_1,..., t_n)\) is a term.}
\end{tree}

\begin{tree}{title={Atomic Formula}, taxon={definition}, slug={BMT-d008}}
Given a language \(\mathcal {L}\), an \emph{atomic formula} is defined as follows:\par{If \(R \in   \mathcal {L}\) is an n-ary relation symbol, and \(t_1, \dots , t_n\) are \ForesterRef{BMT-d007}{terms}, then \(R(t_1, \dots , t_n)\) is an atomic formula.}\par{If \(t_1\) and \(t_2\) are \(\mathcal {L}\)-terms, then \(t_1 = t_2\) is an atomic formula. }
\end{tree}

\begin{tree}{title={\(\mathcal {L}\)-Formula}, taxon={definition}, slug={BMT-d009}}
Given a language \(\mathcal {L}\), an \(\mathcal {L}\)-formula is defined inductively as follows:\par{\begin{enumerate}
\item{
            If \(\phi\) is an \(\mathcal {L}\)-\ForesterRef{BMT-d008}{atomic formula} then \(\phi\) is a formula.
        }
        \item{
            If \(\phi\) and \(\psi\) are both formulas, then \(\neg \phi\), \(\phi   \land   \psi\), and 
            \(\phi \lor \phi\) are formulas, where \(\lor , \neg , \land\) are the usual \ForesterRef{BMT-d010}{logical connectives}.
        }
        \item{
            If \(\phi\) is a formula, then \(\exists  x  \phi\) and \(\forall  x  \phi\) are formulas, 
            where \(\exists\) and \(\forall\) are \ForesterRef{BMT-d011}{quantifiers}.
        }
        \item{
            If \(\phi\) is a formula, then \(\exists  x  \phi\) and \(\forall  x  \phi\) are formulas, 
            where \(\exists\) and \(\forall\) are \ForesterRef{BMT-d011}{quantifiers}.
        }
\end{enumerate}}
\end{tree}

\begin{tree}{title={Definition of an \(\mathcal {L}\)-theory}, taxon={definition}, slug={BMT-d017}}

    Given a \ForesterRef{BMT-d001}{language} \(\mathcal {L}\), a \emph{\(\mathcal {L}\)-theory} is a set of \ForesterRef{BMT-d013}{\(\mathcal {L}\)-sentences}.
    Sometimes it is notationally convenient to assume that theories are deductively closed.
\par{
    An \(\mathcal  L\)-theory is \emph{complete} when for every \ForesterRef{BMT-d009}{\(\mathcal  L\)-formula} \(\phi\) we have that
    \(\phi\) may be deduced from \(T\) or \(\neg \phi\) may be deduced from \(T\).
}
\end{tree}

\end{tree}


  
  
\begin{tree}{title={Quantifier Elimination}, taxon={section}, slug={BMT-s003}}
Quantifier Elimination is a part of a broader technique in Model Theory where for a structure in a given language, an arbitrary formula can be written as a boolean combination of perhaps simpler formulas
\begin{tree}{title={Elimination Set}, taxon={definition}, slug={BMT-d014}}
An \emph{Elimination Set} for a language \(\mathcal {L}\) and class \(K\) of \(\mathcal {L}\)-structures, then a set \(\Gamma\) of formulas \(\phi\) is an elimination set for \(K\) if for every formula \(\phi ( \bar {x})\) of \(\mathcal {L}\) there is a formula \(\phi ^*( \bar {x})\) which is a boolean combinations of formulas in \(\Gamma\) and \(\phi\) is equivalent to \(\phi ^*\) in every structure in \(K\)
\end{tree}
\par{In particular, we will be most interested in elimination sets that are comprised of the set of quantifier free formulas. It is worth noting that, in some cases it is not possible to have a full quantifier elimination down to the level of a quantifier free set but perhaps we can restrict ourself to some reasonable set of formulas.}\par{Frequently, we use the following lemma to simplify the task of showing that \(\Phi\) is an elimination set}
\begin{tree}{title={Elimination of Atomic Formulas}, taxon={lemma}, slug={BMT-d015}}
Let \(K\) be a class of \(L\) structures and \(\Phi\) be a set of \(L\)-formulas. Denote \(\Phi ^-\) as the set of negations of formulas in \(\Phi\).\par{Suppose that 
  \begin{enumerate}
\item{every atomic formula of \(L\) is in \(\Phi\) and} 
    \item{for every formula \(\theta ( \overline {x})\) of \(L\) which is of the form \(\exists  y  \bigwedge _{i<n}  \psi _i ( \overline {x},y)\) with each \(\psi _i  \in   \Phi \cup   \Phi ^-\), there is an \(L\)-formula \(\theta ^*( \overline {x})\) which is a boolean combination of formulas in \(\Phi\) and is equivalent to \(\theta\) in every structure in \(K\).}
\end{enumerate}
  Then \(\Phi\) is an elimination set of \(K\)}
\end{tree}
\par{The art of quantifier elimination lies in choosing an appropriate elimination set that allows the above lemma to be used with minimal obstructions. }\par{There are many examples of theories which have quantifier elimination for a relatively tame set of formulas: }
  
  
\begin{tree}{title={Algebraically Closed Fields}, taxon={example}, slug={BMT-e031}}

    The theory of \emph{algebraically closed fields} (ACF) is a \ForesterRef{BMT-d017}{theory} in the \ForesterRef{BMT-d001}{language} \(\mathcal  L= \{ 0,1,+, \cdot \}\)
    consisting of the following \ForesterRef{BMT-d013}{sentences}.
    \begin{enumerate}
\item{\(\forall  x \forall  y \forall  z \, (x+y)+z=x+(y+z)\).
        }
        \item{\(\forall  x \, x+0=x \land0 +x=x\).
        }
        \item{\(\forall  x \exists  y \, x+y=0 \land  y+x=0\).
        }
        \item{\(\forall  x \forall  y \, x+y=y+x\).
        }
        \item{\(\forall  x \forall  y \forall  z \, (x \cdot  y) \cdot  z=x \cdot  (y+ \cdot  z)\).
        }
        \item{\(\forall  x \, x \cdot1 =x \land1 \cdot  x=x\).
        }
        \item{\(\forall  x \exists  y \, x \cdot  y=1 \land  y \cdot  x=1\).
        }
        \item{\(\forall  x \forall  y \, x+ \cdot  y=y \cdot  x\).
        }
        \item{\(\forall  x \forall  y \forall  z \, x \cdot (y+z)=(x \cdot  y)+(x \cdot  z)\).
        }
        \item{
            For each \(n \in \mathbb  N\), a sentence \(\forall  y_1 \cdots \forall  y_n \exists  x \, (y_1 \cdot  x)+ \cdots +(y_n \cdot  x)=0\).
        }
\end{enumerate}\par{
    The theory of algebraically closed fields of characteristic \(p\) (\(\text {ACF}_p\)) is ACF together with the sentence
    \(1+ \cdots +1=0\), where \(1\) is being added \(p\) times. \(\text {ACF}_p\) is a \ForesterRef{BMT-d017}{complete theory} having
    \ForesterRef{BMT-s003}{quantifier elimination}.
}\par{
    The theory of algebraically closed fields of characteristic \(0\) (\(\text {ACF}_0\)) is ACF together with the collection of sentences
    of the form \(\neg (1+ \cdots +1=0)\), for any number of \(1\)s being added together. \(\text {ACF}_0\) is also a
    \ForesterRef{BMT-d017}{complete theory} having \ForesterRef{BMT-s003}{quantifier elimination}.
}
\end{tree}


  
  
\begin{tree}{title={DLO}, taxon={example}, slug={BMT-e030}}

    The \emph{dense linear order without endpoints} (DLO) is a \ForesterRef{BMT-d017}{theory} in the \ForesterRef{BMT-d001}{language} \(\mathcal  L= \{ < \}\)
    consisting of the following \ForesterRef{BMT-d013}{sentences}.
    \begin{enumerate}
\item{\(\forall  x \, \neg  x<x\)}
        \item{\(\forall  x \forall  y \, (x<y \rightarrow \neg (y<x))\)}
        \item{\(\forall  x \forall  y \forall  z \, ((x<y \land  y<z) \rightarrow  x<z)\)}
        \item{\(\forall  x \forall  y \, (( \neg  x=y) \rightarrow (x<y \lor  y<x))\)}
        \item{\(\forall  x \forall  y \, x<y \rightarrow ( \exists  z \, x<z \land  z<y)\)}
        \item{\(\forall  x \exists  z \, z<x\)}
        \item{\(\forall  x \exists  z \, x<z\)}
\end{enumerate}\par{
    The set of rationals \(\mathbb  Q\) with the usual ordering is a model of DLO, and DLO is \ForesterRef{BMT-d201}{\(\aleph _0\)-categorical},
    so every countable model is isomorphic to \(( \mathbb  Q,<)\). DLO is also a \ForesterRef{BMT-d017}{complete theory} with
    \ForesterRef{BMT-s003}{quantifier elimination}.
}
\end{tree}

\par{Furthermore, if one is willing to accept arbitrary expansions of the language, we can always force a theory to have quantifier elimination. This is done as follows:}
\begin{tree}{title={Morleyisation}, taxon={definition}, slug={BMT-d026}}
 The Morleyisation of an \(L\)-theory \(T\) is the theory \(T^m\) in the language \(L^m \supset  L\). The expanded language \(L^m\) is formed by taking every \(L\)-formula \(\phi (x_1,...,x_n)\) and adding an \(n\)-placed relation symbol \(R_ \phi\). To \(T\) we add the axioms \par{\(\forall  x_1,...,x_n (R_ \phi  (x_1,...,x_n) \leftrightarrow \phi (x_1,...,x_n)) \)}\par{Morleyisation preserves many properties such as \ForesterRef{BMT-d201}{\(\kappa\)-categoricity}, though the relations \(R_ \phi\) and the corresponding definable sets may be quite hard to understand. }
\end{tree}

\end{tree}


  
  
\begin{tree}{title={Back and Forth}, taxon={section}, slug={BMT-s004}}
In this section we give the basic definitions and theorems for quantifier Elimination
\end{tree}


  
  
\begin{tree}{title={Types}, taxon={section}, slug={BMT-s005}}
In this section we give the basic definitions and theorems for types.
\begin{tree}{title={Definition of a type}, taxon={definition}, slug={BMT-d019}}
Given a \ForesterRef{BMT-d001}{language}, \(\mathcal {L}\), and a \ForesterRef{BMT-d017}{\(\mathcal {L}\)-theory \(T\)}, a \emph{partial type} of \(T\) is a \ForesterRef{BMT-d022}{finitely satisfiable} set of formulas in a fixed tuple of variables \(x\).\par{A partial type \(p(x)\) (meaning that the denoted type consists of formulas each in the variables \(x\)) is \emph{complete} if for every \(\mathcal {L}\)-formula in the variables \(x\), \(\varphi (x) \in  p(x)\), either \(\varphi (x) \in  p(x)\) or \(\neg   \varphi (x) \in  p(x)\).}
\end{tree}

\begin{tree}{title={Definition of an isolated/principal type}, taxon={definition}, slug={BMT-d020}}

    Given a \ForesterRef{BMT-d001}{language}, \(\mathcal {L}\), and a \ForesterRef{BMT-d017}{\(\mathcal {L}\)-theory \(T\)},
    a partial type of T {p(x)} is \emph{principal} when there is an \(\mathcal {L}\)-formula \(\varphi (x)\) such that
    \(\models   \exists  x  \varphi (x)\) and for every \(\varpsi (x)  \in  p(x)\) we have that
    \(\models   \forall  x ( \varphi (x)  \rightarrow   \varpsi (x))\).
    In the case that \(p(x)\) is complete, we must have that \(\varphi (x) \in  p(x)\),
    and the first condition is redundant.
\par{
    A complete type of T, \(p(x)\) is \emph{isolated} if it is principal as a partial type.
    Equivalently, {p(x)} is \emph{isolated} if \(\{ p(x) \}\) is open \ForesterRef{BMT-d021}{S_n(T)}.
    This coincides with the usual topological terminology.
}
\end{tree}

\begin{tree}{title={The Stone Space \(S_n(T)\)}, taxon={definition}, slug={BMT-d021}}
Given a \ForesterRef{BMT-d001}{language}, \(\mathcal {L}\), and a \ForesterRef{BMT-d017}{\(\mathcal {L}\)-theory \(T\)},
\(S_n(T)\) denotes a topological space whose underlying set is the set of all complete \emph{n-types} of \(T\).
The topology on \(S_n(T)\) has a basis of open sets given by all sets of the form \([ \varphi (x)] :=  \{ p(x) \in  S_n(T):
 \varphi (x)  \in  p(x) \}\) (for \(\varphi (x)\) an \(\mathcal {L}\)-formula).
This space is a Stone Space in the usual sense and the Boolean Algebra associated to it via Stone Duality is
the \(n\)th Tarski-Lindenbaum Algebra of \(T\).
\end{tree}

\begin{tree}{title={Finite Satisfiability}, taxon={definition}, slug={BMT-d023}}

    An \ForesterRef{BMT-d017}{\(\mathcal  L\)-theory} \(T\) is \emph{finitely satisfiable} when every finite subset of \(T\) is \ForesterRef{BMT-d022}{satisfiable}.

\end{tree}

\end{tree}


  
  
\begin{tree}{title={Saturation}, taxon={section}, slug={BMT-s006}}
In this section we give the basic definitions and theorems for saturation of models.
\begin{tree}{title={Saturated Model}, taxon={definition}, slug={BMT-d209}}

    A \ForesterRef{BMT-d002}{structure} \(M\) is \(\lambda\)-saturated, for a cardinal \(\lambda\), when for every \(A \subseteq  M\) such that \(|A|< \lambda\), every \ForesterRef{BMT-d019}{type} over \(A\) is realized in \(M\). \(M\) is simply said to be saturated if \(M\) is \(|M|\)-saturated.

\end{tree}

\begin{tree}{title={Saturated Model}, taxon={definition}, slug={BMT-d210}}

    A \ForesterRef{BMT-d002}{structure} \(M\) is \(\lambda\)-homogeneous, for a cardinal \(\lambda\), 
    when for every cardinal \(\eta < \lambda\) and sequences \((a_i \in  M|i \in \eta ), \{ b_i \in  M|i \in \eta \}\) such that \((M,(a_i)_{i< \eta }) \equiv (M,(b_i)_{i< \eta })\), 
    for each \(c \in  M\) there is \(d \in  M\) such that \((M,(a_i)_{i< \eta },c) \equiv (M,(b_i)_{i< \eta },d)\). 

    Here \((M,(a_i)_{i< \eta }),(M,(b_i)_{i< \eta })\) refers to \(M\) with its language expanded by a constant \(C_i\) for each \(i< \eta\), such that \(C_i\) is interpreted as \(a_i\), \(b_i\) respectively. 
    \((M,(a_i)_{i< \eta },c)\) and \((M,(b_i)_{i< \eta },d)\) are further expansions, when the language is further expanded by a constant \(D\) that is interpreted as \(c\) or \(d\) respectively.

    The definition here follows Chang & Keisler, and is equivalent to the definition in Shelah.

\end{tree}

\begin{tree}{title={Saturated Model}, taxon={definition}, slug={BMT-d211}}

    A countable \ForesterRef{BMT-d002}{structure} \(M\) is ultrahomogeneous, per Hodges, when every \ForesterRef{BMT-d005}{isomorphism} between two finite \ForesterRef{BMT-d006}{substructure} of \(M\) extends to a global \ForesterRef{BMT-d005}{automorphism}.

    Ultrahomogeneous is strictly stronger than \(\omega\)-\ForesterRef{BMT-d210}{homogeneous}. For an \(\omega\)-\ForesterRef{BMT-d210}{homogeneous} structure that is not ultrahomogeneous, consider the disjoint union of a countable \ForesterRef{Fra-e001}{random graph} and a countable triangle-free random graph.

\end{tree}

\begin{tree}{title={}, taxon={}, slug={BMT-d212}}

\end{tree}

\begin{tree}{title={}, taxon={}, slug={BMT-d213}}

\end{tree}

\end{tree}


  
  
\begin{tree}{title={Ultraproducts}, taxon={section}, slug={BMT-s007}}
In this section we give the basic definitions and theorems for the use of ultraproducts in model theory.
\begin{tree}{title={Definition of a Filter}, taxon={definition}, slug={BMT-d071}}
Let \(I\) be any set. Then a filter \(\mathcal {F}\) on the power set \(\mathcal {P}(I)\) is a collection of subsets of \(I\) with the following properties:\par{1: \(I \in \mathcal {F}\).}\par{2: If \(A \in   \mathcal {F}\) and \(A \subseteq  B\), then \(B \in   \mathcal {F}\).}\par{3: If \(A,B  \in   \mathcal {F}\), then we have \(A \cap  B \in   \mathcal {F}\).}\par{Furthermore, a filter is proper if \(\emptyset \notin \mathcal {F}\).}
\end{tree}

\begin{tree}{title={Definition of an Ultrafilter}, taxon={definition}, slug={BMT-d072}}
Let \(I\) be any set. Then an ultrafilter \(\mathcal {U}\) on \(\mathcal {P}(I)\) is a proper \ForesterRef{BMT-d071}{filter} on \(I\) with the additional property:\par{For every \(A \subseteq  I\), either \(A \in \mathcal {U}\) or \(I \backslash  A \in \mathcal {U}\).}\par{Every proper filter can be extended into an ultrafilter.}
\end{tree}

\begin{tree}{title={Principal Ultrafilters}, taxon={example}, slug={BMT-e071}}
Fix any \(i \in  I\). Then \(\mathcal {U}_i :=  \{ A \subseteq  I: i \in  A \}\) is an \ForesterRef{BMT-d072}{ultrafilter}. This is referred to as the principal ultrafilter concentrated at \(i\).
\end{tree}

\begin{tree}{title={The Fréchet Filter}, taxon={example}, slug={BMT-e072}}
For any set \(I\) of infinite cardinality, the family of cofinite sets forms a proper \ForesterRef{BMT-d071}{filter}. This filter is referred to as the Fréchet filter.\par{When extending the Fréchet filter to an \ForesterRef{BMT-d072}{ultrafilter}, the result is not \ForesterRef{BMT-e071}{principal}.}
\end{tree}

\begin{tree}{title={Ultraproduct}, taxon={definition}, slug={BMT-d073}}
Fix an index set \(I\) and a \ForesterRef{BMT-d001}{language} \(L\). Let \(\{ \mathcal {A}_i: i \in  I \}\) be a family of \(L\)-\ForesterRef{BMT-d002}{structures}, and let \(\mathcal {U}\) be any \ForesterRef{BMT-d072}{ultrafilter} on \(\mathcal {P}(I)\). Then the ultraproduct is the \(L\)-structure \(\mathcal {A}_* :=  \prod \limits _{i \in  I} A_i / \mathcal {U}\). \par{In \(\mathcal {A}_*\), the elements of the universe of \(\mathcal {A}_*\) are functions \(\underline {a}\) with domain \(I\) and \(\underline {a}(i) \in  A_i\) under the equivalence class \(\sim _{ \mathcal {U}}\), where \[\underline {a} \sim _{ \mathcal {U}} \underline {b}  \text { if and only if } \{ i \in  I:  \underline {a}(i)= \underline {b}(i) \}   \in   \mathcal {U}.\]}\par{The constant symbols are interpreted by \(c^{ \mathcal {A}_*} :=  \prod \limits _{i \in  I} c^{ \mathcal {A}_i} / \mathcal {U}\). }\par{The function symbols are interpreted by \(f^{ \mathcal {A}_i}( \underline {a}_1, \dots ,  \underline {a}_n) :=  \prod \limits _{i \in  I} f^{ \mathcal {A}_i}( \underline {a}_1(i), \dots , \underline {a}_n(i)) /  \mathcal {U} \). }\par{The relation symbols are interpreted by \(\mathcal {A}_* \models  R^{ \mathcal {A}_*}( \underline {a}_1, \dots ,  \underline {a}_n)\) if and only if \(\{ i \in  I:  \mathcal {A}_i \models  R^{ \mathcal {A}_i}( \underline {a}_1(i), \dots ,  \underline {a}_n(i)) \} \in   \mathcal {U}\).}
\end{tree}

\begin{tree}{title={ Łoś's Theorem}, taxon={Theorem}, slug={BMT-t071}}
Let \(\mathcal {A}_* =  \prod \limits _{i \in  I}  \mathcal {A}_i /  \mathcal {U}\) be an \ForesterRef{BMT-d073}{ultraproduct}. Then for any \(L\)-formula \(\phi (x_1, \dots ,x_n)\), we have \[\mathcal {A}_* \models   \phi ( \underline {a}_1, \dots , \underline {a}_n)  \text { if and only if }  \{ i \in  I: A_i \models \phi ( \underline {a}_1(i), \dots ,  \underline {a}_n(i)) \} \in \mathcal {U}.\]
\end{tree}

\begin{tree}{title={Compactness Theorem}, taxon={Theorem}, slug={BMT-t072}}
Let \(L\) be any language, and \(T\) be any \(L\)-theory. Then \(T\) is satisfiable if and only if every finite subset \(T_0 \subseteq  T\) is satisfiable.
\end{tree}

\end{tree}


  
  
\begin{tree}{title={Model Completeness}, taxon={Section}, slug={BMT-s008}}
In many important examples of theories, typically originating in the field of algebra, all embeddings are elementary embeddings. We define this as follows
\begin{tree}{title={Model Complete Theory}, taxon={definition}, slug={BMT-d016}}
A Theory \(T\) in a first order language \(L\) is \emph{Model-Complete} if  every \ForesterRef{BMT-d004}{\(L\)-embedding} between models of \(T\) is an elementary embedding. 
\end{tree}
\par{Here we provide a simple test for for determining if something is model complete:}
\begin{tree}{title={Robinson's Test}, taxon={lemma}, slug={BMT-l001}}
 Let \(T\) be a theory. Then the following are equivalent:\begin{enumerate}
\item{\(T\) is model complete.}
    \item{For all models \(M \subseteq  M'\) of \(T\) and all existential setnences \(\phi\) from \(L(M)\), then \(M' \vDash   \phi   \implies  M  \vDash   \phi\)}
    \item{Each formula is, modulo \(T\), equivalent to a universal formula. }
\end{enumerate}\par{Proof: }\begin{enumerate}
\item{\(1 \implies  2\), Given that the definition of \ForesterRef{BMT-d016}{model complete} implies that every embedding of models is an \ForesterRef{BMT-d030}{elementary emedding},then condition 2 above is actually true for all sentences, in particular existential sentences. }
    \item{\(2 \implies  3\), Given an existential formula \(\phi\),  }
    \item{\(3 \implies  1\)}
\end{enumerate}
\end{tree}

\end{tree}


  
  
\begin{tree}{title={Categoricity}, taxon={section}, slug={BMT-s009}}
In this section, we give the basic definitions for \(\kappa\)-categoricity. We then establish some classical characterisations of \(\aleph _0\)-categoricity.
\begin{tree}{title={\(\kappa\)-categoricity}, taxon={definition}, slug={BMT-d201}}

    Given a cardinal \(\kappa\) and a \ForesterRef{BMT-d001}{language} \(\mathcal {L}\), an \ForesterRef{BMT-d017}{\(\mathcal {L}\)-theory} is \emph{\(\kappa\)-categorical} if whenever \(\mathcal {M}\) and \(\mathcal {N}\) are \(\mathcal {L}\)-structures with \(| \mathcal {M}|=| \mathcal {M}|= \kappa\), then \(\mathcal {M}  \cong   \mathcal {N}\). 

\end{tree}

\begin{tree}{title={Tarski-Lindenbaum Algebra}, taxon={definition}, slug={BMT-d202}}
Given a \ForesterRef{BMT-d001}{language}, \(\mathcal {L}\), and an \ForesterRef{BMT-d017}{\(\mathcal {L}\)-theory}, \(T\), the \emph{\(n\)'th Tarski-Lindenbaum algebra of \(T\)} is the set of \ForesterRef{BMT-d009}{\(\mathcal {L}\)-formulas} quotiented by the relation of \ForesterRef{BMT-d018}{\(T\)-equivalence}.
\end{tree}

\begin{tree}{title={(Ryll-Nardjewski) Characterisations of \(\omega\)-categoricity}, taxon={theorem}, slug={BMT-t003}}
For \(\aleph _0\)-categoricity, there are a number of useful characterisations. The following are due to Ryll-Nardjewski.\par{Given a \ForesterRef{BMT-d001}{language}, \(\mathcal {L}\), and an \ForesterRef{BMT-d017}{\(\mathcal {L}\)-theory} \(T\), the following are equivalent:}\par{1 \(T\) is \(\aleph _0\)-categorical.}\par{For every \(n \in   \omega\), there are finitely many \ForesterRef{BMT-d009}{\(\mathcal {L}\)-formulas} in \(n\)-variables up to \ForesterRef{BMT-d018}{\(T\)-equivalence}.}\par{For every \(n \in   \omega\), the \ForesterRef{BMT-d202}{\(n\)'th Tarski-Lindenbaum algebra of \(T\)} is finite. 
Every \ForesterRef{BMT-d019}{type} over \(T\) is \ForesterRef{BMT-d020}{isolated}. }
\end{tree}

\end{tree}


  
  
\begin{tree}{title={Examples}, taxon={section}, slug={BMT-s010}}

    Here, we give the basic and canonical examples of Model Theory, summarizing their properties.

\begin{tree}{title={DLO}, taxon={example}, slug={BMT-e030}}

    The \emph{dense linear order without endpoints} (DLO) is a \ForesterRef{BMT-d017}{theory} in the \ForesterRef{BMT-d001}{language} \(\mathcal  L= \{ < \}\)
    consisting of the following \ForesterRef{BMT-d013}{sentences}.
    \begin{enumerate}
\item{\(\forall  x \, \neg  x<x\)}
        \item{\(\forall  x \forall  y \, (x<y \rightarrow \neg (y<x))\)}
        \item{\(\forall  x \forall  y \forall  z \, ((x<y \land  y<z) \rightarrow  x<z)\)}
        \item{\(\forall  x \forall  y \, (( \neg  x=y) \rightarrow (x<y \lor  y<x))\)}
        \item{\(\forall  x \forall  y \, x<y \rightarrow ( \exists  z \, x<z \land  z<y)\)}
        \item{\(\forall  x \exists  z \, z<x\)}
        \item{\(\forall  x \exists  z \, x<z\)}
\end{enumerate}\par{
    The set of rationals \(\mathbb  Q\) with the usual ordering is a model of DLO, and DLO is \ForesterRef{BMT-d201}{\(\aleph _0\)-categorical},
    so every countable model is isomorphic to \(( \mathbb  Q,<)\). DLO is also a \ForesterRef{BMT-d017}{complete theory} with
    \ForesterRef{BMT-s003}{quantifier elimination}.
}
\end{tree}

\begin{tree}{title={Algebraically Closed Fields}, taxon={example}, slug={BMT-e031}}

    The theory of \emph{algebraically closed fields} (ACF) is a \ForesterRef{BMT-d017}{theory} in the \ForesterRef{BMT-d001}{language} \(\mathcal  L= \{ 0,1,+, \cdot \}\)
    consisting of the following \ForesterRef{BMT-d013}{sentences}.
    \begin{enumerate}
\item{\(\forall  x \forall  y \forall  z \, (x+y)+z=x+(y+z)\).
        }
        \item{\(\forall  x \, x+0=x \land0 +x=x\).
        }
        \item{\(\forall  x \exists  y \, x+y=0 \land  y+x=0\).
        }
        \item{\(\forall  x \forall  y \, x+y=y+x\).
        }
        \item{\(\forall  x \forall  y \forall  z \, (x \cdot  y) \cdot  z=x \cdot  (y+ \cdot  z)\).
        }
        \item{\(\forall  x \, x \cdot1 =x \land1 \cdot  x=x\).
        }
        \item{\(\forall  x \exists  y \, x \cdot  y=1 \land  y \cdot  x=1\).
        }
        \item{\(\forall  x \forall  y \, x+ \cdot  y=y \cdot  x\).
        }
        \item{\(\forall  x \forall  y \forall  z \, x \cdot (y+z)=(x \cdot  y)+(x \cdot  z)\).
        }
        \item{
            For each \(n \in \mathbb  N\), a sentence \(\forall  y_1 \cdots \forall  y_n \exists  x \, (y_1 \cdot  x)+ \cdots +(y_n \cdot  x)=0\).
        }
\end{enumerate}\par{
    The theory of algebraically closed fields of characteristic \(p\) (\(\text {ACF}_p\)) is ACF together with the sentence
    \(1+ \cdots +1=0\), where \(1\) is being added \(p\) times. \(\text {ACF}_p\) is a \ForesterRef{BMT-d017}{complete theory} having
    \ForesterRef{BMT-s003}{quantifier elimination}.
}\par{
    The theory of algebraically closed fields of characteristic \(0\) (\(\text {ACF}_0\)) is ACF together with the collection of sentences
    of the form \(\neg (1+ \cdots +1=0)\), for any number of \(1\)s being added together. \(\text {ACF}_0\) is also a
    \ForesterRef{BMT-d017}{complete theory} having \ForesterRef{BMT-s003}{quantifier elimination}.
}
\end{tree}

\begin{tree}{title={Vector Spaces}, taxon={example}, slug={BMT-e032}}

    The theory of vector spaces over a field \(F\) is a \ForesterRef{BMT-d017}{theory} in the \ForesterRef{BMT-d001}{language} \(\mathcal  L= \{ 0,+,( \times _f)_{f \in  F} \}\)
    consisting of the following \ForesterRef{BMT-d013}{sentences}.
    \begin{enumerate}
\item{\(\forall  x \forall  y \forall  z \, (x+y)+z=x+(y+z)\).
        }
        \item{\(\forall  x \, x+0=x \land0 +x=x\).
        }
        \item{\(\forall  x \exists  y \, x+y=0 \land  y+x=0\).
        }
        \item{\(\forall  x \forall  y \, x+y=y+x\).
        }
        \item{
            For each \(f,g \in  F\), the sentence \(\forall  x \, \times _f( \times _g(x)) =  \times _{fg}(x)\), where \(fg\) is the product of \(f\) and \(g\) in \(F\).
        }
        \item{\(\forall  x \, \times _1(x) = x\) where \(1\) is the multiplicative identity in \(F\).
        }
        \item{
            For each \(f \in  F\), the sentence \(\forall  x \forall  y \, \times _f(x+y) =  \times _f(x)+ \times _f(y)\).
        }
        \item{
            For each \(f,g \in  F\), the sentence \(\forall  x \, \times _(f+g)(x) =  \times _f(x)+ \times _g(x)\), where \(f+g\) is the sum of \(f\) and \(g\) in \(F\).
        }
\end{enumerate}
\end{tree}

\begin{tree}{title={Equivalence Relation}, taxon={example}, slug={BMT-e033}}

    The theory of an equivalence relation is a \ForesterRef{BMT-d017}{theory} in the \ForesterRef{BMT-d001}{language} \(\mathcal  L= \{ E \}\),
    where \(E\) is a binary relation, consisting of the following \ForesterRef{BMT-d013}{sentences}.
    \begin{enumerate}
\item{\(\forall  x \, xEx\).
        }
        \item{\(\forall  x \forall  y \, xEy \rightarrow  yEx\).
        }
        \item{\(\forall  x \forall  y \forall  z \, (xEy \land  yEz) \rightarrow  xEz\).
        }
\end{enumerate}\par{
    Being uncomplicated, this theory is usually shown as an initial example (or rather, counterexample) of various properties.
    It is straightforward to add more restrictions, such as stating that there are only \(n\) equivalence classes:
    \begin{enumerate}
\item{\(\exists  x_1 \cdots \exists  x_n \forall  y \, \bigvee _{i=1}^nx_iEy\).
        }
\end{enumerate}
    Or, that each equivalence class has at most \(k\) elements:
    \begin{enumerate}
\item{\(\forall  x \exists  y_1 \cdots \exists  y_k \forall  z \, (x=z) \rightarrow \bigvee _{i=1}^ny_iEz\).
        }
\end{enumerate}
    Or, that each equivalence class has infinitely many elements:
    \begin{enumerate}
\item{
            For each \(k\), the sentence \(\forall  x_1 \cdots \forall  x_k \exists  y \, \bigwedge _{i=1}^n \neg (x_iEy)\).
        }
\end{enumerate}}
\end{tree}

\begin{tree}{title={Graphs (Rado Graph)}, taxon={example}, slug={BMT-e034}}

    The theory of a graph is a \ForesterRef{BMT-d017}{theory} in the \ForesterRef{BMT-d001}{language} \(\mathcal  L= \{ R \}\), where \(R\) is a binary relation.
    Depending on the author, the theory may only demand symmetry, \(\forall  x \forall  y \, xRy \rightarrow  yRx\),
    or additionally include irreflexivity: \(\forall  x \, \neg  xRx\).
\par{
    The theory of the \emph{Rado graph} (also called the random graph) is an extension of the theory of the graph by the extension axiom schema:
    \begin{enumerate}
\item{
            For each \(n\) and \(m\), the sentence
            \(\forall  x_1 \cdots \forall  x_n \forall  y_1 \cdots \forall  y_m \exists  z \, \bigwedge _{i=1}^nzRx_i \land \bigwedge _{j=1}^m \neg  zRy_j\).
        }
\end{enumerate}
    This ``single'' schema is enough to make the theory \ForesterRef{BMT-d017}{complete},
    and in fact the theory of the Rado graph is \ForesterRef{BMT-d201}{\(\aleph _0\)-categorical}.
}
\end{tree}

\end{tree}


\end{tree}


  
  
\begin{tree}{title={Fraisse Classes and Variants}, taxon={part}, slug={Fra-p001}}
We present Fraisse Classes, Smooth classes, and related results and variants.
\begin{tree}{title={Fraisse Classes}, taxon={section}, slug={Fra-s001}}
We define Fraisse constructions and their properties. Most of the concepts can be referred to in Wilfred Hodges' Model Theory~\cite{Ref-0001}
\begin{tree}{title={Class of Structures}, taxon={definition}, slug={Fra-d001}}
Given a language \(\mathcal {L}\), a class \(K\) is defined to be a set of \(\mathcal {L}\)-structures\par{Often, we want there to be a relation \(\leq\) acting on the structures in \(K\), and when we do, we write the class and relation as a pair \((K,  \leq )\)}
\end{tree}

\begin{tree}{title={Age}, taxon={definition}, slug={Fra-d002}}
Given a structure \(A\) in a language \(\mathcal {L}\), the \emph{Age} (pronounced ah-zh) of \(A\) is \[\{ B \subseteq  A: B  \text { is finitely generated } \}\]\par{The age of a structure is itself a class, often under the relation of substructure}
\end{tree}

\begin{tree}{title={Amalgamation Property (AP)}, taxon={definition}, slug={Fra-d003}}
Given a class and relation \((K, \leq )\), we say that \(K\) has the \emph{Amalgamation Property} if for every \(A,B,C  \in  K\), if there 
exists an embedding \(f:A \rightarrow  B\) such that \(f(A)  \leq  B\) and and embedding \(g:A \rightarrow  C\) such that \(A \leq  B\), then there exists a structure \(D\) 
and embeddings \(h_1:C \rightarrow  D\) and \(h_2:B \rightarrow  D\) such that \(h_1(C) \leq  D\) and \(h_2(B)  \leq  D\) and \(h_2(f(A)) = h_1(g(A))\)
\end{tree}

\begin{tree}{title={Joint Embedding Property (JEP)}, taxon={definition}, slug={Fra-d004}}
Given a class and relation \((K, \leq )\), we say that \(K\) has the \emph{Amalgamation Property} if for every \(A,B  \in  K\), there exists a structure \(D\) 
and embeddings \(h_1:C \rightarrow  D\) and \(h_2:B \rightarrow  D\) such that \(h_1(C) \leq  D\) and \(h_2(B)  \leq  D\)\par{\ForesterRef{Fra-d003}{Amalgamation Property} does NOT necessarily imply JEP, unless the empty set \(\emptyset\) is in \(K\) and for all \(A \in  K\), we have that \(\emptyset   \leq  A\)}
\end{tree}

\begin{tree}{title={Hereditary Property (HP)}, taxon={definition}, slug={Fra-d005}}
A class \((K, \leq )\) has the hereditary property if for every \(A \in  K\), and any \(B \subseteq  A\) finitely generated, then there is some \(C \in  K\) such that \(C \cong  B\)
\end{tree}

\begin{tree}{title={Fraisse's Theorem}, taxon={theorem}, slug={Fra-t001}}
If \(\mathcal {L}\) is countable, and \((K, \subset )\) a class of finitely generated structures has the \ForesterRef{Fra-d003}{amalgamation property}, the \ForesterRef{Fra-d004}{joint embedding property}, and the \ForesterRef{Fra-d005}{hereditary property}, 
Then there is a unique, countable structure \(\mathcal {M}\) whose \ForesterRef{Fra-d002}{age} is \(K\) with the property that any isomorphism \(f:A \rightarrow  B\) with \(A,B \subseteq   \mathcal {M}\) and \(A,B \in  K\) extends to
an automorphism of \(\mathcal {M}\).\par{The structure \(\mathcal {M}\) is the \emph{Fraisse limit} of \(K\).}\par{We call a class \((K, \subseteq )\) satisfying AP, HP, and JEP a Fraisse class}\par{When a structure has the above isomorphism extension property, we say it is \emph{ultrahomogeneous} (or \emph{homogeneous}). 
The converse of Fraisse's Theorem is also true in the sense that if we have a structure which is ultrahomogeneous with respect to its age, then its age is a Fraisse class.}
\end{tree}

\begin{tree}{title={Properties of a Fraisse Limit}, taxon={theorem}, slug={Fra-t002}}
If \(M\) is the \ForesterRef{Fra-t001}{Fraisse Limit} of a class \((K,  \leq )\), then \(Th(M)\) has quantifier elimination and is \(\omega\)-categorical
\end{tree}

\begin{tree}{title={The Random Graph}, taxon={Example}, slug={Fra-e001}}
Rado's famous random graph is indeed the Fraisse Limit of the class of finite graphs in the language \(\mathcal {L} =  \{ E \}\) where \(E\) is a binary relation representing "there is an edge" between two points.\par{More precisely, a graph is an \(\mathcal {L}-\)structure for which \(E\) is anti-reflexive and symmetric.}
\end{tree}

\begin{tree}{title={\(( \mathbb {Q},  \leq )\)}, taxon={Example}, slug={Fra-e002}}
The Dense Linear Order \(( \mathbb {Q},  \leq )\) is the Fraisse Limit of the class of all finite linear orders
\end{tree}

\end{tree}

\begin{tree}{title={Other Properties of Classes}, taxon={section}, slug={Fra-s002}}
We define other common properties and variants of the set up shown in the \ForesterRef{Fra-s001}{Fraisse section}
\begin{tree}{title={Disjoint Amalgamation Property (dAP)}, taxon={definition}, slug={Fra-d006}}
Given a class and relation \((K, \leq )\), we say that \(K\) has the \emph{Disjoint Amalgamation Property} if for every \(A,B,C  \in  K\) such that \(B \cap  C =A\), if there 
exists an embedding \(f:A \rightarrow  B\) such that \(f(A)  \leq  B\) and and embedding \(g:A \rightarrow  C\) such that \(A \leq  B\), then there exists a structure \(D\) 
and embeddings \(h_1:C \rightarrow  D\) and \(h_2:B \rightarrow  D\) such that \(h_1(C) \leq  D\) and \(h_2(B)  \leq  D\), \(h_2(f(A)) = h_1(g(A))\), and \(h_2(C)  \cap  h_1(B) = h_1(A)\)\par{An example of a class which has the dAP is the \ForesterRef{Fra-e001}{Fraisse class of finite graphs}}\par{An example of a smooth class which does not have the dAP is \ForesterRef{Smc-e001}{The class of initial segments}}
\end{tree}

\begin{tree}{title={Ramsey Property on Substructures (RPS)}, taxon={definition}, slug={Fra-d007}}
For structures \(A,B\), let \(\binom {B}{A} =  \{ A_0: A_0 \leq  B & ; A_0 \cong  B \}\)\par{Given a class and relation \((K, \leq )\), we say that \(K\) has the Ramsey property on substructures if for every \(A \leq  B  \in  K\), there is some \(C \in  K\) with \(B \leq  C\)
such that for every \(k\)-coloring \(c:  \binom {C}{A} \rightarrow  k\), there is some \(B'  \in   \binom {C}{B}\) with \[c|_{ \binom {B'}{A}}\] constant.}\par{Ramsey classes are highly related to descriptive set theory, in particular, extremely amenable sets.}\par{Take care not to mix up RPS and \ForesterRef{Fra-d008}{RPE}}
\end{tree}

\end{tree}

\begin{tree}{title={Smooth Classes}, taxon={section}, slug={Smc-s001}}
We define Smooth class constructions and their properties. Sources for this section are the papers On Generic Structures~\cite{Smc-r001} and Stable generic structures~\cite{Smc-r002}
\begin{tree}{title={Smooth Class}, taxon={definition}, slug={Smc-d001}}
We assume the language \(\mathcal {L}\) is countable and only relational.\par{A class \((K, \leq )\) of finite \(\mathcal {L}\)-structures (closed under isomorphism) is called \emph{a smooth class} if \(\leq\) is transitive, \(A \leq  B  \Rightarrow  A \subsetq  B\), and for each \(A \in  K\), there is a set of universal formulas \(\Phi _A\) such that:
\[A \leq  B  \Leftrightarrow  B \models   \Phi _A(A)\]
and we require that \(A \cong  A'  \Leftrightarrow   \Phi _A =  \Phi _{A'}\)}\par{This definition comes from On Generic structures~\cite{Smc-r001}}\par{An alternative definition, or characterization, from Baldwin and Shi~\cite{Smc-r002}, is a class that satisfies the following:}\par{If \(A \in  K\), \(A \leq  A\)}\par{If \(A \leq  B\), then \(A \subseteq  B\)}\par{\(\leq\) is transitive}\par{If \(A \leq  C\), and \(A \subseteq  B \subseteq  C\), then \(A \leq  B\) for \(A,B,C  \in  K\)}\par{\(\emptyset \in  K\) and \(\emptyset   \leq  A\) for all \(A \in  K\)}\par{The only difference between these two characterizations is that in the first definition, we essentially stipulate that \(\leq\) is definable/determined by a 
set of universal formulas. This turns out to be an advantage in working with these classes. Certainly, classes satisfying the first definition will satisfy Baldwin-Shi's. In this way,
it is perhaps wiser to regard the second definition as a characterization as opposed to an actual definition.}
\end{tree}

\begin{tree}{title={Initial Segments}, taxon={example}, slug={Smc-e001}}
 Let \((K, \leq _*)\) be the class of be finite, initial segements of the linear order \(( \omega , \leq )\), where \(A \leq _* B\) is "\(A\) is an initial segment of \(B\)".\par{It is clear that \(\leq\) fits into the definition of a smooth class, simply by the universal formula \(\Phi _A(y_1, \dots , y_n) =  \forall  x(( x=y_1 \lor \dots \lor  x=y_n)  \lor   \bigwedge _i^n x \geq  y_i)\)}\par{This class has \ForesterRef{Fra-d003}{AP} and \ForesterRef{Fra-d004}{JEP}, but not \ForesterRef{Fra-d005}{HP} or \ForesterRef{Fra-d006}{dAP}}
\end{tree}

\begin{tree}{title={Smooth Extension of Fraisse's Theorem}, taxon={theorem}, slug={Smc-t001}}
If \(\mathcal {L}\) is countable, and \((K, \leq )\) is a smooth class of finite structures has the \ForesterRef{Fra-d003}{amalgamation property} and the \ForesterRef{Fra-d004}{joint embedding property}, 
then there is a unique, countable structure \(\mathcal {M}\)  with the following properties:\par{1. Any isomorphism \(f:A \rightarrow  B\) with \(A,B \leq   \mathcal {M}\) and \(A,B \in  K\) extends to
an automorphism of \(\mathcal {M}\).}\par{2. \(\mathcal {M} =  \bigcup ^ \omega _n A_n\) where \(A_n  \leq  A_{n+1}\) and for all \(n\), \(A_n \in  K\)}\par{3. For every \(A \in  K\), there is an embedding \(f:A \rightarrow  M\) of \(A\) into \(M\) such that \(f(A)  \leq  M\)}\par{The structure \(\mathcal {M}\) is the \emph{generic}, or sometimes, the \emph{limit} of \(K\).}\par{An equivalent characterization of a generic is properties 2 and 3 and the following property:}\par{4. If \(A \leq  M\) and \(A \leq  B\) for \(B \in  K\), then there is an isomorphism \(f: B \rightarrow  M\) extending the identity map \(id: A \rightarrow  M\) 
such that \(f(B)  \leq  M\)}
\end{tree}

\begin{tree}{title={Smooth Classes and Saturation}, taxon={theorem}, slug={Smc-t002}}
The following theorem first appears in \href{}{this paper} by Laskowski and Kueker\par{Let \((K,  \leq )\) be a smooth class with a \href{}{generic} \(\mathcal {A}\). If \(\mathcal {A}\) is weakly saturated, then \(\mathcal {A}\) is indeed saturated}
\end{tree}

\begin{tree}{title={Atomic Generic of Smooth Class (Kueker & Laskowski)}, taxon={theorem}, slug={Smc-t003}}
Let \((K, \leq )\) be a smooth class with a generic \(\mathcal {A}\). If for every \(A  \in  K\), \(\Phi _A\) (see \ForesterRef{Smc-d001}{here}) consists of a single universal formula,
then \(\mathcal {A}\) is an atomic model.
\end{tree}

\end{tree}

\begin{tree}{title={Abstract Elementary Classes (AEC)s}, taxon={section}, slug={Fra-s003}}
We define AEC constructions and their properties
\end{tree}

\end{tree}


  
  
\begin{tree}{title={Dividing lines}, taxon={part}, slug={mon-p001}}
Dividing lines play a fundamental role in model theory. They form the basis of Shelah's approach to classification theory and have been a central influence on model theory since their conception. Here is an interactive (though not complete) map of the \href{https://forkinganddividing.com}{model theoretic universe}, maintained by Gabriel Conant. Important examples of dividing lines include stability, NIP, and o-minimality.
  
  
\begin{tree}{title={Stability Theory}, taxon={section}, slug={mon-s002}}
\textbf{Stable theories}
\begin{tree}{title={Forking and Dividing}, taxon={subsection}, slug={mon-s003}}
\textbf{Forking and Dividing}
\end{tree}

\end{tree}


    
    
\begin{tree}{title={NIP Theories}, taxon={section}, slug={mon-s001}}
\textbf{NIP theories} are a class of theories generalizing stable
theories, but allowing for an ordering. Aside from stable theories,
important examples are real closed fields, ACVF, \(p\)-adically closed
fields, and o-minimal theories\par{Fix a complete theory \(T\) with monster model \(\mathbb {M}\).
Also fix a formula \(\varphi (x;y)\) with a fixed partitioning into
the two tuples \(x\) and \(y\).}
\begin{tree}{title={Definition of the independence property}, taxon={definition}, slug={mon-d004}}
The formula \(\varphi (x;y)\) has the \emph{independence property} if there
are sequences of tuples \((a_i : i  \in   \omega )\) and
\((b_S : S  \subseteq   \omega )\) such that for every subset \(S  \subseteq   \omega\)
\[i  \in  S  \Longleftrightarrow   \mathbb {M}  \models   \varphi (a_i; b_S).\]
\end{tree}

\begin{tree}{title={Definition of NIP}, taxon={definition}, slug={mon-d005}}
The formula \(\varphi (x;y)\) is said to be \emph{NIP} if it does not have
the \href{mon-0004}{independence property}.
\end{tree}

\begin{tree}{title={Characterizations of NIP for a formula}, taxon={theorem}, slug={mon-t001}}
The following conditions are equivalent:
\begin{enumerate}
\item{ The formula \(\varphi (x;y)\) has the independence property.}
  \item{ The formula \(\varphi ^{ \vee }(y;x)\) has the independence property,
	where \(\varphi ^{ \vee }(y;x)\) is the formula \(\varphi (x;y)\) with the
	opposite partition.}
	\item{ For any two finite sets \(U\) and \(V\), and any subset \(R  \subseteq  U  \times  V\), there are \((a_i : i  \in  U)\) and \((b_j : j  \in  U)\) such that \(\mathbb   \models   \varphi (a_i;b_j)  \Longleftrightarrow  (i,j)  \in  R\).}
	\item{ There is an indiscernible sequence \((a_i : i  \in   \omega )\)
	and some \(b\) such that \(\mathbb  M  \models   \varphi (a_i;b)  \Longleftrightarrow  i\)
	is even.}
\end{enumerate}\par{We are also interested in the following characterization, which is more amenable to computations.}
\begin{tree}{title={Alternation Number}, taxon={definition}, slug={mon-d007}}
The \emph{alternation number} of a formula \(\varphi (x;y)\), denoted \(\operatorname {alt}( \varphi (x;y))\) is the maximal number \(n  \in   \omega\) (if it exists) such that there is an indiscernible sequence \((a_i : i  \in   \omega )\), some \(b\), and indices \(i_0 <  \dots  < i_n\) with \(\mathbb  M  \models   \varphi (a_i,b)  \Longleftrightarrow  i  \text { is even}\). If no such maximum exists, we let \(\operatorname {alt}( \varphi (x;y)) =  \infty\).
\end{tree}

\begin{tree}{title={Alternation Lemma}, taxon={lemma}, slug={mon-d008}}
A formula \(\varphi (x;y)\) is NIP if and only if \(\operatorname {alt}( \varphi (x;y)) <  \infty\).
\end{tree}

\end{tree}

\end{tree}


\end{tree}


  
  
\begin{tree}{title={Infinitary Model Theory}, taxon={part}, slug={inf-p001}}
We present the fundamentals of infinitary model theory. The main reference will be David Marker's book Lectures on Infinitary Logic
\end{tree}


  
  
\begin{tree}{title={Monadic Expansions}, taxon={part}, slug={moe-p001}}
Model Theory of Monadic Expansions
  
  
\begin{tree}{title={Monadic NFCP}, taxon={section}, slug={moe-s001}}

\begin{tree}{title={Monadic Expansion of a Language}, taxon={definition}, slug={moe-d001}}
Let \(L\) be any \ForesterRef{BMT-d001}{language}. Then a monadic expansion of \(L\) is a language \(L':= L \cup \{ R_i \} _{i \in  I}\), where \(\{ R_i \} _{i \in  I}\) is a collection of unary relation symbols.
\end{tree}

\begin{tree}{title={Monadically NFCP}, taxon={definition}, slug={moe-d002}}
Let \(M\) be an \(L\)-\ForesterRef{BMT-d002}{structure}. Then \(M\) is monadically NFCP if it is NFCP under every \ForesterRef{moe-d001}{monadic expansion of \(L\)}.
\end{tree}

\begin{tree}{title={Mutually Algebraic Sets}, taxon={definition}, slug={moe-d003}}
Let \(X\) be a non-empty set and \(l \geq1\). Then a subset \(C \subseteq  X^l\) is mutually algebraic if there exists some \(K\) such that for all \(a \in  X\), we have \(| \{ \overline {c} \in  C: a \in \overline {c} \} | \leq  K\).
\end{tree}

\begin{tree}{title={Mated Pairs}, taxon={example}, slug={moe-e001}}
For \(l\)=2, if \(C \subseteq  X^2\) is a set of mated pairs, i.e. every element in \(X\) has a unique "mate," which is symmetric, then \(C\) is \ForesterRef{moe-d003}{mutually algebraic}.
\end{tree}

\begin{tree}{title={Mutually Algebraic Formulas}, taxon={definition}, slug={moe-d004}}
Suppose \(M\) is an \(L\)-\ForesterRef{BMT-d002}{structure} and \(\phi ( \overline {z})\) is an \(L(M)\) definable set. Then \(\phi\) is a mutually algebraic formula if \(D= \phi ( \overline {z})\) is a \ForesterRef{moe-d003}{mutually algebraic subset} of \(M^{lg( \overline {z})}\).
\end{tree}

\end{tree}


\end{tree}


\printbibliography
\end{document}