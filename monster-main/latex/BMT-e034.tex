\documentclass[a4paper]{article}
\usepackage[final]{microtype}
\usepackage{fontspec}
\setmonofont{inconsolata}
\usepackage{amsmath,amsthm,amssymb,stmaryrd,mathtools,biblatex,forester}
\addbibresource{forest.bib}

\title{Graphs (Rado Graph)}

\date{October 10, 2024}

\author{Oscar Coppola}

\begin{document}
\maketitle
\par{
    The theory of a graph is a \ForesterRef{BMT-d017}{theory} in the \ForesterRef{BMT-d001}{language} \(\mathcal  L= \{ R \}\), where \(R\) is a binary relation.
    Depending on the author, the theory may only demand symmetry, \(\forall  x \forall  y \, xRy \rightarrow  yRx\),
    or additionally include irreflexivity: \(\forall  x \, \neg  xRx\).
}\par{
    The theory of the \emph{Rado graph} (also called the random graph) is an extension of the theory of the graph by the extension axiom schema:
    \begin{enumerate}
\item{
            For each \(n\) and \(m\), the sentence
            \(\forall  x_1 \cdots \forall  x_n \forall  y_1 \cdots \forall  y_m \exists  z \, \bigwedge _{i=1}^nzRx_i \land \bigwedge _{j=1}^m \neg  zRy_j\).
        }
\end{enumerate}
    This ``single'' schema is enough to make the theory \ForesterRef{BMT-d017}{complete},
    and in fact the theory of the Rado graph is \ForesterRef{BMT-d201}{\(\aleph _0\)-categorical}.
}
\printbibliography
\end{document}