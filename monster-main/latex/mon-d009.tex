\documentclass[a4paper]{article}
\usepackage[final]{microtype}
\usepackage{fontspec}
\setmonofont{inconsolata}
\usepackage{amsmath,amsthm,amssymb,stmaryrd,mathtools,biblatex,forester}
\addbibresource{forest.bib}

\title{Elimination of Imaginaries}

\date{November 7, 2024}

\author{Ruohan Hu}

\begin{document}
\maketitle
\par{
    A model \(M\) has elimination of imaginaries, when given a formula \(\theta ( \bar {x}, \bar {y})\) where \(l( \bar {x})=l( \bar {y})=n\) that defines an equivalence relation on \(M^n\), for every equivalence class \(\bar {a}/ \theta\) represented by \(\bar {a}\), there is a formula \(\phi\), such that the equivalence class \(\bar {a}/ \theta\) is defined by \(\phi ( \bar {x}, \bar {b})\), where \(\bar {b}\) is unique given \(\phi\) and \(\bar {a}/ \theta\).

    A theory \(T\) is said to have elimination of imaginaries, if every model \(M\) of \(T\) has elimination of imaginaries.
}
\printbibliography
\end{document}