\documentclass[a4paper]{article}
\usepackage[final]{microtype}
\usepackage{fontspec}
\setmonofont{inconsolata}
\usepackage{amsmath,amsthm,amssymb,stmaryrd,mathtools,biblatex,forester}
\addbibresource{forest.bib}

\title{Elimination of Atomic Formulas}

\date{February 27, 2024}

\author{Connor Lockhart}

\begin{document}
\maketitle
\par{Let \(K\) be a class of \(L\) structures and \(\Phi\) be a set of \(L\)-formulas. Denote \(\Phi ^-\) as the set of negations of formulas in \(\Phi\).}\par{Suppose that 
  \begin{enumerate}
\item{every atomic formula of \(L\) is in \(\Phi\) and} 
    \item{for every formula \(\theta ( \overline {x})\) of \(L\) which is of the form \(\exists  y  \bigwedge _{i<n}  \psi _i ( \overline {x},y)\) with each \(\psi _i  \in   \Phi \cup   \Phi ^-\), there is an \(L\)-formula \(\theta ^*( \overline {x})\) which is a boolean combination of formulas in \(\Phi\) and is equivalent to \(\theta\) in every structure in \(K\).}
\end{enumerate}
  Then \(\Phi\) is an elimination set of \(K\)}
\printbibliography
\end{document}