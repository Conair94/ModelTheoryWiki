\documentclass[a4paper]{article}
\usepackage[final]{microtype}
\usepackage{fontspec}
\setmonofont{inconsolata}
\usepackage{amsmath,amsthm,amssymb,stmaryrd,mathtools,biblatex,forester}
\addbibresource{forest.bib}

\title{Universal Model}

\date{October 17, 2024}

\author{Ruohan Hu}

\begin{document}
\maketitle
\par{
    A model \(M\) is \(\lambda\)-Universal, per Chang & Keisler, when for every model \(N\) such that \(|N|< \lambda\) that is elementarily equivalent to \(M\) can \ForesterRef{BMT-d030}{elementarily embeds} into \(M\). 
}\par{
    A model \(M\) is \(\lambda\)-Universal, per Shelah, when for every model \(N\) that is elementarily equivalent to \(M\), and a set \(A \subseteq  N\) such that \(|A| \leq \lambda\), there is a \ForesterRef{BMT-d030}{partial elementary embedding} \(f:M \to  N\) such that \(dom(f)=A\).
}\par{\(\lambda\)-universality per Shelah implies \(\lambda ^+\)-universality per Chang & Keisler.
    By the Lowenheim-Skolem Theorem, if the language \(|L| \leq   \lambda\), then \(\lambda\)-universality per Shelah and \(\lambda ^+\)-universality per Chang & Keisler are equivalent.
}\par{
    Per Shelah, a model \(M\) is \(< \lambda\)-universal if it is \(\mu\)-universal for all \(\mu < \lambda\), and a model \(M\) simply said to be universal if it is \(|M|\)-universal. 
}
\printbibliography
\end{document}