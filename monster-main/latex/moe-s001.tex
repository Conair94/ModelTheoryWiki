\documentclass[a4paper]{article}
\usepackage[final]{microtype}
\usepackage{fontspec}
\setmonofont{inconsolata}
\usepackage{amsmath,amsthm,amssymb,stmaryrd,mathtools,biblatex,forester}
\addbibresource{forest.bib}

\title{Monadic NFCP}

\date{May 2, 2024}

\author{Connor Lockhart \and Garrett Peters}

\begin{document}
\maketitle

\begin{tree}{title={Monadic Expansion of a Language}, taxon={definition}, slug={moe-d001}}
Let \(L\) be any \ForesterRef{BMT-d001}{language}. Then a monadic expansion of \(L\) is a language \(L':= L \cup \{ R_i \} _{i \in  I}\), where \(\{ R_i \} _{i \in  I}\) is a collection of unary relation symbols.
\end{tree}

\begin{tree}{title={Monadically NFCP}, taxon={definition}, slug={moe-d002}}
Let \(M\) be an \(L\)-\ForesterRef{BMT-d002}{structure}. Then \(M\) is monadically NFCP if it is NFCP under every \ForesterRef{moe-d001}{monadic expansion of \(L\)}.
\end{tree}

\begin{tree}{title={Mutually Algebraic Sets}, taxon={definition}, slug={moe-d003}}
Let \(X\) be a non-empty set and \(l \geq1\). Then a subset \(C \subseteq  X^l\) is mutually algebraic if there exists some \(K\) such that for all \(a \in  X\), we have \(| \{ \overline {c} \in  C: a \in \overline {c} \} | \leq  K\).
\end{tree}

\begin{tree}{title={Mated Pairs}, taxon={example}, slug={moe-e001}}
For \(l\)=2, if \(C \subseteq  X^2\) is a set of mated pairs, i.e. every element in \(X\) has a unique "mate," which is symmetric, then \(C\) is \ForesterRef{moe-d003}{mutually algebraic}.
\end{tree}

\begin{tree}{title={Mutually Algebraic Formulas}, taxon={definition}, slug={moe-d004}}
Suppose \(M\) is an \(L\)-\ForesterRef{BMT-d002}{structure} and \(\phi ( \overline {z})\) is an \(L(M)\) definable set. Then \(\phi\) is a mutually algebraic formula if \(D= \phi ( \overline {z})\) is a \ForesterRef{moe-d003}{mutually algebraic subset} of \(M^{lg( \overline {z})}\).
\end{tree}

\printbibliography
\end{document}