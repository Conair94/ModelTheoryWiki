\documentclass[a4paper]{article}
\usepackage[final]{microtype}
\usepackage{fontspec}
\setmonofont{inconsolata}
\usepackage{amsmath,amsthm,amssymb,stmaryrd,mathtools,biblatex,forester}
\addbibresource{forest.bib}

\title{Axioms of Set Theory}

\date{February 17, 2024}

\author{Morgan Bryant}

\begin{document}
\maketitle
\par{We describe the different set theory axioms and different accepted sets of these axioms, such as ZFC, ZFC+, etc.}
\begin{tree}{title={Axiom of Existentionality (AE)}, taxon={definition}, slug={Set-d001}}
The axiom of existentionality is as follows:\par{\(\forall  x  \forall  y( x = y  \Leftrightarrow  ( \forall  z (z \in  x  \leftrightarrow  z \in  y)))\)}
\end{tree}

\begin{tree}{title={Axiom of Pairing (PA)}, taxon={definition}, slug={Set-d002}}
The Pairing Axiom is as follows:\par{\(\forall  x  \forall  y  \exists  z ( \forall  w(w \in  z  \leftrightarrow  (w= x \lor  w=y)))\)}
\end{tree}

\begin{tree}{title={Axiom of Union (UA)}, taxon={definition}, slug={Set-d003}}
The Union Axiom is as follows:\par{\(\forall  x  \exists  z( \forall  y( y \in  z  \leftrightarrow  ( \exists  w  \in  x (y \in  w))))\)}
\end{tree}

\begin{tree}{title={Axiom of Seperation (SA)}, taxon={definition}, slug={Set-d004}}
The axiom of seperation, or the seperation schema, is as follows:\par{If \(\varphi (x_1, \dots , x_n,z)\) is any formula in the appropriate language, then there exists a set \( y =  \{ z:  \varphi (x_1, \dots , x_n, z)\)}
\end{tree}

\printbibliography
\end{document}