\documentclass[a4paper]{article}
\usepackage[final]{microtype}
\usepackage{fontspec}
\setmonofont{inconsolata}
\usepackage{amsmath,amsthm,amssymb,stmaryrd,mathtools,biblatex,forester}
\addbibresource{forest.bib}

\title{Dividing lines}

\date{February 2, 2024}

\author{Adam Melrod\thanks{With contributions from Connor Lockhart, Ruohan Hu.}}

\begin{document}
\maketitle
\par{Dividing lines play a fundamental role in model theory. They form the basis of Shelah's approach to classification theory and have been a central influence on model theory since their conception. Here is an interactive (though not complete) map of the \href{https://forkinganddividing.com}{model theoretic universe}, maintained by Gabriel Conant. Important examples of dividing lines include stability, NIP, and o-minimality.}
  
  
\begin{tree}{title={Stability Theory}, taxon={section}, slug={mon-s002}}
\textbf{Stable theories}
\begin{tree}{title={Indiscernible Sequence}, taxon={definition}, slug={mon-d014}}


Given a cardinal \(\kappa\), the sequence of \(n\)-tuples of elements in a model \(M\), \(\{ \overline {b_i} \} _{i< \kappa }\) is an indiscernibles sequence in \(M\) over \(A \subseteq  M\) means that: for any \(k< \omega\) and indices \(i_1,...,i_k\) and \(j_1,...,j_k\), and any formula \(\varphi ( \overline {x_1},..., \overline {x_k}, \overline {a})\) where \(\overline {a}\) is a tuple of parameters from \(A\), we have \(M \models   \varphi ( \overline {b_{i_1}},..., \overline {b_{i_k}}, \overline {a}) \leftrightarrow   \varphi ( \overline {b_{j_1}},..., \overline {b_{j_k}}, \overline {a})\).

In other words, any two finite subsequence of \(\{ \overline {b_i} \} _{i< \kappa }\) of the same length have the same \ForesterRef{BMT-d019}{type} over \(A\), i.e, \(tp(( \overline {b_{i_1}},..., \overline {b_{i_k}})/A)=tp(( \overline {b_{j_1}},..., \overline {b_{j_k}})/A)\). 

Notice that the above definition, the index of the sequence, i.e. the ordinal \(\omega\), can be substituted by any other ordinal, or just any linear order. This would define another kind of "indiscernible seuqnece," but we still refer to it with the same term.

A set of \(n\)-tuples is an Indiscernible set if it is an indiscernible sequence under any well-order over the set.

\end{tree}

\begin{tree}{title={Forking and Dividing}, taxon={subsection}, slug={mon-s003}}
\textbf{Forking and Dividing}
\begin{tree}{title={Dividing}, taxon={definition}, slug={mon-d012}}

The formula \(\varphi ( \overline {x}, \overline {a})\) (with parameter \(\overline {a}\)) \emph{divides} over \(A\), if there are \(n< \omega\) and sequence \(\{ \overline {a_i}:i< \omega \}\) such that:
\begin{enumerate}
\item{\(tp( \overline {a},A)=tp( \overline {a_i},A)\);}
    \item{\(\{ \varphi (x, \overline {a_i}):i< \omega \}\) is \(n\)-inconsistent.}
\end{enumerate}
Usually we have \(\bar {a} \not \subset  A\), because obviously \(\varphi ( \bar {x}, \bar {a})\) does not divide over \(A\) if \(\bar {a} \subseteq  A\).

We say a type \(\mathfrak {p}\) divides over \(A\), if it implies a formula dividing over \(A\).

\end{tree}

\begin{tree}{title={Forking}, taxon={definition}, slug={mon-d013}}
\ForesterRef{BMT-d019}{Type} \(p \in  S(B)\) forks over \(A\) if for some finite \(n\), and formulas \(\varphi _k( \overline {x}, \overline {a_{k}})\) where \(\overline {a_k} \subset  B\), we have \(p \vdash \bigvee _{k<n} \varphi _k( \overline {x}, \overline {a_{k}})\) where each \(\varphi _k( \overline {x}, \overline {a_{k}})\) divides over \(A\), and \(\bar {x}\) consist of some of the variables in \(p\)/

Usually, we have \(p \in  S(B)\) when \(B \not \subseteq  A\), because if \(B \subseteq  A\) then we will see that that \(p\) does not fork over \(A\). Moreover, we often takes \(A \subseteq  B\).

Notice that forking can also be defined for types over infinitely many variables.

\end{tree}

\end{tree}

\end{tree}


    
    
\begin{tree}{title={NIP Theories}, taxon={section}, slug={mon-s001}}
\textbf{NIP theories} are a class of theories generalizing stable
theories, but allowing for an ordering. Aside from stable theories,
important examples are real closed fields, ACVF, \(p\)-adically closed
fields, and o-minimal theories\par{Fix a complete theory \(T\) with monster model \(\mathbb {M}\).
Also fix a formula \(\varphi (x;y)\) with a fixed partitioning into
the two tuples \(x\) and \(y\).}
\begin{tree}{title={Definition of the independence property}, taxon={definition}, slug={mon-d004}}
The formula \(\varphi (x;y)\) has the \emph{independence property} if there
are sequences of tuples \((a_i : i  \in   \omega )\) and
\((b_S : S  \subseteq   \omega )\) such that for every subset \(S  \subseteq   \omega\)
\[i  \in  S  \Longleftrightarrow   \mathbb {M}  \models   \varphi (a_i; b_S).\]
\end{tree}

\begin{tree}{title={Definition of NIP}, taxon={definition}, slug={mon-d005}}
The formula \(\varphi (x;y)\) is said to be \emph{NIP} if it does not have
the \href{mon-0004}{independence property}.
\end{tree}

\begin{tree}{title={Characterizations of NIP for a formula}, taxon={theorem}, slug={mon-t001}}
The following conditions are equivalent:
\begin{enumerate}
\item{ The formula \(\varphi (x;y)\) has the independence property.}
  \item{ The formula \(\varphi ^{ \vee }(y;x)\) has the independence property,
	where \(\varphi ^{ \vee }(y;x)\) is the formula \(\varphi (x;y)\) with the
	opposite partition.}
	\item{ For any two finite sets \(U\) and \(V\), and any subset \(R  \subseteq  U  \times  V\), there are \((a_i : i  \in  U)\) and \((b_j : j  \in  U)\) such that \(\mathbb   \models   \varphi (a_i;b_j)  \Longleftrightarrow  (i,j)  \in  R\).}
	\item{ There is an indiscernible sequence \((a_i : i  \in   \omega )\)
	and some \(b\) such that \(\mathbb  M  \models   \varphi (a_i;b)  \Longleftrightarrow  i\)
	is even.}
\end{enumerate}\par{We are also interested in the following characterization, which is more amenable to computations.}
\begin{tree}{title={Alternation Number}, taxon={definition}, slug={mon-d007}}
The \emph{alternation number} of a formula \(\varphi (x;y)\), denoted \(\operatorname {alt}( \varphi (x;y))\) is the maximal number \(n  \in   \omega\) (if it exists) such that there is an indiscernible sequence \((a_i : i  \in   \omega )\), some \(b\), and indices \(i_0 <  \dots  < i_n\) with \(\mathbb  M  \models   \varphi (a_i,b)  \Longleftrightarrow  i  \text { is even}\). If no such maximum exists, we let \(\operatorname {alt}( \varphi (x;y)) =  \infty\).
\end{tree}

\begin{tree}{title={Alternation Lemma}, taxon={lemma}, slug={mon-d008}}
A formula \(\varphi (x;y)\) is NIP if and only if \(\operatorname {alt}( \varphi (x;y)) <  \infty\).
\end{tree}

\end{tree}

\end{tree}


    
    
\begin{tree}{title={Elimination of Imaginaries}, taxon={section}, slug={mon-s004}}

\begin{tree}{title={Elimination of Imaginaries}, taxon={definition}, slug={mon-d009}}

    A model \(M\) has elimination of imaginaries, when given a formula \(\theta ( \bar {x}, \bar {y})\) where \(l( \bar {x})=l( \bar {y})=n\) that defines an equivalence relation on \(M^n\), for every equivalence class \(\bar {a}/ \theta\) represented by \(\bar {a}\), there is a formula \(\phi\), such that the equivalence class \(\bar {a}/ \theta\) is defined by \(\phi ( \bar {x}, \bar {b})\), where \(\bar {b}\) is unique given \(\phi\) and \(\bar {a}/ \theta\).

    A theory \(T\) is said to have elimination of imaginaries, if every model \(M\) of \(T\) has elimination of imaginaries.

\end{tree}

\begin{tree}{title={Uniform Elimination of Imaginaries}, taxon={definition}, slug={mon-d010}}

    A model \(M\) has uniform elimination of imaginaries, if, in addition to having elimination of imaginaries, the formula \(\phi\) is independent of \(\bar {a}\) and is dependent only on \(\theta\). 
    
    Equivalently, we can say $M$ has uniform elimination of imaginaries if there is a $0$-definable function \(f\), such that \(f( \bar {a})= \bar {b}\) where \(\bar {b}\) is the unique tuple that defines \(\bar {a}/ \theta\).

\end{tree}

\begin{tree}{title={Weak Elimination of Imaginaries}, taxon={definition}, slug={mon-d011}}
  
    A model \(M\) has weak elimination of imaginaries, when given a formula \(\theta ( \bar {x}, \bar {y})\) where \(l( \bar {x})=l( \bar {y})=n\) that defines an equivalence relation on \(M^n\), for every equivalence class \(\bar {a}/ \theta\), there is a formula \(\phi\), and a finite set of tuples \(X\) in \(M\) such that the equivalence class \(\bar {a}/ \theta\) is defined by \(\phi ( \bar {x}, \bar {b})\) if and only if \(\bar {b} \in  X\).
    
    A theory \(T\) is said to have weak elimination of imaginaries, if every model \(M\) of \(T\) has weak elimination of imaginaries.

\end{tree}

\end{tree}


\printbibliography
\end{document}